\subsection{Methodologies}
\label{sec:lfd}
This section is dedicated to present and analyze various approaches for solving the LfD problem described in Section~\ref{sec:problem_formulation}. Specifically,  the proposed taxonomy is derived from studying different review papers~\cite{kaelbling1996reinforcement_survey,argall2009robot_learning_from_demonstration,hussein2017imitation_learning_survey,fang2019survey,zheng2021imitation_progress_taxonomies_opportunities,zare2024survey}. Figure~\ref{fig:il_taxonomy} provides a graphical representation of the proposed taxonomy. The methods are first categorized based on the type of demonstration, either \textit{State-Action} or \textit{State-only}, followed by an \\ overview of the different methodologies. The proposed taxonomy highlights the learning algorithm and main components for each methodology.

This chapter is divided into the following sections. Section~\ref{sec:bc} reviews methods for the fully-supervised learning methodology known as \textit{Behavioral Cloning}. 
\newline Section~\ref{sec:irl} discusses methods under the umbrella of \textit{Inverse Reinforcement Learning}, which solve the inverse optimization problem by first learning the reward function and then using it to guide policy optimization following a reinforcement paradigm.
\newline Section~\ref{sec:gail} reviews methods leveraging the concept of \textit{Generative Adversarial Learning} to optimize policy parameters and learn demonstrated behaviors, classified as Generative Adversarial Imitation Learning methods.
\newline Finally, Section~\ref{sec:lfo} covers the most recent methodology, \textit{Learning from Observation}, characterized by optimizing the learner policy using state-only demonstrations.
\begin{figure}[tb]
    
    \includegraphics[width=\textwidth]{figures/images/il_taxonomy.png}
    \caption{Taxonomy of LfD methods, divided based on type of demonstration and the learning algorithm used to learn the learner policy $\pi^{L}$ }
    \label{fig:il_taxonomy}
    
\end{figure}


Each paragraph will present the research in the same way, i.e., the proposed works will be presented in a temporal order, highlighting the evolution of techniques and approaches over time.
It is important to note that with respect to the problem managed in this thesis, the most relevant and most related literature is the one presented in Section \ref{sec:bc}. However, for sake of clarity and completeness, all the other approaches will be described with their pros/cons, with some considerations with respect to the proposal of this thesis.


\subsubsection{Behavioral Cloning}
\label{sec:bc}
\begin{algorithm}[t]
\caption{Abstract Algorithm for BC methods}\label{alg:bc}
\begin{algorithmic}
\REQUIRE A set of expert demonstrations $\mathcal{D}^{E}$, a parameterized policy $\pi_{\theta}^{L}$
\ENSURE The optimal set of policy parameter $\theta^{*}$
\STATE Optimize $\mathcal{L}$ w.r.t.\ policy parameter $\theta$ using $\mathcal{D}^{E}$
\end{algorithmic}
\end{algorithm}
\textit{Behavioral Cloning} is one of the first approaches used to solve the LfD problem~\cite{pomerleau1988alvinn}. The high-level \textbf{supervised-learning} procedure followed by BC methods is outlined in Algorithm~\ref{alg:bc}. Generally, BC methods take as input the expert demonstration dataset $\mathcal{D}^{E}$ and a learner policy modeled as a parameterized function $\pi_{\theta}^{L}$. The parameters $\theta$ can be either the weights of a neural network~\cite{pomerleau1988alvinn} or the parameters of a dynamic system~\cite{ijspeert2002learning}. As a supervised approach, the goal is to find the optimal parameters $\theta^{*}$ that can replicate the \textbf{ground-truth behaviors} contained in the dataset $\mathcal{D}^{E}$. This is achieved by solving an optimization problem, which can generally be described by Formula~\ref{eq:bc_formula}.
\begin{equation}
 \label{eq:bc_formula}
 \theta^{*} = \underset{\theta}{argmin} \ \mathbb{E}_{(\boldsymbol{\tau}, c) \sim \mathcal{D}^{E}} \ [\mathcal{L}((\boldsymbol{\tau}, c), \ \pi^{L}_{\theta})]
\end{equation}


In the following section, different ways in which the general optimization problem described by Formula~\ref{eq:bc_formula} is formulated and solved will be discussed.

\paragraph*{Dynamical Movement Primitives}\mbox{}\\
% Dynamical Movement Primitives (DMPs), offer a robust framework for encoding and reproducing complex movements through differential equations and attractor dynamics.
The \textit{Dynamical-Movement Primitives} (DMPs) methods are among the first successful applications of the BC methodology to the LfD problem. Their success is attributed to their ease of implementation and efficiency in learning. DMPs do not require learning or estimating the system dynamics, nor do they require a reward function or interaction with the environment during the learning procedure, as they are supervised learning methods.

DMPs were first formalized in \cite{ijspeert2002learning}. They derive the policy directly in the trajectory space, allowing for explicit modeling of constraints such as smooth convergence toward the goal state. The core idea behind DMPs, as proposed in \cite{ijspeert2002learning,ijspeert2013dynamical}, is to model the trajectory as a \textbf{point-to-point attractor system}, described by the set of differential equations in Formula~\ref{eq:dmp}.
\begin{equation}
    \begin{matrix}
        \tau \dot{y} = \beta_{s}(\alpha_{s}(g-s)- y) + f(z) \\
        \\
         \tau \dot{s} = y\\
         \\
         \tau \dot{z} = -\alpha_{z} z, \ z(t) = z_{0} \ exp(-\frac{\alpha_{z}}{\tau} t)
        \end{matrix}
    \label{eq:dmp}
\end{equation}


Here, $\beta_{s}$, $\alpha_{s}$, and $\alpha_{z}$ are constants, $s$ is the system state, $z$ is the phase-variable function of time $t$, and $f$ is the forcing term that describes the trajectory's non-linear behavior. Generally, $f$ is a linear combination of basis functions $\psi_{i}(z)$ (e.g., Gaussian basis functions), such that $f(z(t)) = (g - s_{0}) \sum_{i=1}^{M} \psi_{i}(z(t)) \omega_{i} z$. Essentially, a DMP describes a point-attractor system where the current system state $s$ must converge to the goal state $g$, starting from $s_{0}$. In this context, the aim is to learn the set of weights $\{\omega_{i}, i=1,\dots,M\}$, which can be obtained by solving a supervised learning problem with the loss function described in Formula~\ref{eq:dmp_loss}.
\begin{equation}
    \mathcal{L}_{DMP} = \sum_{t=0}^{T}(f_{target}(t) - f(z(t)))^{2}
    \label{eq:dmp_loss}
\end{equation}

The function $f_{target}(t)$ is equal to $f_{target}(t) = \tau^{2} \ \ddot{s}^{E}_{t} - \beta_{s}(\alpha_{s}(g - s^{E}_{t})-\tau \dot{s}^{E}_{t})$ represents the evolution of the expert state $s^{E}_{t}$ towards the goal state $g$, and represents the dynamic to mimic through the learned linear combination of basis function $f(z_{t})$. 

The initial DMPs formulation proposed in \cite{ijspeert2002learning} has several issues that can be categorized as follows:
\begin{itemize}
    \item \textbf{Handling stochasticity in demonstrations}: Different demonstrations can vary slightly due to differences in demonstrators, task completion methods, speeds, and paths. This variability creates a distribution in the demonstration space, requiring a method to manage it.
    
    \item \textbf{Defining different basis function}: In DMPs, the weights $\omega_{i}$ of the basis functions are learned. However, the force term can also be defined using other formalisms, such as Gaussian Mixture Models \cite{si2023composite}, Neural-Network Radial Basis Functions \cite{li2023human}, or Gaussian Processes \cite{fanger2016gaussian}. Therefore, the choice of how to model the force term is a hyper-parameter of the problem.

    \item \textbf{Managing arbitrary desired trajectories with intermediate via-points}: Once the behavior encoded in the demonstration is learned, generating novel trajectories that pass through new points (possibly defined by a human agent) is not possible. Therefore, a method to generalize to different waypoints is needed.
    
    \item \textbf{Handling high-dimensional inputs}: To use the DMPs algorithm, it is necessary to work in the robot space, recording joint and gripper trajectories through teleoperation or kinesthetic teaching. However, in complex scenarios involving interaction with objects that do not have fixed initial positions, it becomes essential to infer the initial object state from high-level inputs, such as images.
\end{itemize}
To address these drawbacks, several solutions have been proposed. Notably, the authors in \cite{paraschos2013ProMPs} introduced the \textit{Probabilistic Movement Primitives} (ProMPs) framework. This probabilistic framework offers an alternative movement primitive representation, capturing the variability across different demonstrations and degrees of freedom (DoFs) through a covariance matrix. Specifically, the trajectory $\mathbf{\tau}$ is modeled as a distribution: $\boldsymbol{\tau} = \underset{t}{\prod} \ \mathcal{N}(s(t)|\Psi(z(t))^{T}\omega, \Sigma_{s})$, where $\Psi$ is a time-dependent basis matrix.

Generally, modeling the problem in probabilistic terms has several advantages, particularly the ability to generalize to new goals by conditioning the learned distribution on a given novel goal state \cite{saveriano2023dynamic}.

About the possibility to manage arbitrary desired trajectories, authors in \cite{zhou2019learning} proposed a novel framework for learning movement primitives, named \textit{Via-points Movement Primitives} (VMP). This is basically an extension of both DMPs (that can only adapt to new starts and goals, but cannot directly handle intermediate via-points) and ProMPs (that can adapt to via-points within the statistical distribution of the demonstrated trajectories).
\textcolor{red}{ToContinue}

Despite all the successfully applications saw previsously, DMPs and all the variants have a very relevant limitation, that is related to the difficulty of handling high-dimensional input such as images. For this reason, the scientific community has focused the attention on methods that leverage deep architecture, that will be explained in detail in the following paragraphs.

\paragraph*{Single-Task Imitation Learning}\mbox{}\\
This paragraph will review the research conducted in the context of Single-Task Imitation Learning. Specifically, within the scope of robotic manipulation problems, the term ``Single-Task" indicates that the learned policy $\pi^{L}$ can perform only the specific task it has been trained on. For example, if the task involves a pick-and-place operation with a fixed place position, the model cannot handle variations in the place location. Additionally, the focus will be primarily on methods that use high-level state representations, such as images, processed by deep architectures to solve the problem.

In this scenario, the scientific literature extends far back in time. One of the seminal works in this field was proposed in 1988 by Pomerleau, who introduced \textit{ALVINN} \cite{pomerleau1988alvinn}. ALVINN is an autonomous vehicle driving system based on a Neural Network that predicts the steering angle from a synthetic camera image input. The network was trained on pairs of (image, steering angle), with the training procedure framed as a supervised classification problem. This was achieved by discretizing the steering angle into 45 units. Pomerleau's work immediately highlighted the issue of \textbf{compounding error}, which arises from the \textbf{covariate shift phenomenon}. This issue occurs because an action $a_{t}$ influences the subsequent state $s_{t+1}$, which becomes the next sample, thereby violating the i.i.d. assumption of Supervised Learning. This results in a test-data distribution that may differ from the training one. This phenomenon has significant consequences on the expected performance of the system and is addressed by methods discussed in the section on \textit{Interactive Imitation Learning}.

Despite the covariate-shift problem, \cite{zhang2018deep_vr_teleoperation} showed that very interesting performance can be obtained in the context of Robot Manipulation, by means of Behavioral Cloning and high quality demonstrations given by teleportation system. In this work, a Convolutional Neural Network was trained to predict the desired linear-velocity, angular-velocity of the end-effector, and the binary gripper state (open/close), given in input the current RGB-D observation of the scene, and the position of three points of the end-effector, during the last 5 time-steps (Figure \ref{fig:deep_bc}). The system was tested on 10 tasks, and the performance are reported in Table \ref{table:deep_vr_teleoperation_results}. The proposed system achieved a high success rate while evaluating all the tasks. The tests were carried out from different initial conditions but still quite similar to those present in the training set (e.g., the initial object positions have been uniformly distributed within the training regime, with random local variations around these positions). The analysis of failure cases showed that the leading cause of errors was the inability to detect critical points in the task execution, such as closing/opening the gripper to pick/place the object or detect the position of the object of interest in order to avoid collision with it.
% What Matters in Learning from Offline Human Demonstrations for Robot Manipulation
% Behavior Transformers
% Learning Latent Plans from Play
% Waypoint-Based Imitation Learning for Robotic Manipulation
% Imitating Task and Motion Planning with Visuomotor Transformers
% Implicit Behavioral Cloning
% Deep imitation learning for bimanual robotic manipulation
\textcolor{red}{ToContinue} 
\begin{figure}[bt]
    \centering
    \includegraphics[width=0.9\textwidth]{figures/images/deep_imitation_bc/deep_imitation_bc.jpg}
    \caption{Architecture proposed in~\cite{zhang2018deep_vr_teleoperation}.}
    \label{fig:deep_bc}
\end{figure}

% \usepackage{graphicx}
% \usepackage{hhline}


\begin{table}
    \centering
    \caption{Statistics of Training set, and Test Success rate~\cite{zhang2018deep_vr_teleoperation}.}
    \label{table:deep_vr_teleoperation_results}
    \resizebox{\linewidth}{!}{%
    \begin{tabular}{|c|c|c|c|c|c|c|c|c|c|c|} 
    \hline
    \textbf{Task} & \textbf{Reaching} & \textbf{Grasping} & \textbf{Pushing} & \textbf{Plane} & \textbf{Cube} & \textbf{Nail} & \begin{tabular}[c]{@{}c@{}}\textbf{Grasp-}\\\textbf{and-}\\\textbf{Place}\end{tabular} & \begin{tabular}[c]{@{}c@{}}\textbf{Grasp-}\\\textbf{Drop-}\\\textbf{Push}\end{tabular} & \begin{tabular}[c]{@{}c@{}}\textbf{Grasp-}\\\textbf{Place-x2}\end{tabular} & \textbf{Cloth} \\ 
    \hhline{|===========|}
    \#demo & 200 & 180 & 175 & 319 & 206 & 215 & 109 & 100 & 60 & 100 \\ 
    \hline
    \begin{tabular}[c]{@{}c@{}}demo duration \\(min)\end{tabular} & 13.7 & 11.1 & 16.9 & 25.0 & 12.7 & 13.6 & 12.3 & 14.5 & 11.6 & 10.1 \\ 
    \hline
    \begin{tabular}[c]{@{}c@{}}Test success rate\\(\%)\end{tabular} & 91.6 & 97.2 & 98.9 & 87.5 & 85.7 & 87.5 & 96.0 & 83.3 & 80.0 & 97.4 \\
    \hline
    \end{tabular}
    }
    \end{table}
\paragraph*{Interactive Imitation Learning}\mbox{}\\
The Interactive Imitation Learning approach encompasses all methods specifically designed to address or mitigate the compounding error phenomenon, which was first described in~\cite{pomerleau1988alvinn}.

As previously mentioned, this issue arises because, although Behavioral Cloning primarily follows a supervised learning procedure, it does not satisfy the i.i.d. assumption. The agent's actions influence subsequent observations, creating dependencies between samples in the training set. Consequently, when the agent interacts with the environment, even small errors can lead to new observations outside the training distribution, potentially resulting in an unrecoverable situation that the robot cannot resolve autonomously.

The significance of this problem was first formalized in \cite{ross2010efficient_reductions}. The authors observed that if a system makes an error with probability $\epsilon$ in a task with a time horizon of $T$, then, due to the compounding of errors, a supervised learner incurs a quadratic total cost of $O(\epsilon \ T^{2})$, instead of the expected linear cost of $O(\epsilon \ T)$. The quadratic term arises because, at any given time step $t$, the agent's state is influenced by errors made in the previous $t-1$ steps. This cumulative effect breaks the independence assumption typically held in the i.i.d. setting.

To attenuate this problem, \textbf{interactive supervised learning algorithms} have been proposed, such as the well-known \textit{DAgger} \cite{ross2011dagger}. Algorithm \ref{alg:dagger} describes the DAgger procedure. It is an aggregation strategy, based on the idea to train the policy $\pi^{L}$ under the state-distribution induced by the policy itself, but with the correct action performed by the expert. The main problem with DAgger is that it requires the expert to interact with the system during the training, introducing both \textbf{safety} and \textbf{data-efficiency} problems, especially when the system does not provide the human expert with sufficient control authority during the sampling process \cite{laskey2017comparing_hc_rc}. 
\begin{algorithm}
\caption{DAgger Algorithm \cite{ross2011dagger}}\label{alg:dagger}
\begin{algorithmic}
\REQUIRE Initial Dataset $\mathcal{D} \leftarrow \emptyset$, Initial policy $\pi^{L}_{1}$
\ENSURE The best policy $\pi^{L}_{i}$
\FOR {$i=1, \dots N$}
    \STATE Sample $T-step$ trajectories using $\pi^{L}_{i}$
    \STATE Let $\mathcal{D}_{i} = {(s_{t}, \pi^{E}(s_{t}))}$, state $s_{t}$ visited by policy $\pi^{L}_{i}$, and actions given by the expert
    \STATE Aggregate Dataset, $\mathcal{D} \leftarrow \mathcal{D} \bigcup \mathcal{D}_{i}$
    \STATE Train policy $\pi^{L}_{i}$ on $\mathcal{D}$
    \STATE Let $\pi^{L}_{i+1} = \beta_{i}\pi^{E} + (1- \beta_{i})\pi^{L}_{i}$
\ENDFOR
\end{algorithmic}
\end{algorithm}

Human-Guided DAgger (HG-DAgger) \cite{kelly2019hg_dagger} is an enhancement of the traditional DAgger strategy, where a human expert oversees the rollout of the current policy. If the agent moves into an unsafe region of the state space, the expert steps in to guide the system back to safety. Specifically, HG-DAgger was proposed in the context of autonomous vehicle driving, however, in \cite{jang2022bc_z}, it was shown that HG-DAgger is particularly effective in robotic manipulation tasks. The study found that, given the same total number of episodes, a policy trained exclusively on expert demonstrations has a significantly lower success rate than one trained on a dataset that includes both expert demonstrations and expert corrections.

In the context of Interactive Learning for Robot Manipulation, other works of interest include \cite{mandlekar2020human_in_the_loop,chisari2022correct}. 

In \cite{chisari2022correct}, a human expert provides both \textbf{corrective} and \textbf{evaluative} feedback. The former consists in the human that takes control of the robot to adjust the trajectory, the latter consists in a scalar weight $q$, set to 1 if the trajectory is satisfactory, 0 if the trajectory is not satisfactory, $\alpha$ if the trajectory is adjusted by the expert, where $\alpha$ is the ratio between non-corrected and corrected samples. Then a Neural Network was trained by minimizing a weighted version of the maximum-likelihood, $\mathcal{L}(a_{t},s_{t}) = - q \ log(\pi^{L}_{\theta}(a_{t}|s_{t}))$. Real-world experiments show that with a training time of \textbf{41 minutes}, including environmental reset, it was possible to have an agent capable of performing tasks such as picking up a cube or pulling a plug.
\paragraph*{Multi-Task Imitation Learning}\mbox{}\\
The methods discussed in the previous paragraphs describe architectures and approaches specifically designed to solve a single task, with limited generalization to the object's category (e.g., picking blocks rather than balls) and initial state (e.g., the object's starting position). For instance, a system trained for pick-and-place operations cannot be repurposed for tasks like assembly operations. The methods described in this paragraph address these limitations, solving the problem of \textit{Multi-Task Imitation Learning} (MTIL)

Before starting to present and describe the different methods and approaches, it is necessary to describe the problem. Starting from a reformulation of the dataset used to train the system and the learned policy.
Indeed, in Section~\ref{sec:problem_formulation}, the expert dataset $\mathcal{D}^{E}$ has been introduced. Based on the problem to solve this dataset can composed in different way. Specifically, in the context of MTIL, the dataset $\mathcal{D}^{E}$ can be seen as a composition of $n$ datasets, $\mathcal{D}^{E}=\left \{\mathcal{D}_{1}, \dots, \mathcal{D}_{n}\right \}$, where $\mathcal{D}_{i} = \left \{ (\tau_{m_{i}}^{j}, \ c_{m_{i}}), j=1,\dots,N, \ m_{i} \in \mathcal{M}_{i}\right \}$ is the \textit{single-task dataset}, composed of:
\begin{itemize}
    \item $N$ expert demonstration for each $m^{th}$ variation of the $i^{th}$ task, where $M_{i}$ is the number of variation for the $i^{th}$ task.
    \item Agent trajectories $\tau_{m_{i}}^{j} = (s_{0}, a_{0}, s_{1}, a_{1}, \dots, a_{T-1}, s_{T})$. where $s_{t}$ is the state at time $t$ and $a_{t}$ is the corresponding action (Section \ref{sec:problem_formulation}).
    \item Task-conditioning signal $c_{m_{i}}$ for the $m^{th}$ variation of $i^{th}$ task, which describes the desired task in terms of video demonstrations \cite{james2018task_embedded,bhutani2022attentive_one_shot,dasari2021transformers_one_shot,mandi2022towards_more_generalizable_one_shot}, natural language description \cite{stepputtis2020language,jang2022bc_z,mees2022calvin,doasIcan2022,mees2022hulc,brohan2022rt,shridhar2023perceiver} or multi-modal prompt, that exploits both visual information (e.g., an image of the target object) and text information that contains the information related to the action to be performed \cite{jiang2023vima}.
\end{itemize}
The goal of Multi-Task Imitation Learning is to learn a \textit{conditioned policy} $\pi^{L}_{\theta}(a_{t}|s_{t}, c_{m_{i}})$, that is able to map the current state and command into the corresponding action.
Depending on how the policy is defined, various loss functions come into play. In the case of deterministic policies, the learning process focuses on minimizing the Mean-Squared Error (refer to Formula \ref{eq:mse}). However, for probabilistic policies, the learning process centers around minimizing the Negative Log-likelihood (refer to Formula \ref{eq:nll}). This approach aims to enhance the probability of correctly executing the action.
% \begin{equation}
%     \label{eq:mse}
%     \mathcal{L}(D^{E}, \pi^{L}_{\theta}) = \frac{1}{n} \sum_{\mathcal{D}^{i} \sim \mathcal{D}^{E}} \frac{1}{N} \sum_{(\tau_{m_{i}}^{j}, c_{m_{i}}) \sim \mathcal{D}^{i}} \frac{1}{T}\sum_{t=0}^{T} (a_{t} - \pi^{L}_{\theta}(s_{t}, c_{m_{i}}))
% \end{equation}
% \begin{equation}
%     \label{eq:nll}
%     \mathcal{L}(D^{E}, \pi^{L}_{\theta}) = \frac{1}{n} \sum_{\mathcal{D}^{i} \sim \mathcal{D}^{E}} \frac{1}{N} \sum_{(\tau_{m_{i}}^{j}, c_{m_{i}}) \sim \mathcal{D}^{i}} \frac{1}{T}\sum_{t=0}^{T} (a_{t} - \pi^{L}_{\theta}(s_{t}, c_{m_{i}}))
% \end{equation}

\begin{equation}
    \label{eq:mse}
    \mathcal{L}(\tau_{m_{i}}^{j}, c_{m_{i}},\pi^{L}_{\theta}) = \frac{1}{T}\sum_{t=0}^{T} (a_{t} - \pi^{L}_{\theta}(s_{t}, c_{m_{i}}))^{2}
\end{equation}
\begin{equation}
    \label{eq:nll}
    \mathcal{L}(\tau_{m_{i}}^{j}, c_{m_{i}},\pi^{L}_{\theta}) = - \frac{1}{T}\sum_{t=0}^{T} log(\pi^{L}_{\theta}(a_{t}|s_{t}, c_{m_{i}}))
\end{equation}
The following sections will describe the various approaches proposed to address the problem. Figure \ref{fig:mtil_taxonomy} illustrates the taxonomy used for the Multi-Task Imitation Learning methods. Specifically, the methods are categorized based on the type of generalization required by the algorithm (Few-Shot vs. Zero-Shot). For Few-Shot generalization, the Meta-Learning paradigm will be discussed, as it is most relevant to the problem at hand. For Zero-Shot generalization, the methods are further divided based on the type of conditioning signal used, whether it is provided through natural language descriptions or visual information.
\begin{figure}[t]
    \centering
    \includegraphics[width=0.8\textwidth]{figures/images/MTIL_taxonomy.png}
    \caption{Multi-Task Imitation Learning Taxonomy}
    \label{fig:mtil_taxonomy}
    
\end{figure}


\textbf{Few-Shot MTIL} refers to approaches designed to train a model on a variety of tasks so that it can effectively solve a new task using only a small number of samples and consequently requires only a few back-propagation steps \cite{finn2017maml}. In this context, one of the most significant learning paradigms, especially relevant to robotic manipulation, is \textit{Meta-Learning}. The goal of Meta-Learning is to train a model that can ``learn to learn," meaning it develops a set of general weights $\theta$ that, while not directly usable for solving any specific task within a distribution of tasks $\mathcal{T}$, can be quickly adapted through a few backpropagation steps to solve a given task within that distribution, $\mathcal{T}_{i} \in \mathcal{T}$. One of the most popular Meta-Learning algorithms is the \textit{Model-Agnostic Meta-Learning} (MAML) algorithm \cite{finn2017maml}, described in Algorithm \ref{alg:maml}. The MAML algorithm follows an iterative learning procedure consisting of two steps:
\begin{itemize}
    \item \textbf{Meta-Learning}: During this phase, task-specific weights $\theta_{i}$ are computed for each sampled task $\mathcal{T}_{i}$. Specifically, the \textit{meta-parameters} $\theta$ are updated according to the gradient obtained from evaluating the loss function on the $i^{th}$ task $\mathcal{T}_{i}$, where the function $f$ is parameterized by the meta-parameters $\theta$.

    \item \textbf{Meta-Adaptation}: In this phase, the meta-parameters are further refined. The loss function $f$, now parameterized by the task-specific parameters for the $i^{th}$ task, is used to adjust the meta-parameters based on the gradients derived from the sum of the loss functions evaluated on the task-specific weights. This process provides feedback to the meta-parameters $\theta$ from each task, leading to a generalized point that can be easily adapted to new tasks (Figure \ref{fig:maml_weights}).
\end{itemize}
\begin{algorithm}[t]
\caption{Model-Agnostic Meta-Learning (MAML) \cite{finn2017maml}}
\label{alg:maml}
\begin{algorithmic}
\REQUIRE Distribution over tasks $p(\mathcal{T})$
\STATE Randomly initialize $\theta$
\WHILE {$i=1, \dots N$}
    \STATE Sample batch of tasks $ \mathcal{T}_{i} \sim p(\mathcal{T})$
    \FOR {\textbf{all} $\mathcal{T}_{i}$}
        \STATE Evaluate $\nabla_\theta \mathcal{L}_{\mathcal{T}_{i}}(f_{\theta})$ w.r.t. $K$ examples
        \STATE Compute adapted parameters with gradient descent: $\theta'_{i} = \theta - \alpha \nabla_\theta\mathcal{L}_{\mathcal{T}_{i}}(f_{\theta})$
    \ENDFOR
    \STATE Update $\theta \leftarrow \theta - \beta \nabla_\theta \sum_{\mathcal{T}_{i} \sim p(\mathcal{T})} \mathcal{L}_{\mathcal{T}_{i}}(f_{\theta'_{i}})$
\ENDWHILE
\end{algorithmic}
\end{algorithm}
\begin{figure}[tb]
    \centering
    \includegraphics[width=0.6\textwidth]{figures/images/maml_weights.jpg}
    \caption{Diagram of MAML algorithm, which optimizes for a representation $\theta$ that can quickly adapt to new tasks.}
    \label{fig:maml_weights}
\end{figure}

The MAML algorithm is the base for different methods which apply Few-Shot Imitation Learning in the context of Behavioral Cloning \cite{finn2017one_shot_visual_il,yu2018daml,yu2018one_shot_hil}.

In \cite{finn2017one_shot_visual_il}, MAML algorithm was used to prove the effectiveness of Meta-Learning in the context of real robot manipulation, with visual observations, as opposite to \cite{duan2017one_shot_il}. A Convolutional Neural Network was trained by following the Algorithm \ref{alg:maml}, using as loss-function the Mean Squared Error, computed between the predicted action and the ground truth one. For real-robot experiments a dataset of \textbf{1300} placing demonstrations (i.e., place an holded object in a target container), containing near to \textbf{100} different objects, was collected through teleportation. The trained system was tested by performing the adaptation step on one video demonstration, over 29 new objects, moreover, between the video demonstration and the actual execution, the objects configuration was changed. In this setting the system reached the $\mathbf{90\%}$ of success rate, outperforming baseline methods based on LSTM \cite{duan2017one_shot_il}, and contextual network (i.e., a Convolutional Neural Network that takes in input the current observation and the image representing the target state).

In \cite{yu2018daml}, the \textit{Domain Adaptive Meta-Learning} algorithm (DAML) was proposed with the goal of learning to infer a policy from a single human demonstration. To achieve it, a two-step algorithm was proposed. In the first-step, called \textbf{Meta-Learning step}, given in input, for each task $\mathcal{T}$, a set of human demo $\mathcal{D}^{h}_{\mathcal{T}}$ and a set or robot demo $\mathcal{D}^{r}_{\mathcal{T}}$ (Figure \ref{fig:daml_tasks}), the \textit{initial policy parameters} $\theta$ and the \textit{adaptive loss} parameters $\psi$ are learned, solving the problem in Formula \ref{eq:daml}.
\begin{equation}
 \label{eq:daml}
 \underset{\theta,\psi}{\min} \sum_{\mathcal{T} \sim p(\mathcal{T})} \sum_{\mathbf{d}^{h} \sim D^{h}_{\mathcal{T}}} \sum_{\mathbf{d}{^r} \sim D^{r}_{\mathcal{T}}} \mathcal{L}_{BC}(\theta - \alpha \nabla_\theta\mathcal{L}_{\psi}(\theta,\mathbf{d}^{h}), \mathbf{d}^{r})
\end{equation}

\newline The outer loss is the classic supervised Behavioral Cloning loss, defined as $\mathcal{L}_{BC}(\phi, \mathbf{d^{r}}) = \sum_{t} \log(\pi_{\phi}(a_{t} \mid s_{t}, o_{t}))$. The inner loss, $\mathcal{L}_{\psi}$, is a learned \textbf{adaptive loss}. Specifically, $\mathcal{L}_{\psi}$ is used during Meta-Adaptation, where the policy parameters are updated by evaluating the gradients derived from $\mathcal{L}_{\psi}$. This process involves using a video of a human demonstrating a new task $\mathcal{T}$ as input, leading to the policy update defined by $\phi_{\mathcal{T}} = \theta - \alpha \nabla_{\theta} \mathcal{L}_{\psi}(\theta, \mathbf{d}^{h})$. 
\newline This adaptive loss is the key component proposed in DAML. To use it effectively, it is necessary to learn the parameters $\psi$, observing how there is no direct correspondence between the human video demonstration and the robot's ground truth actions. To address this challenge, the authors of DAML observed that while the policy learns to produce appropriate actions through the $\mathcal{L}_{BC}$ loss, the adaptive loss should instead adjust the perceptual aspect of the policy, focusing on human motion and the manipulated object. Based on this insight, the authors implemented the function $\mathcal{L}_{\psi}$ using a 1D Temporal Convolutional Network (Figure \ref{fig:daml_temporal_adaptation_loss}). The convolutional layers are applied to a stack of embeddings generated by the policy $\pi$ across different frames of the video demonstrations. The parameters of this module are learned during the meta-training phase, following the weight update process described in Formula \ref{eq:daml_temporal_adaptation_loss}. The objective of $\mathcal{L}_{\psi}$ is to generate task-specific policy parameters $\phi_{\mathcal{T}}$ that guide the policy to produce effective actions.

\begin{equation}
 \label{eq:daml_temporal_adaptation_loss}
 \begin{matrix}
    (\theta, \psi) \leftarrow(\theta, \psi)-\beta \nabla_{\theta, \psi} \mathcal{L}_{\mathrm{BC}}\left(\phi_{\mathcal{T}}, \mathbf{d}^r\right) \\ \\
   \phi_{\mathcal{T}}=\theta-\alpha \nabla_\theta \mathcal{L}_\psi\left(\theta, \mathbf{d}^h\right)
   \end{matrix}
\end{equation}

\newline Experimental evaluation on tasks such as placing, pushing, and pick-and-place, has shown that: \begin{itemize}
    \item The system was able to generalize across both new objects and objects configuration starting from only a single human demonstration;
    \item A performance degradation was observed in large domain-shift experiments, such as novel backgrounds and different camera view-points.
\end{itemize}
\begin{figure}[t]
    \centering
    \includegraphics[width=\textwidth]{figures/images/daml/tasks.jpg}
    \caption{Tasks performed in \cite{yu2018daml}. (Top row) Human demonstration, (Bottom row) robot demonstration. (Left) Placing task, (Middle) pushing task, (Right) pick-and-place task.}
    \label{fig:daml_tasks}
\end{figure}

\begin{figure}[t]
    \centering
    \includegraphics[width=0.8\textwidth]{figures/images/daml/daml_temporal_adaptation_loss.jpg}
    \caption{The Temporal Adaptation Loss architecture applies 1D temporal convolutional layers to the stacked embeddings generated by the policy $\pi$ from the frames of the human video demonstration.}
    \label{fig:daml_temporal_adaptation_loss}
\end{figure}

Meta-Learning algorithms have demonstrated intriguing properties, notably their capacity for few-shot generalization to novel objects and object configurations. However, it has been observed that during the adaptation step, these methods tend to lose their effectiveness in performing other tasks. This limitation has underscored the need for the development of Multi-Task Imitation Learning methods, which aim to address these shortcomings and enable more versatile task execution in complex scenarios. These kind of methods will be discussed in the following paragraphs.

\textbf{Zero-Shot MTIL} refers to approaches that aim to train a model capable of solving different tasks without any further adaptation or backpropagation steps. This approach addresses a key issue in Meta-Learning methods, which is the problem of forgetting how to solve previous tasks after adapting to a new one. The goal is to develop a single policy that can handle multiple tasks in a zero-shot manner.

In this context, a crucial design choice is how to convey the desired task to the policy. The literature identifies two main methods to address this problem:

\begin{enumerate}
    \item \textit{Language Conditioned}: These methods leverage natural language descriptions of tasks to inform the model about the task to be executed.
    \item \textit{Visual Conditioned}: These methods use visual information (e.g., goal-state images, video demonstrations) to provide the model with the task instructions.
\end{enumerate}
\textcolor{red}{ToDo}

\textit{Language Conditioned}

\textit{Visual Conditioned}
\paragraph*{Object-Oriented Imitation Learning}\mbox{}\\
\textit{Object-Oriented Imitation Learning} focuses on learning behaviors in relation to specific objects and their interactions, providing a more structured and contextual approach to imitation learning.

\paragraph{Discussion}
\subsubsection{Inverse Reinforcement Learning}
\label{sec:irl}
This section will be dedicated to the introduction of the \textit{Inverse Reinforcement Learning} (IRL) approach. 

\subsubsection{Generative Adversarial Imitation Learning}
\label{sec:gail}
The \textit{Generative Adversarial Imitation Learning} (GAIL) is a Learning from Demonstration approach proposed by the authors of \cite{ho2016gail}. The rationale behind GAIL was to improve the Inverse Reinforcement Learning (IRL) setting, which is expensive to run due to the double-nested optimization procedure. To achieve this objective, the authors in \cite{ho2016gail} started from the Max-Ent formulation in Formula \ref{formula:regularized_max_ent}, and obtained a characterization of the learned policy. This characterization combines the learning of the reward function and the learning of the policy through a reinforcement learning algorithm. In Formula \ref{formula:regularized_max_ent}, $\psi(c)$ is a cost-regularizer, $\psi^{*}(c)$ is its conjugate, and $\rho_{\pi}$ is the occupancy measure, i.e., the distribution of state-action pairs that the agent encounters when navigating the environment with policy $\pi$. 

The next step was to choose an appropriate regularization function. In particular, by choosing the regularizer in Formula \ref{formula:ga_regularization}, the conjugate in Formula \ref{formula:ga_regularizer_conjugate} can be obtained. This is the classic Adversarial Learning Loss, where the current policy $\pi^{L}$ acts as the GAN generator, and $D$ is the GAN discriminator, which must distinguish between state-action pairs generated either by the expert policy or by the current policy.

\begin{equation}
    \label{formula:regularized_max_ent}
    \begin{aligned}
        IRL_{\psi}(\pi^{E}) = \underset{c \in \mathbb{R}^{S \times A}}{\operatorname{arg\,max}} &  - \psi(c) +  \\
        & \left(\underset{\pi^{L} \in \Pi}{\min} -\mathcal{H}(\pi^{L}) + \mathbb{E}_{\pi^{L}} \left [ c(s,a) \right ]\right) - \\
        &  \mathbb{E}_{\pi^{E}} \left [ c(s,a) \right ]
    \end{aligned}
\end{equation}

\begin{equation}
    \label{formula:policy_characterization}
    \begin{aligned}
    RL \circ IRL_{\psi}(\pi^{E}) = \underset{\pi^{L} \in \Pi}{arg \ min}-\mathcal{H}(\pi^{L}) + \psi^{*}(\rho_{\pi^{L}} - \rho_{\pi^{E}}) 
    \end{aligned}
\end{equation}

\begin{equation}
    \label{formula:ga_regularization}
    \begin{split}
    \psi_{GA}(c) = \left\{
    \begin{matrix}
        \mathbb{E}_{\pi^{E}}\left[ g(c(s,a)) \right] & \text{if } c < 0\\ 
        +\infty & \text{otherwise}
    \end{matrix}
    \right.,
    \\
    g(x) = \left\{
    \begin{matrix}
        -x - \log(1- e^{x}) & \text{if } c < 0\\ 
        +\infty & \text{otherwise}
    \end{matrix}
    \right.
    \end{split}
\end{equation}

\begin{equation}
    \begin{aligned}
    \label{formula:ga_regularizer_conjugate}
    \psi^{*}_{GA}(\rho_{\pi^{L}} - \rho_{\pi^{E}}) = \underset{D\in(0,1)^{S \times A}}{max} \mathbb{E}_{\pi^{L}}\left [ \log(D(s,a))\right ] + & \\ \mathbb{E}_{\pi^{E}}\left [ \log(1 - D(s,a))\right ]
\end{aligned}
\end{equation}
Based on these considerations, Algorithm \ref{alg:gail} has been proposed. Specifically, the GAIL algorithm is an iterative procedure composed of two main steps. The first step involves updating the discriminator $D$, which must distinguish between trajectories produced by the learned policy $\tau^{L}_{i}$ and trajectories produced by the expert $\tau^{E}$. The second step involves updating the learner policy $\pi^{L}$ according to some reinforcement learning algorithm (e.g., TRPO was used in \cite{ho2016gail}). The policy is updated in such a way that the trajectories it generates become indistinguishable from those of the expert for the discriminator, i.e., the learner produces state transitions that are similar to those of the expert.
\begin{algorithm}[tb]
\caption{Generative Adversarial Imitation Learning Algorithm}\label{alg:gail}
\begin{algorithmic}
\Require Expert Trajectories $\tau^{E} \sim \pi^{E}$, initial policy $\pi^{L}_{\theta}$, discriminator $D_{\omega}$
\For {$i=1, \dots, N$} 
    \State Sample trajectories, $\tau^{L}_{i} \sim \pi^{L}_{\theta}$
    \State Update Discriminator, $\mathbb{\hat{E}}_{\tau^{L}_{i}}\left [\nabla_{\omega} \log(D_{\omega}(s,a))\right ] +\mathbb{\hat{E}}_{\tau^{E}}\left [\nabla_{\omega} \log(1 - D_{\omega}(s,a))\right ]$
    \State Update Policy $\pi_{\theta}$, with TRPO \cite{schulman2015trpo}, and cost-function $C(s,a)=\log(D_{\omega}(s,a))$ 
\EndFor
\end{algorithmic}
\end{algorithm}

In the seminal work \cite{ho2016gail}, GAIL has proven to be more effective than classic IRL (Inverse Reinforcement Learning) algorithms \cite{ziebart2008maximum_entropy,ho2016model}. The authors evaluated the GAIL algorithm on nine classic simulated control tasks from the OpenAI Gym simulator \cite{brockman2016openai}: Cartpole, Acrobot, MountainCar, HalfCheetah, Hopper, Walker, Ant, Humanoid, and Reacher.

\begin{figure}[tb]
    
    \includegraphics[width=\textwidth]{figures/images/gail_performance.jpg}
    \caption{The performance comparison proposed in \cite{ho2016gail} is presented here. The y-axis shows the scaled reward, where the expert's reward is set to 1 and the random baseline is set to 0. The IRL baselines FEM and GTAL refer to the IRL algorithm described in \cite{ho2016model}, but with different cost functions.}
    \label{fig:gail_performance}
    
\end{figure}




\begin{table}[t]
    \centering
    \tiny
    \selectfont
    \caption{Observation and Action space for the tasks used in~\cite{ho2016gail}}
    \label{table:gail_tasks}
    \resizebox{\linewidth}{!}{%
    \begin{tabular}{|c|c|c|} 
    \hline
    \textbf{Task} & \textbf{Observation space} & \textbf{Action space} \\ 
    \hhline{|===|}
    Cartpole & 4 (continuous) & 2 (discrete) \\ 
    \hline
    Acrobot & 4 (continuous) & 3 (discrete) \\ 
    \hline
    Mountain Car & 2 (continuous) & 3 (discrete) \\ 
    \hline
    Reacher & 11 (continuous) & 2 (continuous) \\ 
    \hline
    HalfCheetah & 17 (continuous) & 6 (continuous) \\ 
    \hline
    Hopper & 11 (continuous) & 3 (continuous) \\ 
    \hline
    Walker & 17 (continuous) & 6 (continuous) \\ 
    \hline
    Ant & 111 (continuous) & 8 (continuous) \\ 
    \hline
    Humanoid & 376 (continuous) & 17 (continuous) \\
    \hline
    \end{tabular}
    }
    \end{table} 

The observation and control spaces for these tasks are detailed in Table \ref{table:gail_tasks}, and the performance results are shown in Figure \ref{fig:gail_performance}. From these results, it is evident that the GAIL algorithm overcomes classic IRL algorithms in terms of both pure reward and sample efficiency. Consequently, subsequent research has focused on improving the algorithm's efficiency in terms of environment interaction. This has been achieved by replacing the model-free, on-policy TRPO algorithm with an off-policy RL algorithm, as seen in \cite{kostrikov2018discriminator}, or by modifying the reward function input to the RL algorithm \cite{fu2018airl,ghasemipour2020divergence_minimization_perspective}.

However, as noted in Table \ref{table:gail_tasks}, the tested tasks are characterized by low-dimensional state spaces. More recent research \cite{liu2018imitation_from_observation,reddy2019sqil,zolna2021task_relevant_ail,rafailov2021visual_ail} has focused on testing the GAIL algorithms in high-dimensional state spaces, where the input is an image. Specifically, with respect to the Adversarial Imitation Learning setting, works of interest are \cite{zolna2021task_relevant_ail,rafailov2021visual_ail}. 

In \cite{zolna2021task_relevant_ail}, the authors focused on solving the \textbf{casual-confusion} problem. This problem occurs when the discriminator, during the learning process, focuses on task-irrelevant features between expert and policy generated transitions, for example a difference in the expert and agent embodiment like the gripper, this causes the rewards to become uninformative. To reduce the casual-confusion problem, in \cite{zolna2021task_relevant_ail} two elements have been proposed: 
\begin{enumerate}[label=\arabic*.]
    \item A \textit{regularization term}, with the aim to make the discriminator \textbf{unable} to distinguish between constraining sets $I_{E}$ and $I_{A}$. These sets are composed of expert and agent observations, such that a sample can belong either to $I_{E}$ or $I_{A}$, based on spurious features (e.g., a different gripper color);
    \item An \textit{early-stopping policy} called Actor Early-Stopping (AES), that restarts the episode if the discriminator score at the current step exceeds the median score of the episode so far for $T_{patience}$ consecutive steps.
\end{enumerate} 

To prevent the discriminator from focusing on task-irrelevant features, the authors proposed a regularization term based on the constraining-set accuracy defined in Formula \ref{eq:trail}. The idea is that if the discriminator achieves an accuracy greater than $\frac{1}{2}$ on the constraining set, the maximized adversarial cost function should be inverted, as shown in Formula \ref{eq:trail_discriminator}.

\begin{equation}
\label{eq:trail}
\textit{accuracy}(I_{E}, I_{A}) = \frac{1}{2} \ \mathbb{E}_{s \in I_{E}} \left[ \mathbf{1}_{D_{\omega} \geq  \frac{1}{2}}\right] + \frac{1}{2} \ \mathbb{E}_{s \in I_{A}} \left[ \mathbf{1}_{D_{\omega} <  \frac{1}{2}}\right] 
\end{equation}
\begin{equation}
    \label{eq:trail_discriminator}
    \begin{aligned}
        \mathcal{L}_\psi\left(s_E, s_A, \hat{s}_E, \hat{s}_A\right) 
        &= G_\psi\left(s_E, s_A\right) \\
        &\quad - \mathbf{1}_{\text {accuracy }\left(\hat{s}_E, \hat{s}_A\right) \geq \frac{1}{2}} G_\psi\left(\hat{s}_E, \hat{s}_A\right), \\ \\
        \text{where} \ \ G_\psi\left(s_E, s_A\right) 
        &= \sum_{i=1}^N \log D_\psi\left(s_E^{(i)}\right) \\
        &\quad + \log \left[1 - D_\psi\left(s_A^{(i)}\right)\right]
    \end{aligned}
\end{equation}
\begin{figure}[t]
    \centering
    \begin{subfigure}[b]{0.6\textwidth}
        \centering
        \includegraphics[width=\textwidth]{figures/images/trail/block_lifting.png}
        \caption{Lifting task}
    \end{subfigure}
    \vfill
    \begin{subfigure}[b]{0.6\textwidth}
        \includegraphics[width=\textwidth]{figures/images/trail/block_lifting_with_distractors}
        \caption{Lifting with distractors task}
    \end{subfigure}
    \vfill
    \begin{subfigure}[b]{0.6\textwidth}
        \includegraphics[width=\textwidth]{figures/images/trail/block_stacking}
        \caption{Block Stacking task}
    \end{subfigure}
    \vfill
    \begin{subfigure}[b]{0.6\textwidth}
        \includegraphics[width=\textwidth]{figures/images/trail/block_insertion_with_distractors}
        \caption{Block Insertion with distractors task}
    \end{subfigure}
    \caption{Tasks solved in~\cite{zolna2021task_relevant_ail}}
    \label{fig:trail_tasks}
\end{figure}

The proposed system was tested on 4 tasks (Figure \ref{fig:trail_tasks}), with the agent trained on each single task, according to the \textit{Distributed Distributional Deterministic Policy Gradients} (D4PG) \cite{barth2018d4pg} RL algorithm, with reward-function $R(s_{t}) = - \log(1-D_{\omega}(s_{t}))$. Experimental results have shown how the proposed system overcomes the GAIL \cite{ho2016gail} baseline, both in setting with spurious features and without spurious features (Figure \ref{fig:trail_results}).
\begin{figure}[t]
    \centering
    \includegraphics[width=1.0\textwidth]{Figures/images/trail/trail_results.jpg}
    \caption{Experimental results on tasks without and with spurious features~\cite{zolna2021task_relevant_ail}.}    
   \label{fig:trail_results}
\end{figure}




The authors of \cite{rafailov2021visual_ail} proposed a more data-efficient Adversarial Imitation Learning method. They leveraged a model-based approach within a high-dimensional state space. Instead of generating a dynamic model in the image space, i.e., training a generative model to produce the next image based on the current image and the performed action.Their method encodes observations defined in the image space into a corresponding latent space characterized by vectors of smaller dimensions. Then, they learn a dynamic model in that space, training a generative model to produce the next embedding based on the current encoded observation and the performed action. The proposed learning procedure is based on three main steps:
\begin{enumerate}[label=\arabic*.]
    \item Learn the \textit{Latent Dynamic Model}, $(\hat{\mathcal{U}}_{\beta},\hat{\mathcal{T}}_{\beta}, q_{beta})$, by maximizing the Evidence Lower Bound (Formula \ref{formula:elbo}), where $\hat{\mathcal{U}}_{\beta}$ is the decoder, $q_{beta}$ is the encoder, and $\hat{\mathcal{T}}_{\beta}$ is the transition model;
    \item Train a \textit{discriminator}, $D_{\theta}$, by minimizing the Adversarial Loss function (Formula \ref{formula:discriminator});
    \item Train a \textit{policy} $\pi^{L}_{\theta}$, by maximizing the Value function (Formula \ref{formula:value_function}).
\end{enumerate}
\begin{equation}
    \label{formula:elbo}
    \underset{\beta}{max} \ \mathbb{E}_{q_{\beta}}\left[\sum_{t} \log(\mathcal{\hat{U}}_{\beta}(s_{t}|z_{t})) + \mathbb{D}_{KL}(q_{\beta}(z_{t}|s_{t},z_{t-1},a_{t-1})|| \mathcal{\hat{T}}_{\beta}(z_{t}|z_{t-1},a_{t-1})) \right]
\end{equation}

\begin{equation}
    \label{formula:discriminator}
    \underset{\theta}{min} \ \mathbb{E}_{(z,a) \sim \rho^{E}(z,a)} \left[-log(D_{\theta}(z,a))\right] + \mathbb{E}_{(z,a) \sim \rho^{\pi^{L}}_{\hat{\mathcal{T}}}} \left[-\log(1 - D_{\theta}(z,a))\right]
\end{equation}

\begin{equation}
    \label{formula:value_function}
    \underset{\pi^{L}_{\theta}}{max} \ V^{K}_{\theta,\beta}(z_{t}) = \underset{\pi^{L}_{\theta}}{max} \ \mathbb{E}_{\pi^{L}_{\theta}, \mathcal{\hat{T}}_{\beta}}\left[ \sum_{\tau = t}^{t+K-1} \gamma^{\tau-t} \log(D_{\theta}(z_{\tau}^{\pi^{L}_{\theta}}, a_{\tau}^{\pi^{L}_{\theta}})) + \gamma^{K}V_{\beta}(z_{t+K}^{\pi^{L}_{\theta}})\right]
\end{equation}
    
    
With this learning setting the proposed system outperforms previous works such as \cite{reddy2019sqil,kostrikov2018discriminator} both in terms of \textbf{data-efficiency} and \textbf{overall performance}, on a set of continous control tasks.
% \begin{algorithm}[t]
    \caption{Variational Model-Based Adversarial Imitation Learning \cite{rafailov2021visual_ail}}
    \label{alg:vmail}
    \begin{algorithmic}
    \REQUIRE Expert Demonstrations $\mathcal{D}^{E}$, environment buffer $\mathcal{D}^{\pi^{L}}$
    \REQUIRE Policy $\pi^{L}_{\theta}$, discriminator $D_{\omega}$, variational model $(q_{\beta}, \hat{\mathcal{T}}_{\beta})$
    \FOR {$i=1, \dots, N$}
        \FOR {$t=1, \dots, T$}
            \STATE Estimate latent state, $z_{t} \sim q_{\theta}(.|s_{t},z_{t-1},a_{t-1})$
            \STATE Sample action, $a_{t} \sim  \pi^{L}_{\theta}(a_{t}|z_{t})$
            \STATE Observ new state, $s_{t+1}$
        \ENDFOR
        \STATE Add transitions ${s_{1:T}, a_{1:T-1}}$ to $\mathcal{D}^{\pi^{L}}$
        \FOR {training iterations}
            \STATE Optimize Variational Model $(q_{\beta}, \hat{\mathcal{T}}_{\beta})$, Formula \ref{}
            \STATE Infer Expert Latent Space, $z_{1:T}^{E} \sim q_{\theta}(.| s_{1_T}^{E}, a_{1:T-1}^{E})$
            \STATE Generate latent rollouts, $z_{1:H}^{L} $ 
        \ENDFOR
    \ENDFOR
\end{algorithmic}
\end{algorithm}

Generally speaking, the Generative Adversarial Imitation Learning has shown very promising performance in simulated control tasks and simulated robot manipulation tasks, even in complex high-dimensional state-space. However, it is not so clear, how these methods could perform in real-world robotic manipulation tasks, in terms of data-efficiency, generalization capability, and safety during real-world interactions. 
\subsubsection{Learning from Observation}
\textcolor{red}{VEDERE CITAZIONI RECENTI MIMICPLAY,qian2024contrast}
\begin{figure}[t]
    
    \includegraphics[width=\textwidth]{figures/images/embodiment_mismatch/embo.jpg}
    \caption{Representation of \textbf{embodiment mismatch problem}. (Left) The source domain
    represented by a video of human performing a task. (Right) The target domain, represented
    by the robot that executes the observed task}
    \label{fig:embo_mismatch}
    
\end{figure}

\label{sec:lfo}
In all the previous sections, the methodologies assume access to the agent actions, working with state-action trajectories. In \textit{Learning from Observation} (LfO), this assumption is relaxed, and methods for learning from \textbf{state-only} demonstrations are proposed. This approach has gained attention in recent years~\cite{torabi2019recent_advances_lfo} because it theoretically allows a robotic system to be programmed as naturally as possible. Ideally, a robotic system should be able to reproduce a task by observing a human or another robot performing it, without access to the actions performed, unlike the methods described so far.

To address this problem, several questions need to be answered:

\begin{enumerate}
\item How can embodiment mismatches be resolved when the demonstrator has a different embodiment than the imitator?
\item How can the correspondence problem be handled when the demonstrator viewpoint differs from the imitator?
\item Once the perception subsystem issues are resolved, how is the policy $\pi^{L}$ obtained?
\end{enumerate}

The first question refers to the \textit{correspondence problem} introduced in Section~\ref{sec:sod}. This problem arises when the demonstrator embodiment differs from that of the learner, meaning that methods cannot directly use the recorded trajectories of the demonstrator.

One approach to solving this problem is to use methods that perform \textit{image-to-image} translation. This involves using generative deep architectures to transform images of a subject in one domain (e.g., a human demonstrator) into images where the context remains the same, but the subject is different (e.g., the human demonstrator is replaced by the target robot) as depicted in Figure \ref{fig:embo_mismatch}. This approach has been followed by authors in \cite{smith2019avid,xiong2021learning_by_watching,li2021meta_watching_video_demonstration}.

Specifically, the authors in \cite{smith2019avid, xiong2021learning_by_watching} used the Cycle-GAN architecture \cite{zhu2017cycle_gan} to translate images from the source domain (human images) to the target domain (robot images) in an unsupervised manner. The work in \cite{zhu2017cycle_gan} shifted the translation problem from a paired image setting, where each source domain image has a corresponding target domain image, to an unpaired image setting, where the source domain image does not have a corresponding target domain image.

To address this, Cycle-GAN introduces a novel learning procedure involving two translation models: $G: X \rightarrow Y$ and $F: Y \rightarrow X$. The first model maps inputs from the source domain to the target domain, while the second model maps inputs from the target domain back to the source domain. These two models are trained in an adversarial setting by minimizing the loss function shown in Formula \ref{eq:cycle_gan_loss}.
\begin{equation}
    \label{eq:cycle_gan_loss}
    \begin{aligned}
        \mathcal{L}(G,F,D_X,D_Y) &= \mathcal{L}_{GAN}(G,D_Y,X,Y) + \\
        &\quad \mathcal{L}_{GAN}(F,D_X,Y,X) + \lambda\mathcal{L}_{cyc}(G,F) 
        \\ \\
        \mathcal{L}_{GAN}(Z,D_{K},S,T) &= \mathbb{E}_{t \sim p_{data}(t)}\left[ \log(D_K(t)) \right] + \\
        &\quad \mathbb{E}_{s \sim p_{data}(s)}\left[ \log(1 - D_K(Z(s))) \right]    
        \\ \\
        \mathcal{L}_{cyc}(G,F) &= \mathbb{E}_{x \sim p_{data}(x)}\left[ \|F(G(x)) - x\|_{1} \right] +  \\
        &\quad \mathbb{E}_{y \sim p_{data}(y)}\left[ \|G(F(y)) - y\|_{1} \right]
        \end{aligned}
\end{equation}

Here, $\mathcal{L}_{GAN}$ is the adversarial loss component, where the discriminator $D_{K}$ is trained to distinguish between real samples $t \in T$ and translated samples $Z(s)$. The generator $Z$ is trained to generate samples that are as similar as possible to those in the target domain, starting from samples in the source domain. Meanwhile, $\mathcal{L}_{cyc}$ is a loss term that aims to maintain consistency between the generated samples and the ground truth one.

The application of this concept in the domain of interest, lead to a dataset for the source domain was composed of human demonstrations as well as a small amount of ``random" data, in which the human moves around the scene but does not specifically attempt the task, while for the target domain, it consists of robot images executing randomly sampled actions in a few different settings.

The second question also addresses a variant of the correspondence problem, where, in addition to the embodiment mismatch, the problem of \textit{different viewpoints} is also encountered (Figure \ref{fig:time_contrastive}). This issue has been tackled in \cite{sermanet2018time_contrastive, liu2018imitation_from_observation}.

In \cite{sermanet2018time_contrastive}, a Convolutional Neural Network was trained using a \textit{Triplet-Loss} \cite{schroff2015triplet_loss}. The aim was to train a network to predict an embedding independent of the viewpoint, but containing only task-relevant features. To achieve this, the network had to produce an embedding, $f(x)$, such that $|| f(x^{a}_{i}) - f(x^{p}_{i})||^{2}_{2} + \alpha < || f(x^{a}_{i}) - f(x^{n}_{i})||^{2}_{2}$ for all $(f(x^{a}_{i}), f(x^{p}_{i}), f(x^{n}_{i})) \in \mathcal{T}$, where $\mathcal{T}$ is the set of all possible triplets in the dataset. This implies that embeddings produced by samples from different viewpoints, but sharing the same time-step, $(x^{a}_{i},x^{p}_{i})$, should be similar, while embeddings produced by samples from the same viewpoint, but at different time-steps, $(x^{a}_{i},x^{n}_{i})$, should be different (Figure \ref{fig:time_contrastive}).

In \cite{liu2018imitation_from_observation}, a different approach was employed. Here, a \textit{context translation problem} was addressed using an Encoder-Decoder architecture (Figure \ref{fig:context-translation}). The proposed architecture was trained on pairs of demonstrations, $\mathcal{D}_{i}=[o^{i}_{0},o^{i}_{1},\dots,o^{i}_{T}]$ and $\mathcal{D}_{j}=[o^{j}_{0},o^{j}_{1},\dots,o^{j}_{T}]$, composed of visual observations. Samples in $\mathcal{D}_{i}$ come from the source context $\omega_{i}$, while samples in $\mathcal{D}_{j}$ come from the target context $\omega_{j}$. The model must output the observations in $\mathcal{D}_{j}$ conditioned on both $\mathcal{D}_{i}$ and the first observation $o^{j}_{0}$ from the target domain.

As will be explained next, the outputs of both the Time-Contrastive and the Context-Translation networks can be used to obtain an engineered reward function.

\begin{figure}[t]
    \centering
    \begin{subfigure}[b]{0.50\textwidth}
        \centering
        \includegraphics[width=\textwidth]{figures/images/view_point_mismatch/time-contrastive-network.jpg}
        \caption{\textit{Time-Contrastive network}, proposed in~\cite{sermanet2018time_contrastive}.}
        \label{fig:time_contrastive}
    \end{subfigure}
    \hfill
    \begin{subfigure}[b]{0.45\textwidth}
        \includegraphics[width=\textwidth]{figures/images/view_point_mismatch/context-translation-model.jpg}
        \caption{\textit{Context-Translation network}, proposed in~\cite{liu2018imitation_from_observation}}
        \label{fig:context-translation}
    \end{subfigure}
    \caption{Examples of how the mismatch between demonstrator viewpoint and learner viewpoint can be handled.}
    \label{fig:differet_viewpoint}
\end{figure}


The third question concerns the method by which the final learned policy is derived. To present the different approaches, it is essential to distinguish between \textit{Model-Free} and \textit{Model-Based} methods (Figure \ref{fig:lfo_taxonomy}).

\begin{figure}[t]
    \centering
    \includegraphics[width=0.8\textwidth]{figures/images/lfo_taxonomy.png}
    \caption{Learning from Observation taxonomy}
    \label{fig:lfo_taxonomy}
    
\end{figure}


\paragraph*{Model-Free}\mbox{}\\
Model-Free methods are characterized by the fact that they do not leverage knowledge about the environment dynamics, which can either be given a priori or learned through data-driven approaches. A further classification must be done between \textit{Reward Engineering} and \textit{Adversarial Learning} approaches.

\textbf{Reward Engineering} methods are characterized by the use of a hand-designed reward function to train the policy according to the Reinforcement Learning paradigm \cite{sutton2018reinforcement}. Methods based on this approach include \cite{liu2018imitation_from_observation,sermanet2018time_contrastive,xiong2021learning_by_watching,zakka2022xirl}.

In \cite{liu2018imitation_from_observation}, the reward function is defined as in Formula \ref{eq:liu2018imitation_from_observation_reward}.
\begin{equation}
    \label{eq:liu2018imitation_from_observation_reward}
    \begin{aligned}
        R(o^{l}_{t}) = -\left\|Enc_{1}(o^{l}_{t}) - \frac{1}{n} \sum_{i=1}^{n}F(o_{t}^{i},o_{0}^{l})\right\|^{2}_{2} - \\ w_{rec} \left\|o^{l}_{t} - \frac{1}{n} \sum_{i=1}^{n}M(o_{t}^{i},o_{0}^{l})\right\|^{2}_{2}
    \end{aligned}
\end{equation}
The first term is the classic Feature Tracking reward function, which aims to minimize the Euclidean Distance between the encoding of the current learner observation $o^{l}_{t}$ and the encoding of the demonstration in the learner context. The second term penalizes the policy for experiencing observations that differ from the translated observation.

In \cite{sermanet2018time_contrastive}, the reward function is defined according to Formula \ref{eq:sermanet2018time_contrastive_reward}.
\begin{equation}
    \label{eq:sermanet2018time_contrastive_reward}
    \begin{aligned}
        R(\textbf{v}_{t}, \textbf{w}_{t}) = - \alpha \left\| \textbf{w}_{t} - \textbf{v}_{t} \right\|^{2}_{2} - \beta \sqrt{\gamma + \left\| \textbf{w}_{t} - \textbf{v}_{t} \right\|^{2}_{2}}
    \end{aligned}
\end{equation}
Where $\textbf{v}_{t}$ is the TCN embedding of the video demonstration at timestep $t$, and $\textbf{w}_{t}$ is the TCN embedding produced by the robot observation (Figure \ref{fig:time_contrastive}).

In \cite{xiong2021learning_by_watching}, a \textbf{keypoint-representation} (Figure \ref{fig:lbw}) is obtained for both the current robot observations $z_{t}$ and each frame of the translated demonstration video $\{z^{E}_{p}\}_{p=1}^{T}$. The reward is then computed as in Formula \ref{eq:xiong2021learning_by_watching_reward}.
\begin{equation}
    \label{eq:xiong2021learning_by_watching_reward}
    \begin{aligned}
        R(z_{t},z_{t+1},z^{E}) = - \lambda_{1} \min_{p} \left\|z_{t}-z^{E}_{p}\right\| - \\ \lambda_{2} \min_{p} \left\|(z_{t+1}-z_{t}) - (z^{E}_{p+1}-z^{E}_{p})\right\|
    \end{aligned}
\end{equation}
The main idea is to generate actions that minimize the distance between the translated keypoints and the keypoints obtained from the current robot observation, thereby reproducing the demonstrated trajectory. The \textit{min} operator is necessary because the robot and the demonstration are not temporally aligned; there is no a priori knowledge about which demonstration frame corresponds to the current agent state.

\begin{figure}[t]
    \centering
    \includegraphics[width=0.8\textwidth]{figures/images/learning_by_watching/learning_by_watching.jpg}
    \caption{Architecture proposed by~\cite{xiong2021learning_by_watching}}
    \label{fig:lbw}
\end{figure}

In \cite{zakka2022xirl}, the reward function was defined as $R(s_{t}) = -\frac{1}{k} \ || \phi(s_{t}) - g||^{2}_{2}$, where $g$ is the goal embedding, defined as the mean embedding of the last frame of all the demonstration videos in the dataset, while $\phi(s_{t})$ is the embedding of the current observation. Experimental results, on simulation data proved that the proposed method can be used to learn tasks from cross-embodiment demonstrations, outperforming baseline \cite{sermanet2018time_contrastive} in terms of both sample efficiency and performance.


\paragraph*{Adversarial Learning} methods rely on the Adversarial Learning paradigm and are closely related to the GAIL methods (Section \ref{sec:gail}). Unlike the methods discussed in the GAIL section, these methods do not assume access to the demonstrator's actions. Preliminary works in this area have been proposed in \cite{merel2017learning,torabi2018gaifo}.

The goal of authors in \cite{merel2017learning} was to demonstrate that the Adversarial Learning setting can be effectively used even without action information. To test this hypothesis, the authors conducted a series of experiments in simulation for a walking task, where the same RL policy was trained in two contexts: one where the Discriminator had access to the (state, action) pair, and another where the Discriminator had access to state-only demonstrations. The results showed no substantial difference between the two settings, supporting the hypothesis that the essential information for task learning is contained in the state.

\begin{algorithm}[b]
\caption{GAIfO algorithm \cite{torabi2018gaifo}}
\label{alg:gaifo_algorithm}
\begin{algorithmic}
\Require Initial policy $\pi^{L}_{\phi}$, Initial Discriminator $D_{\theta}$
\Require State-only expert demonstration trajectories $\tau^{E} = \left \{ (s,s') \right \}$
\While {Policy Improves}
    \State Execute $\pi^{L}_{\phi}$ and collect state transitions $\tau^{L} = \left \{ (s,s') \right \}$
    \State Update $D_{\theta}$, with $\mathcal{L}_{D_{\theta}} = - \ ( \ \mathbb{E}_{\tau^{L}}[\log (D_{\theta}(s, s')) ] + \mathbb{E}_{\tau^{E}}[\log(1 - D_{\theta}(s, s'))] \ )$
    \State Update $\pi^{L}_{\phi}$, with reward $ r_{\pi^{L}_{\phi}} = - \ ( \ \mathbb{E}_{\tau^{L}}[\log(D_{\theta}(s, s'))] \ )$
\EndWhile
\end{algorithmic}
\end{algorithm}

The next significant work was proposed by the authors of \cite{torabi2018gaifo}, who formalized the \textit{GAIfO} algorithm (Algorithm \ref{alg:gaifo_algorithm}), an extension of GAIL \cite{ho2016gail} to state-only demonstrations. The proposed algorithm was used to train a network to solve tasks in a simulation environment \cite{brockman2016openai}, with both low-dimensional state representation and visual-state representation. Results regarding the number of demonstrated trajectories are reported in Figure \ref{fig:gaifo_results}.

As noted from the results, GAIfO outperforms previous observation-based methods \cite{sermanet2018time_contrastive,torabi2018bco} in settings with a low number of expert trajectories. The main drawback of GAIfO is the \textbf{high number of environmental interactions} needed to learn a policy, as it uses the model-free TRPO \cite{schulman2015trpo} algorithm. This issue was addressed by DEALIO \cite{torabi2021dealio}, which replaced the model-free algorithm with PILQR \cite{chebotar2017pilqr}, a model-based RL algorithm discussed next.

\begin{figure}[tb]
     \centering
     \begin{subfigure}[b]{0.8\textwidth}
        \centering
         \includegraphics[width=\textwidth]{Figures/images/gaifo_results/gaifo_results.jpg}
         \caption{Experimental results in low-dimensional state space}
         \label{fig:low_dimensional}
     \end{subfigure}
     \vfill
     \begin{subfigure}[b]{0.8\textwidth}
        \centering
         \includegraphics[width=\textwidth]{Figures/images/gaifo_results/gaifo_results_visual.jpg}
         \caption{Experimental results in high-dimensional state space}
         \label{fig:high_dimensional}
     \end{subfigure}
    \hfill
    \caption{Experimental results reported in \cite{torabi2018gaifo}.}
    \label{fig:gaifo_results}
\end{figure}



\paragraph*{Model-Based}\mbox{}\\
Model-Based methods leverage knowledge about the environment dynamics, which is learned through data-driven approaches. These methods can be further classified into \textit{Inverse Dynamic Model} and \textit{Forward Dynamic Model} approaches. 

The \textit{Inverse Dynamic Model} approach, given a transition $(s_{t}, s_{t+1})$, obtains a function $M$ that maps state transitions to actions, i.e., $a_{t} = M(s_{t}, s_{t+1})$. In contrast, the \textit{Forward Dynamic Model} approach, given a state-action pair $(s_{t}, a_{t})$, aims to learn a function $F$ that generates the next state $s_{t+1}$, i.e., $s_{t+1} = F(s_{t}, a_{t})$.

\textbf{Inverse Dynamic Model} methods include \cite{nair2017combining,torabi2018bco,guo2019hybrid_rl,radosavovic2021state_only_demo}.

In \cite{nair2017combining}, the goal was to develop a system capable of tying a knot in a rope. A self-supervised learning approach was used to train a Convolutional Neural Network. Given a pair of images $(I_{t}, I_{t+1})$ representing two successive rope states, the network was able to determine the action required to transition from state $I_{t}$ to state $I_{t+1}$. The network was trained using a dataset of 30K tuples $(I_{t}, a_{t}, I_{t+1})$ collected via an exploratory policy.

Authors in \cite{torabi2018bco} proposed a general approach, depicted in Figure \ref{fig:bco}, composed of two main parts: the learned Inverse Dynamic Model, $M_{\theta}$, and the learned policy $\pi_{\phi}$. The learning procedure is iterative. The model $M_{\theta}^{i}$ is updated by maximizing the probability $p_{\theta}(a_{t}|s_{t}, s_{t+1})$, using tuples $(s_{t}, a_{t}, s_{t+1})$ collected by the current policy. Once the dynamic model is updated, it infers the action $\tilde{a}_{t}$ given the demonstrations. The policy, having access to both state and action information, is then trained using classic Behavioral Cloning (BC) by optimizing the policy parameters through maximum-likelihood estimation $\phi^{*} = \underset{\phi}{argmax} \prod_{i=0}^{N} \pi^{L}_{\phi}(\tilde{a}_{i}|s_{i})$.

In \cite{guo2019hybrid_rl}, a similar approach to \cite{torabi2018bco} was used, but the agent's policy was trained using a combination of Behavioral Cloning and the Advantage Actor Critic (A2C) objective function \cite{mnih2016a2c} (Formula \ref{eq:a2c_reward}).
\begin{equation}
    \label{eq:a2c_reward}
    \begin{aligned}
        \mathcal{L}^{hyb}_{\theta} = \mathbb{E}_{s,a} \left[ A(s)\log(a|s;\theta) + \alpha \ \mathcal{H}(\pi^{L}(.|s)) \right] 
        + \\ \mathbb{E}_{(\hat{s}_{t}, \hat{s}_{t+1}) \sim D} \left[ \log(\pi^{L}(M(\hat{s}_{t}, \hat{s}_{t+1})|\theta)) \right].
    \end{aligned}
  \end{equation}
The main drawback is the assumption of access to the reward function, which can limit its applicability in real-robot manipulation tasks.

In \cite{radosavovic2021state_only_demo}, the work in \cite{Rajeswaran18_learning_complex_dexterous} was extended to state-only demonstrations, and the \textit{State-Only Imitation Learning} (SOIL) algorithm was proposed. This method addresses complex dexterous manipulation tasks, such as object reallocation, tool use, in-hand manipulation, and door opening, using a simulated humanoid hand. A neural network representing the Inverse Dynamic Model was trained by minimizing the L2-loss, given the action performed during the policy rollout. The policy was then updated according to the \textit{Demo Augmented Policy Gradient} (DAPG) method \cite{Rajeswaran18_learning_complex_dexterous}, adapted for state-only demonstrations, which follows the gradient updates described by Formula \ref{eq:dapg_updates}.
\begin{equation}
    \label{eq:dapg_updates}
    \begin{aligned}
        g_{SOIL} = g + \lambda_{0} \ \lambda_{1}^{k} \ \sum_{(s_{t}, \tilde{a}_{t}) \in D'} \nabla_{\theta}\log\pi^{L}_{\theta}(\tilde{a}_{t}, s_{t}),
    \end{aligned}
  \end{equation}

where $g$ is the Natural Policy Gradient term. The idea is to leverage the demonstrations at the beginning of the training and then exploit the RL algorithm to improve the behavior. Experiments performed in simulation showed that, compared to pure RL, the proposed method converges faster and produces more human-like behaviors.

\begin{figure}[tb]
    \centering
    \includegraphics[width=0.6\textwidth]{figures/images/bco/bco.jpg}
    \caption{Representation of the learning procedure proposed by~\cite{torabi2018bco}.}
    \label{fig:bco}
\end{figure}


\textbf{Forward Dynamic Model} methods include \cite{smith2019avid,torabi2021dealio}.

In \cite{smith2019avid}, once the human video demonstration was translated into the corresponding robot video, the policy was learned using the model-based RL algorithm SOLARIS \cite{zhang2019solar}. This algorithm optimizes a controller using the Linear-Quadratic Regulator (LQR) procedure. The policy optimization occurs in a low-dimensional, highly regularized \textit{latent space}, generated using Variational Inference \cite{Kingma2014_vae}. Starting from a sequence of observations and actions, a Global Dynamic Model over the latent trajectory is obtained. Then, given the Latent Dynamic Model, a Linear-Gaussian Controller is derived using LQR-FLM \cite{levine2014lqr_flm}. Real-world robotic experiments demonstrated that with just \textbf{2 hours} of robot interaction, it was possible to outperform previous works such as \cite{sermanet2018time_contrastive,torabi2018bco} and classic BC algorithms in tasks like "coffee making" (Figure \ref{fig:embo_mismatch}) and cup-retrieving, where the robot retrieves a cup from a closed drawer.

In \cite{torabi2021dealio}, the sample inefficiency problem of GAIfO \cite{torabi2018gaifo} was addressed. The approach combined the adversarial learning setting with state-only demonstrations, which had shown promising results (Figure \ref{fig:gaifo_results}), with a more data-efficient RL algorithm like PILQR \cite{chebotar2017pilqr}. PILQR's core is the LQR optimization procedure. Generally, it returns a \textit{linear-gaussian controller} (Formula \ref{formula:linear_gaussian_controller}) that optimizes a \textit{quadratic-cost function} (Formula \ref{formula:quadratic_cost_function}) under the assumption of \textit{linear-gaussian dynamics} (Formula \ref{formula:gaussian_dyn}).

\begin{equation}
\label{formula:linear_gaussian_controller}
    \pi(a_{t}|s_{t}) = \mathcal{N}(K_{t}s_{t} + k_{t}, S_{t})
\end{equation}

\begin{equation}
\label{formula:quadratic_cost_function}
c(s_{t},a_{t}) = \begin{bmatrix}
s_{t}
\\ 
a_{t}
\end{bmatrix}^{T}C_{t}\begin{bmatrix}
s_{t}
\\ 
a_{t}
\end{bmatrix} + \begin{bmatrix}
s_{t}
\\ 
a_{t}
\end{bmatrix}^{T} c_{t} + cc_{t}
\end{equation}

\begin{equation}
\label{formula:gaussian_dyn}
s_{t+1} \sim P(s_{t+1}|s_{t},a_{t}) = \mathcal{N}(F_{t}
\begin{bmatrix}
s_{t}
\\ 
a_{t}
\end{bmatrix} + f_{t}, \Sigma_{t})
\end{equation}




To use this framework, the linear-gaussian dynamic model was fitted using the current policy rollouts. Then, to obtain a quadratic cost function as needed by LQR, the dynamic model was used to express the modified discriminator output (Formula \ref{formula:output_discriminator}) as a function of the pair $(s_{t}, a_{t})$.

\begin{equation}
\label{formula:output_discriminator}
D_{\theta}(s_{t},s_{t+1}) = \frac{1}{2}
\begin{bmatrix}
s_{t}
\\ 
s_{t+1}
\end{bmatrix}^{T}C^{ss}(s_{t},s_{t+1})
\begin{bmatrix}
s_{t}
\\ 
s_{t+1}
\end{bmatrix} + \begin{bmatrix}
s_{t}
\\ 
s_{t+1}
\end{bmatrix}^{T} c^{ss}(s_{t}, s_{t+1})
\end{equation}

Experiments performed in simulation with low-dimensional state spaces showed promising results (Figure \ref{fig:dealio_performance}) in terms of sample efficiency compared to the GAIfO baseline. However, improvements are needed to:
\begin{enumerate*}[label=\textbf{(\arabic*)}]
    \item Reduce variance to make the learning process more reliable,
    \item Increase overall performance,
    \item Adapt the algorithm for real-world robot manipulation tasks.
\end{enumerate*}

\begin{figure}[t]
    \centering
    \begin{subfigure}[b]{0.8\textwidth}
        \centering
        \includegraphics[width=\textwidth]{Figures/images/dealio/dealio_performed_task.jpg}
        \caption{Control Tasks solved in~\cite{torabi2021dealio}.}
        \label{fig:dealio_task}
    \end{subfigure}
    \vfill
    \begin{subfigure}[b]{0.8\textwidth}
        \includegraphics[width=\textwidth]{Figures/images/dealio/dealio_performance.jpg}
        \caption{Performance of DEALIO~\cite{torabi2021dealio} compared against GAIfO~\cite{torabi2018gaifo}, with respect to the number of trajectories sampled during the learning process.}
        \label{fig:dealio_performance}
    \end{subfigure}
    \caption{DEAILO: (\ref{fig:dealio_task}) Control Tasks, (\ref{fig:dealio_performance}) Performance Level.}
    \label{fig:dealio}
\end{figure}








