\subsection{Learning from demonstration}
\label{sec:lfd}
\subsubsection{Behavioral Cloning}
This section is dedicated to the presentation of \textit{ Behavioral Cloning} (BC) methods.
\paragraph*{Dynamical Movement Primitives}\mbox{}\\
Dynamical Movement Primitives (DMPs), offer a robust framework for encoding and reproducing complex movements through differential equations and attractor dynamics.
\paragraph*{Single-Task Imitation Learning}\mbox{}\\
The \textit{Single-Task Imitation Learning} refers to deep architecture designed to learn and replicate specific tasks from given demonstrations.
\paragraph*{Interactive Imitation Learning}\mbox{}\\
In \textit{Interactive Imitation Learning} the learning process is augmented by interaction with a teacher, allowing for real-time feedback and adjustments to improve performance.
\paragraph*{Multi-Task Imitation Learning}\mbox{}\\
\textit{Multi-Task Imitation Learning} enables the learning and execution of multiple tasks from a set of demonstrations, highlighting the scalability and versatility of these methods.
\paragraph*{Object-Oriented Imitation Learning}\mbox{}\\
\textit{Object-Oriented Imitation Learning} focuses on learning behaviors in relation to specific objects and their interactions, providing a more structured and contextual approach to imitation learning.

\subsubsection{Inverse Reinforcement Learning}
This section will be dedicated to the introduction of the \textit{Inverse Reinforcement Learning} (IRL) approach. 

\subsubsection{Generative Adversarial Imitation Learning}
This section will be dedicated to the introduction of the \textit{Generative Adversarial Imitation Learning} (GAIL) approach. 

\subsubsection{Learning from Observation}
This section will be dedicated to the introduction of the \textit{Learning from Observation} (LfO) approach. 
\paragraph{Model-Free}
\paragraph{Model-Based}