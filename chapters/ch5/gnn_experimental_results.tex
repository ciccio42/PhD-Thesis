\section{Experimental Results}
\label{sec:gnn_experimental_results}
This section presents preliminary results aimed at evaluating the performance and limitations of using GNNs for learning heuristics, specifically through the STRIPS-HGN method proposed in \cite{shen2020learning}.

The method was tested in the well-known \textit{BlockWorld} domain \cite{slaney2001blocks}. This domain consists of a single object type, the \textbf{block}. 
This domain contains five predicates, and four actions that are listed below: 

\begin{itemize}
    \item \textit{On}: True if block $x$ is on block $y$.
    \item \textit{On-Table}: True if block $x$ is on the table.
    \item \textit{Clear}: True if block $x$ has no object on top, making it available for manipulation.
    \item \textit{Handempty}: True if the robotic gripper is not holding any object.
    \item \textit{Holding}: True if the robotic gripper is holding block $x$.

    \item \textit{Pick-Up}: Picks up block $x$ if it is clear, on the table, and the gripper is empty.
    \item \textit{Put-Down}: Places the block currently held by the gripper onto the table.
    \item \textit{Stack}: Stacks block $x$ (held by the gripper) onto a clear block $y$.
    \item \textit{Unstack}: Removes block $x$ from on top of block $y$ (if block $x$ is clear) and picks it up with the gripper.
\end{itemize}

For the experiments, 50 problem instances of varying complexity were generated, with complexity measured by the number of blocks, ranging from 3 to 15. The STRIPS-HGN was trained on a subset of problems with block counts between 5 and 10. Testing was conducted in two scenarios:

\begin{itemize}
    \item \textit{In-Training Distribution}. The model was tested on new problems with complexities within the same range as the training data.
    \item \textit{Out-of-Training Distribution}. The model was tested on problems with unseen complexities (3, 4, 11, 12, 13, and 15 blocks).
\end{itemize}

The ground-truth heuristic ($h^{*}$) was computed by solving each problem instance using the Scorpion planning solver \cite{seipp2020saturated}, in a configuration that ensure the optimality of the solution. In this case, the heuristic is represented as the number of actions needed to reach the goal state from the initial state.
% configured with the LAMA search algorithm \cite{richter2010lama}. While this setup does not guarantee path optimality, it provides the most reliable method for obtaining a solution across all 50 problem instances, regardless of task complexity, within a 16-minute time limit per instance.

The STRIPS-HGN performance was compared against several classic heuristics, including:
$h_{max}$, an admissible heuristic; $h_{add}$, a generally faster heuristic; $LM-cut$ \cite{richter2010lama}, a heuristic offering a balance between path optimality and speed. The STRIPS-HGN was also compared against itself with different numbers of refining steps (1, 5, and 10).

Moreover, different metrics have been proposed to compare the different algorithms. Specifically, the algorithms are compared with respect to:
\begin{itemize}
    \item \textit{Number of expanded nodes}, this measures the number of nodes that are traversed during the exploration of the state-space in order to find a given solution. This number is accumulated over the different problems of a given complexity, then the average and standard deviation is computed.
    \item \textit{Search time}, this measures the time needed for the algorithm to return a valid solution.  This number is accumulated over the different problems of a given complexity, then the average and standard deviation is computed.
    \item \textit{Plan length}, this measures the number of actions that are needed to reach a solution. This number is accumulated over the different problems of a given complexity, then the average and standard deviation is computed.
    \item \textit{Ratio of solution found}, this measures the ratio of number of times the search algorithm with a given heuristic returns a solution before a timeout of 5 minutes over all the number of problems.
\end{itemize}

\paragraph*{In-Training Distribution Generalization}\mbox{}\\
The first generalization test performed was related to the In-Training Distribution Generalization. This means that the algorithms are tested on 10 novel problem instances for each problem complexity.
The first results are related to the number of solution found, reported in Table \ref{table:in_distribution_cnt_success}. It can be noted how most of the heuristics are able to completely return a solution in the given timeout. This does not happen for the admissible heuristic $h_{max}$, which starts to not return a valid solution, starting from the complexity of 9 boxes. 

\begin{table}[t]
    \centering
    % \refstepcounter{table}
    \caption{Comparison of ``ratio of solution found'' on BlocksWorld domain across different complexities (Comp.). The table presents the results for $h_{max}$, $h_{add}$, $LM-cut$, and Strips-HGN with 1, 5, and 10 steps.}
    \label{table:in_distribution_cnt_success}
    \resizebox{\linewidth}{!}{%
    \begin{tabular}{|c|c|c|c|c|c|c|} 
    \hline
    \textbf{Comp.} & $\mathbf{h_{max}}$ & $\mathbf{h_{add}}$ & $\mathbf{LM-cut}$ & $\mathbf{Strips-HGN_1}$ & $\mathbf{Strips-HGN_5}$ & $\mathbf{Strips-HGN_{10}}$ \\ 
    \hhline{|=======|}
    5 & 1.00 & 1.00 & 1.00 & 1.00 & 1.00 & 1.00 \\ 
    \hline
    6 & 1.00 & 1.00 & 1.00 & 1.00 & 1.00 & 1.00 \\ 
    \hline
    7 & 1.00 & 1.00 & 1.00 & 1.00 & 1.00 & 1.00 \\ 
    \hline
    8 & 1.00 & 1.00 & 1.00 & 1.00 & 1.00 & 1.00 \\ 
    \hline
    9 & 0.00 & 1.00 & 1.00 & 1.00 & 1.00 & 1.00 \\ 
    \hline
    10 & 0.10 & 1.00 & 0.9 & 1.00 & 1.00 & 1.00 \\
    \hline
    \end{tabular}
    }
    \end{table}

To provide more insight into the generalization capabilities of STRIPS-HGN with respect to the training distribution, Tables \ref{table:in_distribution_avg_search_time}, \ref{table:in_distribution_avg_plan_length}, \ref{table:in_distribution_avg_nodes_expanded}, report key metrics, including search time, plan length, respectively, and the number of expanded nodes.

\begin{table}[t]
    \centering
    \caption{Mean and standard deviation of ``search time''  on the BlocksWorld domain across different complexities. The table presents the results for $h_{max}$, $h_{add}$, $LM-cut$, and Strips-HGN with 1, 5, and 10 steps.}
    \label{table:in_distribution_avg_search_time}
    \resizebox{\linewidth}{!}{%
    \begin{tabular}{|c|c|c|c|c|c|c|} 
    \hline
    \textbf{Complexity} & $\mathbf{h_{max}}$ & $\mathbf{h_{add}}$ & $\mathbf{LM-cut}$ & $\mathbf{Strips-HGN_1}$ & $\mathbf{Strips-HGN_5}$ & $\mathbf{Strips-HGN_{10}}$ \\ 
    \hhline{|=======|}
    5 & $0.05 \pm 0.05$ & $\mathbf{0.01 \pm 0.01}$ & $0.05 \pm 0.06$ & $0.13 \pm 0.08$ & $0.31 \pm 0.10$ & $0.57 \pm 0.20$ \\ 
    \hline
    6 & $0.53 \pm 0.31$ & $\mathbf{0.04 \pm 0.02}$ & $0.18 \pm 0.13$ & $0.20 \pm 0.07$ & $0.56 \pm 0.24$ & $1.05 \pm 0.41$ \\ 
    \hline
    7 & $4.10 \pm 2.99$ & $\mathbf{0.08 \pm 0.03}$ & $0.58 \pm 0.62$ & $0.36 \pm 0.18$ & $1.04 \pm 0.55$ & $1.74 \pm 0.97$ \\ 
    \hline
    8 & $95.44 \pm 53.45$ & $\mathbf{0.34 \pm 0.19}$ & $5.94 \pm 6.98$ & $1.68 \pm 1.55$ & $5.52 \pm 4.60$ & $9.73 \pm 8.03$ \\ 
    \hline
    9 & $300.00 \pm 0.00$ & $\mathbf{0.88 \pm 0.47}$ & $13.10 \pm 18.61$ & $1.79 \pm 1.44$ & $2.10 \pm 1.06$ & $4.11 \pm 2.28$ \\ 
    \hline
    10 & $289.29 \pm 32.15$ & $\mathbf{1.44 \pm 1.14}$ & $69.89 \pm 114.03$ & $4.79 \pm 5.35$ & $8.90 \pm 13.89$ & $11.05 \pm 14.90$ \\
    \hline
    \end{tabular}
    }
    \end{table}

In terms of search time (Table \ref{table:in_distribution_avg_search_time}), $STRIPS-HGN_1$ outperforms the admissible heuristic $h_{max}$ and the trade-off heuristic $LM-cut$ for complexities above 7. The only heuristic faster than STRIPS-HGN is $h_{add}$, known for its speed. However, this comes at the expense of longer paths, as STRIPS-HGN variants generate shorter paths on average (Table \ref{table:in_distribution_avg_plan_length}). For complexities 9 and 10, the learned heuristics produce paths averaging around 23, while $h_{add}$ generates paths averaging 26.

It is particularly noteworthy that the learned heuristics not only generate shorter paths but also achieve this while performing less state-space exploration. Results in Table \ref{table:in_distribution_avg_nodes_expanded} demonstrate that STRIPS-HGN can generate a valid path using a significantly smaller number of nodes compared to traditional heuristics. This advantage becomes especially evident at a complexity of 10 boxes, where $STRIPS-HGN_{10}$ requires an average of only 95.70 nodes to produce a valid path, compared to the 366.20 nodes needed by the $h_{add}$ heuristic. This represents a substantial improvement over other heuristics, which typically need to expand a far greater number of nodes to achieve comparable results. 

Overall, the ability of the learned heuristics to produce shorter paths with fewer node expansions is a promising finding. It suggests that these heuristics are more efficient at exploring the state-space than traditional methods. The learned node representations can effectively guide the search algorithm toward the most relevant nodes, reducing the need to explore a large portion of the state-space to identify a valid path. This highlights the significant potential of learned heuristics for use in path-planning problems.


\begin{table}[t]
    \centering
    \caption{Mean and standard deviation of ``plan length'' on the BlocksWorld domain across different complexities. The table presents the results for $h_{max}$, $h_{add}$, $LM-cut$, and Strips-HGN with 1, 5, and 10 steps.}
    \label{table:in_distribution_avg_plan_length}
    \resizebox{\linewidth}{!}{%
    \begin{tabular}{|c|c|c|c|c|c|c|} 
    \hline
    \textbf{Complexity} & $\mathbf{h_{max}}$ & $\mathbf{h_{add}}$ & $\mathbf{LM-cut}$ & $\mathbf{Strips-HGN_1}$ & $\mathbf{Strips-HGN_5}$ & $\mathbf{Strips-HGN_{10}}$ \\ 
    \hhline{|=======|}
    5 & $10.20 \pm 2.09$ & $10.60 \pm 2.69$ & $10.20 \pm 2.09$ & $10.20 \pm 2.09$ & $10.20 \pm 2.09$ & $10.20 \pm 2.09$ \\ 
    \hline
    6 & $13.00 \pm 2.72$ & $14.80 \pm 3.25$ & $13.00 \pm 2.72$ & $13.00 \pm 2.72$ & $13.00 \pm 2.72$ & $13.00 \pm 2.72$ \\ 
    \hline
    7 & $15.40 \pm 2.69$ & $16.60 \pm 2.97$ & $15.40 \pm 2.69$ & $15.40 \pm 2.69$ & $15.40 \pm 2.69$ & $15.40 \pm 2.69$ \\ 
    \hline
    8 & $18.80 \pm 2.23$ & $20.60 \pm 3.10$ & $18.80 \pm 2.23$ & $18.80 \pm 2.23$ & $18.80 \pm 2.23$ & $18.80 \pm 2.23$ \\ 
    \hline
    9 & - & $26.00 \pm 3.69$ & $23.00 \pm 2.72$ & $23.00 \pm 2.72$ & $23.00 \pm 2.72$ & $23.00 \pm 2.72$ \\ 
    \hline
    10 & - & $26.40 \pm 3.67$ & $23.11 \pm 3.14$ & $23.60 \pm 3.32$ & $23.40 \pm 3.10$ & $23.40 \pm 3.10$ \\
    \hline
    \end{tabular}
    }
    \end{table}
\begin{landscape}
\begin{table}
    \centering
    \caption{Mean and standard deviation of ``number of expanded nodes'' on the BlocksWorld domain across different complexities (Comp.). The table presents the results for $h_{max}$, $h_{add}$, $LM-cut$, and Strips-HGN with 1, 5, and 10 steps.}
    \label{table:in_distribution_avg_nodes_expanded}
    \resizebox{\linewidth}{!}{%
    \begin{tabular}{|c|c|c|c|c|c|c|} 
    \hline
    \textbf{Comp.} & $\mathbf{h_{max}}$ & $\mathbf{h_{add}}$ & \begin{tabular}[c]{@{}c@{}}\textbf{LM-}\textbf{cut}\end{tabular} & \begin{tabular}[c]{@{}c@{}}\textbf{Strips-}\textbf{HGN\textsubscript{1}}\end{tabular} & \begin{tabular}[c]{@{}c@{}}\textbf{Strips-}\textbf{HGN\textsubscript{5}}\end{tabular} & \begin{tabular}[c]{@{}c@{}}\textbf{Strips-}\textbf{HGN\textsubscript{10}}\end{tabular} \\
    \hhline{|=======|}
    5 & $98.80 \pm 97.06$ & $22.00 \pm 14.68$ & $24.40 \pm 23.30$ & $16.50 \pm 10.31$ & $\mathbf{12.40 \pm 3.29}$ & $12.90 \pm 3.70$ \\ 
    \hline
    6 & $643.70 \pm 430.89$ & $38.70 \pm 22.12$ & $35.10 \pm 22.56$ & $17.50 \pm 5.16$ & $\mathbf{17.10 \pm 6.77}$ & $17.30 \pm 6.42$ \\ 
    \hline
    7 & $3918.80 \pm 3124.33$ & $54.40 \pm 23.86$ & $83.70 \pm 100.14$ & $29.80 \pm 15.96$ & $27.90 \pm 12.29$ & $\mathbf{27.90 \pm 12.19}$ \\ 
    \hline
    8 & $77988.50 \pm 53611.24$ & $142.70 \pm 64.76$ & $481.60 \pm 558.93$ & $\mathbf{109.90 \pm 98.16}$ & $126.40 \pm 100.73$ & $119.90 \pm 94.02$ \\ 
    \hline
    9 & $154351.20 \pm 3673.33$ & $307.70 \pm 176.77$ & $758.90 \pm 1073.05$ & $99.70 \pm 94.47$ & $\mathbf{44.00 \pm 20.83}$ & $47.30 \pm 24.00$ \\ 
    \hline
    10 & $103801.70 \pm 11155.55$ & $366.20 \pm 320.12$ & $2799.70 \pm 4595.73$ & $235.90 \pm 270.38$ & $150.80 \pm 229.64$ & $\mathbf{95.70 \pm 110.71}$ \\
    \hline
    \end{tabular}
    }
    \end{table}
\end{landscape}


\paragraph*{Out-of-Training Distribution Generalization} \mbox{}\\
In the second step, we evaluated the generalization capabilities of the STRIPS-HGN on problem instances with complexities outside the training distribution. For each task complexity, we considered 50 problem instances. The results are summarized in Table \ref{table:out_distribution_cnt_success}.

\begin{table}[t]
    \centering
    \caption{Comparison of ``ratio of solution found'' on BlocksWorld domain in out-of-training distribution scenario. The table presents the results for $h_{max}$, $h_{add}$, $LM-cut$, and Strips-HGN with 1, 5, and 10 steps.}
    \label{table:out_distribution_cnt_success}
    \resizebox{\linewidth}{!}{%
    \begin{tabular}{|c|c|c|c|c|c|c|} 
    \hline
    \textbf{Comp.} & $\mathbf{h_{max}}$ & $\mathbf{h_{add}}$ & $\mathbf{LM-cut}$ & $\mathbf{Strips-HGN_1}$ & $\mathbf{Strips-HGN_5}$ & $\mathbf{Strips-HGN_{10}}$ \\ 
    \hhline{|=======|}
    3 & 1.00 & 1.00 & 1.00 & 1.00 & 1.00 & 1.00 \\ 
    \hline
    4 & 1.00 & 1.00 & 1.00 & 1.00 & 1.00 & 1.00 \\ 
    \hline
    11 & 0.00 & 1.00 & 0.76 & 0.98 & 0.98 & 0.98 \\ 
    \hline
    12 & 0.00 & 1.00 & 0.46 & 0.92 & 0.96 & 0.90 \\ 
    \hline
    13 & 0.00 & 1.00 & 0.36 & 0.84 & 0.82 & 0.72 \\ 
    \hline
    14 & 0.00 & 1.00 & 0.2 & 0.62 & 0.48 & 0.60 \\ 
    \hline
    15 & 0.00 & 0.98 & 0.10 & 0.48 & 0.02 & 0.08 \\
    \hline
    \end{tabular}
    }
    \end{table}

As shown in Table \ref{table:out_distribution_cnt_success}, STRIPS-HGN exhibits weaker generalization capabilities compared to the  $h_{add}$ heuristic, which consistently returns valid solutions in most cases. Among the STRIPS-HGN variants, only $ STRIPS-HGN_1 $ consistently produces a reasonable number of valid solutions across all complexities, achieving a success ratio of 0.48 for the complexity involving 15 boxes. However, this performance is notably lower than the $ h_{add} $ heuristic, which achieves a success ratio of 0.98.

To assess whether this behavior stems from overfitting to optimal solutions, we trained a STRIPS-HGN variant using suboptimal heuristics from the Scorpion solver with a non-optimal configuration. The results are shown in Table \ref{table:out_distribution_cnt_success_no_optimal}. Comparing $STRIPS-HGN_1$ performance in Table \ref{table:out_distribution_cnt_success} and Table \ref{table:out_distribution_cnt_success_no_optimal} reveals that the model trained with suboptimal heuristics generalizes better to unseen complexities. Notably, for the 15-box complexity, the success rate increases to 0.82, compared to 0.48 for the model trained with optimal heuristics.

\begin{table}[t]
    \centering
    \caption{Comparison of ``ratio of solution found'' on BlocksWorld domain in out-of-training distribution scenario. Here the STRIPS-HGN is trained with suboptimal heuristics.}
    \label{table:out_distribution_cnt_success_no_optimal}
    \resizebox{\linewidth}{!}{%
    \begin{tabular}{|c|c|c|c|} 
    \hline
    \textbf{Comp.} & $\mathbf{Strips-HGN_1}$ & $\mathbf{Strips-HGN_5}$ & $\mathbf{Strips-HGN_{10}}$ \\ 
    \hhline{|====|}
    3 & 1.00 & 1.00 & 1.00 \\ 
    \hline
    4 & 1.00 & 1.00 & 1.00 \\ 
    \hline
    11 & 1.00 & 1.00 & 1.00 \\ 
    \hline
    12 & 1.00 & 1.00 & 0.98 \\ 
    \hline
    13 & 1.00 & 0.82 & 0.76 \\ 
    \hline
    14 & 0.96 & 0.06 & 0.00 \\ 
    \hline
    15 & 0.82 & 0.00 & 0.00 \\
    \hline
    \end{tabular}
    }
    \end{table}

To further evaluate the quality of STRIPS-HGN, a comparison between average plan length and search time was conducted between $h_{add}$ and $STRIPS-HGN_{1}$. Specifically, the problems considered are those that the optimally-trained \\ $STRIPS-HGN_{1}$ was able to solve.

\begin{table}[t]
    \centering
    \caption{Mean and standard deviation of ``search time'' on the BlocksWorld domain across different complexities. The table presents the results for $h_{add}$, $Strips-HGN_{1}$ trained with optimal-heuristic (Opt) and non-optimal-heuristic (No-Opt).}
    \label{table:out_distribution_avg_search_time}
    \resizebox{\linewidth}{!}{%
    \begin{tabular}{|c|c|c|c|} 
    \hline
    \textbf{Comp.} & $\mathbf{h_{add}}$ & \begin{tabular}[c]{@{}c@{}}$\mathbf{Strips-HGN_1}$\\(Opt)\end{tabular} & \begin{tabular}[c]{@{}c@{}}$\mathbf{Strips-HGN_1}$\\(No-Opt)\end{tabular} \\ 
    \hhline{|====|}
    3 & $\mathbf{0.001 \pm 0.001}$ & $0.027 \pm 0.015$ & $0.027 \pm 0.015$ \\ 
    \hline
    4 & $\mathbf{0.004 \pm 0.002}$ & $0.070 \pm 0.033$ & $0.049 \pm 0.034$ \\ 
    \hline
    11 & $\mathbf{3.906 \pm 5.127}$ & $15.819 \pm 31.468$ & $5.235 \pm 19.507$ \\ 
    \hline
    12 & $\mathbf{4.573 \pm 4.081}$ & $48.437 \pm 70.469$ & $15.426 \pm 42.410$ \\ 
    \hline
    13 & $\mathbf{13.592 \pm 19.182}$ & $52.113 \pm 77.722$ & $21.928 \pm 52.409$ \\ 
    \hline
    14 & $50.439 \pm 63.253$ & $45.355 \pm 66.579$ & $\mathbf{24.637 \pm 54.752}$ \\ 
    \hline
    15 & $29.362 \pm 40.305$ & $65.405 \pm 69.246$ & $\mathbf{27.988 \pm 57.192}$ \\ 
    \hline
    \end{tabular}
    }
\end{table}

\begin{table}[t]
    \centering
    \caption{Mean and standard deviation of ``path length'' on the BlocksWorld domain across different complexities. The table presents the results for $h_{add}$, $Strips-HGN_{1}$ trained with optimal heuristic (Opt) and non-optimal heuristic (No-Opt).}
    \label{table:out_distribution_avg_path_length}
    \resizebox{\linewidth}{!}{%
    \begin{tabular}{|c|c|c|c|} 
    \hline
    \textbf{Complexity} & $\mathbf{h_{add}}$ & \begin{tabular}[c]{@{}c@{}}$\mathbf{Strips-HGN_1}$\\(Opt)\end{tabular} & \begin{tabular}[c]{@{}c@{}}$\mathbf{Strips-HGN_1}$\\(No-Opt)\end{tabular} \\ 
    \hhline{|====|}
    3 & $4.72 \pm 2.55$ & $4.72 \pm 2.55$ & $\mathbf{4.72 \pm 2.55}$ \\ 
    \hline
    4 & $7.84 \pm 2.56$ & $7.84 \pm 2.56$ & $\mathbf{6.28 \pm 2.99}$ \\ 
    \hline
    11 & $31.06 \pm 4.18$ & $25.63 \pm 3.95$ & $\mathbf{12.64 \pm 9.69}$ \\ 
    \hline
    12 & $33.48 \pm 5.68$ & $28.87 \pm 4.29$ & $\mathbf{16.47 \pm 11.11}$ \\ 
    \hline
    13 & $37.86 \pm 4.37$ & $30.67 \pm 3.40$ & $\mathbf{18.99 \pm 11.53}$ \\ 
    \hline
    14 & $42.90 \pm 4.60$ & $33.55 \pm 3.89$ & $\mathbf{20.67 \pm 11.88}$ \\ 
    \hline
    15 & $44.17 \pm 5.44$ & $35.25 \pm 4.16$ & $\mathbf{21.87 \pm 12.12}$ \\
    \hline
    \end{tabular}
    }
    \end{table}

As observed in Table \ref{table:out_distribution_avg_search_time}, the baseline heuristic $h_{add}$ is generally faster than the learned heuristic across 5 out of 7 complexities. However, at higher complexities, specifically with 14 and 15 blocks, the learned heuristic becomes faster than the baseline. It is particularly noteworthy that the heuristic learned with non-optimal heuristics outperforms both the baseline heuristic and the optimally-trained learned heuristic at these higher complexities.

Regarding plan length, as shown in Table \ref{table:out_distribution_avg_path_length}, both learned heuristics generally produce shorter plans compared to the baseline. Moreover, the heuristic trained on non-optimal heuristics consistently generates shorter paths than the optimally-trained heuristic.

These findings highlight the potential of learned heuristics to generalize better to novel problem complexities and to scale more effectively as the dimensionality of the problem increases. In general, the results align with the literature discussed in Section \ref{sec:gnn_related_works}. However, further work is necessary to optimize the inference process of these methods, as their primary limitation remains the computational cost of inference, as demonstrated by the results.
