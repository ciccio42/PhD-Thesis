\documentclass[twoside,a4paper,12pt,openright]{book}

% pagina di ogni capitolo sia sulla pagina destra
% \usepackage[a4paper]{geometry}
\usepackage[inline]{enumitem}
\usepackage{style}
\usepackage[figuresright]{rotating}
\usepackage{tikz}
\usepackage[utf8]{inputenc}
\usetikzlibrary{arrows,automata}
\usetikzlibrary{calc} 
\usepackage{array}
\usetikzlibrary{shapes}
\usepackage{xcolor}
\usepackage{soul}
\usepackage{listings}
\usepackage{style}
\usepackage{epstopdf}
\usepackage{fancybox}
\usepackage{pifont}
\usepackage{amsmath,amssymb,amsthm}
\usepackage{eucal}
\usepackage{algorithm}  % for the algorithm environment
\usepackage{algorithmic}  % for the algorithmic environment
\usepackage{latexsym}
\usepackage{multicol}
\usepackage{multirow}
\usepackage{chngpage}
%\usepackage{natbib}
\usepackage[italian]{babel} % Togliere il commento se si scrive la tesi in italiano

\usepackage{graphicx} 
\usepackage{hhline}
%\usepackage{subfig}
%\usepackage{subfigure}
\usepackage{morefloats}
\usepackage{fancyhdr} % Package dello stile pagina
\usepackage{comment}
% \usepackage{enumerate}
\usepackage{subcaption}
\usepackage{amssymb}
\usepackage{todonotes}
\usepackage{makecell}
\usepackage{multirow}

\newcommand{\smalltodo}[1]{\todo[inline]{\small #1}}
\newcommand{\updated}[1]{{\color{blue}#1}}

\newcommand{\splitcell}[1]{\begin{tabular}{@{}c@{}}#1\end{tabular}}
\newcommand{\bsplitcell}[1]{$\left[\splitcell{#1}\right]$}

\usepackage{hyperref}
%\hypersetup{colorlinks=true,linkcolor=blue}

\graphicspath{ {./figures/}
		{./figures/problems/}}

\def\beq{\begin{equation}}   	% Definisce comando per inizio equazioni
\def\eeq{\end{equation}}			% Definisce comando per fine equazioni

% ------------- Sezione NUOVI COMANDI --------------------------
\definecolor{mygreen}{rgb}{0,0.6,0}

\newcommand{\ale}[1]{{\color{blue}#1}}

\newcommand{\ls}{\hspace{4mm}}

\newcommand{\clearemptydoublepage}{\newpage {\pagestyle{empty} \cleardoublepage}}

% Nuovo comando per le caption delle figure
\renewcommand\caption[1]{\refstepcounter{figure} \linespread{1}\footnotesize  {\bf \figurename\ \thefigure} \ #1 \addcontentsline{lof}{figure}{\thefigure \ls #1}} 

% Nuovo comando per le caption delle tabella
\newcommand\captiont[1]{\refstepcounter{table} \linespread{1}\footnotesize  {\bf \tablename\ \thetable} \ #1}

% ------------- Sezione NUOVI COMANDI (fine) -------------------


\setcounter{secnumdepth}{4}
\setcounter{tocdepth}{3}

\linespread{1} % Imposta l'interlinea singola per l'intero documento

\begin{document}

\pagestyle{plain}
\pagenumbering{roman}

%%%%INSERT FRONT PAGE
% \includepdf{Frontespizio.pdf}


\chapter*{Abstract}
% 5 PAGINE E UN PO

% \tableofcontents       % Inserisce l'indice della Tesi

% \clearemptydoublepage
% \thispagestyle{empty}

% \begin{flushright}
% 	\null\vspace{\stretch{1}}

% 	\emph{Will robots inherit the earth? Yes, but they will be our children.}

% 	\emph{}

% 	- Marvin Lee Minsky -

% 	\vspace{\stretch{5}}\null
% \end{flushright}

\pagestyle{fancy} % Abilita lo stile Fancy per le pagine

%\include{Acknowledgment} % Togliere commento per inserire i ringraziamenti (file esterno Ringraziamenti.tex)


% --- Definisce gli header per le pagine pari e dispari ---
\renewcommand{\sectionmark}[1]{\markright{\thesection.\ #1}}
%\renewcommand{\chaptermark}[1]{\markboth{\MakeUppercase{\chaptername\ \thechapter.\ #1}}{}}
\renewcommand{\chaptermark}[1]{\markboth{\thechapter.\ #1}{}}

\fancyhf{}

\fancyhead[LE,RO]{\footnotesize \slshape \bfseries \thepage}
\fancyhead[LO]{\footnotesize \slshape \bfseries \rightmark}
\fancyhead[RE]{\footnotesize \slshape \bfseries \leftmark}
\fancypagestyle{plain}{\fancyhead{} \renewcommand{\headrulewidth}{0pt}}

\renewcommand{\headrulewidth}{0.5pt}
\renewcommand{\footrulewidth}{0pt}
\addtolength{\headheight}{0.5pt}

% --- Fine definizione header


\pagenumbering{arabic} % Definisce il nuovo stile di numerazione delle pagine

% THEOREMS ----  Si ridefiniscono gli stili dei teor., lemmi corollari
\theoremstyle{plain}
\newtheorem{thm}{Theorem}[section]
\newtheorem{cor}[thm]{Corollary}
\newtheorem{lem}[thm]{Lemma}
\newtheorem{prop}[thm]{Proposition}
%
\theoremstyle{definition}
\newtheorem{defn}{Definition}[section]
%
\theoremstyle{remark}
\newtheorem{rem}{Remark}[section]


% Thesis chapters -------------------------------------------------------
% Ogni capitolo è scritto in un file .tex da includere.
% Tali file devono cominciare con il comando \chapter{}
% I file di Introduzione e Conclusione possono cominciare con \chapter*{}

%\addcontentsline{toc}{chapter}{Introduction}
%\thispagestyle{empty}

%\thispagestyle{empty}


%\addcontentsline{toc}{chapter}{State of the Art}
%\include{StateArt}
%\thispagestyle{empty}

%% Add content
% \section{Introduction}

\input{chapters/ch1/application_context.tex}

% \chapter{Conditioned object detector}
\label{ch:cod}
In this chapter the COD is going to be described. Specifically, Section~\ref{sec:cod_problem} will outline the detection problem being addressed. Section~\ref{sec:cod_architecture} will detail the proposed architecture designed to solve the described problem. Section~\ref{sec:cod_experimental} will discuss the experimental setup and present the results obtained from testing the proposed architecture.
\section{Problem formulation}
\label{sec:cod_problem}
As outlined in previous chapters, the primary objective of this thesis is to validate the possibility of using object priors to enhance the system ability to ignore distractor objects during task execution, while simultaneously providing human-readable information about the robot behavior.

In Section \ref{sec:ooil}, various methods that utilize object priors in the context of Imitation Learning have been discussed. However, all the methods described in that section rely on pre-trained object detectors to identify regions of interest, i.e., areas that potentially contain objects, regardless of the object class (e.g., box, nut, bin, peg, etc.) or its semantic state (i.e., whether the object is relevant to the task or merely a distractor). Furthermore, none of these methods are designed for multi-variation scenarios, where the semantic state of an object (whether it is relevant or a distractor) is dynamically determined by the requirements of the specific task.

In the context of interest, the object detection problem can be formulated as follows: given a pair consisting of an agent observation and a command, $\left( o_{t}^{a}, c_{m_{i}} \right)$,  a parameterized function $f_{\theta}$, must generate a set of bounding boxes (bb), $\left\{ bb^{j}_{t} | j \in C \right\}$ that identifies \textbf{specific} regions of interest, i.e., $\left\{ bb^{j}_{t} | j \in C \right\} = f_{\theta}(o_{t}^{a}, c_{m_{i}})$. Here, $C$ represents a set of classes assigned to the bounding boxes, which differs from the traditional object detection classes. Instead of predicting the object category (e.g., box, nut, bin, etc.), the focus is on predicting the \textbf{semantic attribute} associated with each object. In the most general case, these attributes are:

\begin{itemize} 
    \item \textit{Target}, it indicates that the bounding box refers to the object to be manipulated in the current task variation $c_{m_{i}}$. 
    \item \textit{No-Target}, it refers to any other object in the scene that is not being manipulated. 
    \item \textit{Target-Place}, it represents the final region of interest for the manipulated object, which varies depending on the task. For example, in a pick-and-place task, it refers to the bin where the object is placed; in a nut-assembly task, it refers to the peg where the object is inserted; and in a button-press task, it refers to the region where the button has been pressed. 
    \item \textit{No-Target-Place}, it refers to any other region that is not relevant to the current task variation, such as other bins in a pick-and-place task or other pegs in a nut-assembly task. 
\end{itemize}

The example shown in Figure \ref{fig:example_of_bb} highlights how the semantic attribute assigned to an object changes depending on the requested task $c_{m}$. In the first case, the green box and the first bin are marked as targets (green boxes), while other objects are labeled as distractors (red boxes). In the second case, even though the configuration remains the same, the roles of the objects changes. This is because, in the second task, the demonstrator manipulates the red box and places it into a different location. 

Since the goal is to explicitly manage the changing semantic meaning of objects, traditional object detectors, which predict bounding boxes and object categories for all objects in the scene, are not suitable. Instead, a novel, specialized architecture is required to implement the function $f_{\theta}$. The details of this implementation are provided in Section \ref{sec:cod_architecture}.
\begin{figure}[t]
    \centering
    \includegraphics[width=1.0\textwidth]{figures/images/ch2/example_of_bb.jpg}
    \caption{Example of bounding-boxes assignment, where green boxes refer to target object and placing location while red boxed refers to no-target and no-target-place. (Left) The demonstrator manipulates the green box, placing it into the first bin. (Right) The demonstrator manipulates the red box, placing it into the second bin. Note, how for a given agent environment state, the semantic attribute between objects changes.}
    \label{fig:example_of_bb}
\end{figure}




\section{Architecture}
\label{sec:cod_architecture}
\section{Experiments}
\label{sec:cod_experimental}
In this section, the performed experiments are going to be described. Specifically, in  
Section~\ref{sec:cod_dataset} the dataset used for training procedure will be described. Section~\ref{sec:cod_results} will report the obtained results.
\subsection{Dataset}
\label{sec:cod_dataset}

\subsection{Results}
\label{sec:cod_results}
This section presents the obtained results, divided into two main blocks. The first block (Section~\ref{sec:cod_tod}) discusses the results of the method trained to detect only the target object. The second block (Section~\ref{sec:cod_tod}) covers the results of the method trained to detect both the target object and the final (placing) location. For each method, results are reported for two different scenarios: first, where the method is trained in a single-task multi-variation scenario; and second, where the method is trained in a multi-task multi-variation scenario.

\subsubsection{Target object detector}
\label{sec:cod_tod}
\paragraph*{Single-task multi-variation scenario}\mbox{}\\
\paragraph*{Multi-task multi-variation scenario}\mbox{}\\

\subsubsection{Target object and final position detector}
\label{sec:cod_tofpd}
\paragraph*{Single-task multi-variation scenario}\mbox{}\\
\paragraph*{Multi-task multi-variation scenario}\mbox{}\\
% \chapter{Object conditioned control policy}
\label{ch:occp}
In this chapter, the main contribution of this thesis will be presented: the \textit{Object Conditioned Control Policy} (OCCP). As outlined in Section \ref{sec:motivation}, the central challenge addressed in this thesis is \textit{target-object misidentification} within Visual-Conditioned Multi-Task Imitation Learning systems. To tackle this issue, a modular control architecture is proposed, based on the hypothesis that separating the cognitive task from the control task into distinct modules can improve the overall robustness and interpretability of these systems.

Specifically, Section \ref{sec:occp_related_works} will provide an extensive and detailed review of prior research in both Multi-Task Imitation Learning and Object-Oriented Imitation Learning, which are the two fields most closely related to the proposed approach. Section~\ref{sec:ocpl_problem} will outline the problem being addressed, while Section~\ref{sec:ocpl_architecture} will describe the proposed architecture designed to solve this problem. Finally, Section~\ref{sec:ocpl_experimental} will discuss the experimental setup and present the results obtained from testing the proposed architecture.


\section{Related Works}
\label{sec:occp_related_works}
This Section will review and discuss the prior research performed to address the proble of Multi-Task Imitation Learning (Section \ref{sec:occp_mtil}) and Object-Oriented Imitation Learning (Section \ref{sec:ooil}), that are the two topics monstly related to the thesis proposal.

\subsection{Multi-Task Imitation Learning}
\label{sec:occp_mtil}
The methods discussed in the Section \ref{sec:lfd} describe architectures and approaches specifically designed to solve a single task, with limited generalization to the object category (e.g., picking blocks rather than balls) and initial state (e.g., the object starting position). For instance, a system trained for pick-and-place operations cannot be repurposed for tasks like assembly operations. The methods described here address these limitations, solving the problem of \textit{Multi-Task Imitation Learning} (MTIL).

Before starting to present and describe the different methods and approaches, it is necessary to describe the problem. Starting from a reformulation of the dataset used to train the system and the learned policy.

In Section~\ref{sec:problem_formulation}, the expert dataset $\mathcal{D}^{E}$ has been introduced. Based on the problem to solve this dataset can composed in different way. Specifically, in the context of MTIL, the dataset $\mathcal{D}^{E}$ can be seen as a composition of $n$ datasets, \\ $\mathcal{D}^{E}=\left \{\mathcal{D}_{1}, \dots, \mathcal{D}_{n}\right \}$, where \\ $\mathcal{D}_{i} = \left \{ (\tau_{m_{i}}^{j}, \ c_{m_{i}}), j=1,\dots,N, \  m_{i} \in \mathcal{M}_{i}\right \}$ is the \textit{single-task dataset}, composed of:
\begin{itemize}
    \item $N$ expert demonstration for each $m^{th}$ variation of the $i^{th}$ task, where $M_{i}$ is the number of variation for the $i^{th}$ task.
    \item Agent trajectories $\tau_{m_{i}}^{j} = (s_{0}, a_{0}, s_{1}, a_{1}, \dots, a_{T-1}, s_{T})$,\\ where $s_{t}$ is the state at time $t$ and $a_{t}$ is the corresponding action (Section \ref{sec:problem_formulation}).
    \item Task-conditioning signal $c_{m_{i}}$ for the $m^{th}$ variation of $i^{th}$ task, which describes the desired task in terms of video demonstrations \cite{james2018task_embedded,bhutani2022attentive_one_shot,dasari2021transformers_one_shot,mandi2022towards_more_generalizable_one_shot}, natural language description \cite{stepputtis2020language,jang2022bc_z,mees2022calvin,doasIcan2022,mees2022hulc,brohan2022rt,shridhar2023perceiver} or multi-modal prompt, that exploits both visual information (e.g., an image of the target object) and text information that contains the information related to the action to be performed \cite{jiang2023vima}.
\end{itemize}
The goal of Multi-Task Imitation Learning is to learn a \textit{conditioned policy} $\pi^{L}_{\theta}(a_{t}|s_{t}, c_{m_{i}})$, that is able to map the current state and command into the corresponding action.
Depending on how the policy is defined, various loss functions come into play. In the case of deterministic policies, the learning process focuses on minimizing the Mean-Squared Error (refer to Formula \ref{eq:mse}). However, for probabilistic policies, the learning process centers around minimizing the Negative Log-likelihood (refer to Formula \ref{eq:nll}). This approach aims to enhance the probability of correctly executing the action.
\input{formula/loss_functions.tex}
The following sections will describe the various approaches proposed to address the problem. Figure \ref{fig:mtil_taxonomy} illustrates the taxonomy used for the Multi-Task Imitation Learning methods. Specifically, the methods are categorized based on the type of generalization required by the algorithm (Few-Shot vs. Zero-Shot). For Few-Shot generalization, the Meta-Learning paradigm will be discussed, as it is most relevant to the problem at hand. For Zero-Shot generalization, the methods are further divided based on the type of conditioning signal used, whether it is provided through natural language descriptions or visual information.
\input{figures/ch1/mtil_taxonomy.tex}

\textbf{Few-Shot MTIL} refers to approaches designed to train a model on a variety of tasks so that it can effectively solve a new task using only few samples and consequently requires only a few back-propagation steps \cite{finn2017maml}. In this context, one of the most significant learning paradigms, especially relevant to robotic manipulation, is \textit{Meta-Learning}. The goal of Meta-Learning is to train a model that can ``learn to learn," meaning it develops a set of general weights $\theta$ that, while not directly usable for solving any specific task within a distribution of tasks $\mathcal{T}$, can be quickly adapted through a few backpropagation steps to solve a given task within that distribution, $\mathcal{T}_{i} \in \mathcal{T}$. One of the most popular Meta-Learning algorithms is the \textit{Model-Agnostic Meta-Learning} (MAML) algorithm \cite{finn2017maml}, described in Algorithm \ref{alg:maml}. The MAML algorithm follows an iterative learning procedure consisting of two steps:
\begin{itemize}
    \item \textbf{Meta-Learning}: During this phase, task-specific weights $\theta_{i}$ are computed for each sampled task $\mathcal{T}_{i}$. Specifically, the \textit{meta-parameters} $\theta$ are updated according to the gradient obtained from evaluating the loss function on the $i^{th}$ task $\mathcal{T}_{i}$, where the function $f$ is parameterized by the meta-parameters $\theta$.

    \item \textbf{Meta-Adaptation}: In this phase, the meta-parameters are further refined. The loss function $f$, now parameterized by the task-specific parameters for the $i^{th}$ task, is used to adjust the meta-parameters based on the gradients derived from the sum of the loss functions evaluated on the task-specific weights. This process provides feedback to the meta-parameters $\theta$ from each task, leading to a generalized point that can be easily adapted to new tasks (Figure \ref{fig:maml_weights}).
\end{itemize}
\input{algorithm/maml.tex}
\input{figures/ch1/maml_weigths.tex}
The MAML algorithm is the base for different methods which apply Few-Shot Imitation Learning in the context of Behavioral Cloning \cite{finn2017one_shot_visual_il,yu2018daml,yu2018one_shot_hil}.

In \cite{finn2017one_shot_visual_il}, MAML algorithm was used to prove the effectiveness of Meta-Learning in the context of real robot manipulation, with visual observations, as opposite to \cite{duan2017one_shot_il}. A Convolutional Neural Network was trained by following the Algorithm \ref{alg:maml}, using as loss-function the Mean Squared Error, computed between the predicted action and the ground truth one. For real-robot experiments a dataset of \textbf{1300} placing demonstrations (i.e., place a held object in a target container), containing near to \textbf{100} different objects, was collected through teleportation. The trained system was tested by performing the adaptation step on one video demonstration, over 29 new objects, moreover, between the video demonstration and the actual execution, the objects configuration was changed. In this setting the system reached the $\mathbf{90\%}$ of success rate, outperforming baseline methods based on LSTM \cite{duan2017one_shot_il}, and contextual network (i.e., a Convolutional Neural Network that takes in input the current observation and the image representing the target state).

In \cite{yu2018daml}, the \textit{Domain Adaptive Meta-Learning} (DAML) algorithm was introduced to infer a policy from a single human demonstration. 
\newline It employs a two-step process. First, the \textbf{Meta-Learning step}, where for each task $\mathcal{T}$, human demonstrations $\mathcal{D}^{h}_{\mathcal{T}}$ and robot demonstrations $\mathcal{D}^{r}_{\mathcal{T}}$ (Figure \ref{fig:daml_tasks}) are used to learn the \textit{initial policy parameters} $\theta$ and the \textit{adaptive loss} parameters $\psi$, solving the problem in Formula \ref{eq:daml}.
\input{formula/daml.tex}
\newline The outer loss is the classic supervised Behavioral Cloning loss, defined as $\mathcal{L}_{BC}(\phi, \mathbf{d^{r}}) = \sum_{t} \log(\pi_{\phi}(a_{t} \mid s_{t}, o_{t}))$. The inner loss, $\mathcal{L}_{\psi}$, is a learned \textbf{adaptive loss}. Specifically, $\mathcal{L}_{\psi}$ is used during Meta-Adaptation, where the policy parameters are updated by evaluating the gradients derived from $\mathcal{L}_{\psi}$. This process involves a video of a human demonstrating a new task $\mathcal{T}$ as input, leading to the policy update defined by $\phi_{\mathcal{T}} = \theta - \alpha \nabla_{\theta} \mathcal{L}_{\psi}(\theta, \mathbf{d}^{h})$. 
\newline This adaptive loss is the key component proposed in DAML. To use it effectively, it is necessary to learn the parameters $\psi$, observing how there is no direct correspondence between the human video demonstration and the robot ground truth actions. To address this challenge, the authors of DAML observed that while the policy learns to produce appropriate actions through the $\mathcal{L}_{BC}$ loss, the adaptive loss should instead adjust the perceptual aspect of the policy, focusing on human motion and the manipulated object. Based on this insight, the authors implemented the function $\mathcal{L}_{\psi}$ using a 1D Temporal Convolutional Network (Figure \ref{fig:daml_temporal_adaptation_loss}). The convolutional layers are applied to a stack of embeddings generated by the policy $\pi$ across different frames of the video demonstrations. The parameters of this module are learned during the meta-training phase, following the weight update process described in Formula \ref{eq:daml_temporal_adaptation_loss}. The objective of $\mathcal{L}_{\psi}$ is to generate task-specific policy parameters $\phi_{\mathcal{T}}$ that guide the policy to produce effective actions.

\input{formula/daml_temporal_adaptation_loss.tex}
\newline Experimental evaluation on tasks such as placing, pushing, and pick-and-place, has shown that: \begin{itemize}
    \item The system was able to generalize across both new objects and objects configuration starting from only a single human demonstration.
    \item A performance degradation was observed in large domain-shift experiments, such as novel backgrounds and different camera view-points.
\end{itemize}
\input{figures/ch1/daml.tex}

Meta-Learning algorithms have demonstrated interesting properties, notably their capacity for few-shot generalization to novel objects and object configurations. However, it has been observed that during the adaptation step, these methods tend to lose their effectiveness in performing other tasks. This limitation has underscored the need for the development of Multi-Task Imitation Learning methods, which aim to address these shortcomings and enable more versatile task execution in complex scenarios. These kinds of methods will be discussed in the following paragraphs.

\textbf{Zero-Shot MTIL} refers to approaches that aim to train a model capable of solving different tasks without any further adaptation or backpropagation steps. This approach addresses a key issue in Meta-Learning methods, which is the problem of forgetting how to solve previous tasks after adapting to a new one. The goal is to develop a single policy that can handle multiple tasks in a zero-shot manner.

In this context, a crucial design choice is how to convey the desired task to the policy. The literature identifies two main methods to address this problem:

\begin{enumerate}
    \item \textit{Language Conditioned}: These methods leverage natural language descriptions of tasks to inform the model about the task to be executed.
    \item \textit{Visual Conditioned}: These methods use visual information (e.g., goal-state images, video demonstrations) to provide the model with the task instructions.
\end{enumerate}

\textit{Language Conditioned}. As said a possible and intuitive way to inform the policy about which task to execute is through natural language description. Indeed, by looking to the phrase ``Pick the blue cube and place it in the red bow'', there are both information about the action to perform (pick and place) and which are the objects involved (blue cube and red bow). Consequently, by training a neural network on a diverse set of tasks, the system is expected to generalize its understanding to unseen objects within familiar tasks and entirely novel tasks composed of fundamental actions learned during training. This approach highlights the potential for \textbf{robust} and \textbf{adaptable} \textbf{human-robot interaction} in real-world scenarios.

Foundational research, such as that by \cite{stepputtis2020language} and \cite{jang2022bc_z}, has sought to explore the previously mentioned hypothesis. Notably, \cite{stepputtis2020language} introduced an innovative architectural framework, depicted in Figure \ref{fig:language_conditioned}, marking the first instance of seamlessly integrating language, vision, and control tasks. This model is composed of two critical components: a \textit{Semantic Model}, which generates a task embedding denoted as $e$ by processing the initial scene image and the accompanying text command, and a \textit{Control Model}, which generates the control signal using the current robot state $r_{t}$ and the task embedding $e$.

A crucial aspect of such architectures is the management of the visual state, represented by the image $I$, and the command $v$, to create a meaningful embedding $e$ that encapsulates both the current scene state and the desired command. Specifically, the image is first processed using a pre-trained object detection network (Faster R-CNN \cite{fastrcnn}) to identify salient objects in the robot's environment. The detected objects are represented by feature vectors, which include class and bounding box information. Concurrently, the language command is embedded into a suitable representation using a language embedding technique (e.g., GloVe \cite{pennington2014glove}), with the command vector $V$ encoded by a recurrent GRU unit. To associate the objects identified with the sentence embedding $s$, a likelihood value is computed for each object proposal $a_{i} = w_{a}^{T} f_{a}([\text{f}_{i}, s])$, and a probability distribution is computed over the candidates $\mathbf{a} = \text{softmax}([a_0, \dots, a_c])$. Finally, the task embedding $e$ is formed by a fully connected layer that takes as input the sentence embedding $s$ and the weighted sum of object candidate features $e'= \sum_{i=0}^{c} f_{i}a_{i}$.

Training of this model was conducted on two fundamental tasks, namely ``Picking" and ``Pouring", within scenarios featuring multiple objects of the same category, which served as distractors (see Figure \ref{fig:objects}). The subsequent testing experiments demonstrated the system's capability to successfully complete the picking task 98 out of 100 times and the pouring task 85 out of 100 times within novel scenarios. These results serve as compelling evidence of the efficacy and potential of language-conditioned methodologies in the field.
\input{figures/ch1/language_conditioned.tex}

In \cite{jang2022bc_z}, the authors take a step toward developing a more general agent by proposing a large-scale dataset comprising \textbf{100} diverse manipulation tasks. The demonstrations were collected through both expert teleoperation and a shared autonomy process (HG-DAgger \cite{kelly2019hg_dagger}). The skills covered included pick-and-place, grasp, pick-and-drag, pick-and-wipe, and push skills. The dataset was used to train the network shown in Figure \ref{fig:bcz_architecture}. Notably, the samples consisted of the current robot observation, with conditioning provided by either a \textit{\textbf{natural language description}} or a \textit{\textbf{human video demonstration}}.

The goal was to train a conditioned policy using the current observation $o_{t}$ and a task representation $c$, enabling the policy to generalize to new tasks in a zero-shot manner (i.e., without fine-tuning). In contrast to previous methods, this approach does not rely on pre-trained object detectors for identifying candidate regions. Instead, the Task Embedding is directly injected into the Feature Maps generated by the Convolutional Neural Network (ResNet-18), facilitated through the FiLM layer \cite{perez2018film}.

Experimental results shown that, over 28 held-out tasks, containing both completely new objects, and known objects but in different tasks, an average success rate of \textbf{38\%} was reached in the easiest setting, with only one distractor and with natural language instruction. The success rate dropped to \textbf{4\%} in the hardest setting with 4 distractors and video conditioning.
\input{figures/ch1/bc-z.tex}

Foundational research in the field of Language-Conditioned Multi-Task Imitation Learning has demonstrated promising results in zero-shot generalization. However, the robustness of their performance remains a challenge. Subsequent research, as highlighted in \cite{brohan2022rt,mees2022calvin,mees2022hulc}, has focused on enhancing performance. 

In particular, the authors of \cite{brohan2022rt} sought to investigate whether the transfer of knowledge from extensive, diverse, and task-agnostic datasets, which has enabled modern machine learning models to excel in zero or few-shot learning scenarios for new and specific tasks, is applicable within the realm of robotics. This inquiry arises due to the presence of high-capacity architectures capable of assimilating knowledge from such large datasets. To explore this prospect, the authors in \cite{brohan2022rt} introduced a comprehensive dataset comprising over 130,000 demonstrations collected across more than 700 household tasks (Figure \ref{fig:rt_1_dataset}). Additionally, they proposed a Language-Conditioned Transformer-based architecture (Figure \ref{fig:rt_1_model}). Here the authors made relevant modification in the architecture, with respect to the previous BC-Z architecture \cite{jang2022bc_z}. Indeed, they modified the Visual Module by using an EfficientNet \cite{tan2019efficientnet} instead of the ResNet-18. The instruction was encoded using the Universal Sentence Encoder \cite{cer2018universal}, and the policy was implemented through a Transformer. Additionally, the authors addressed the real-time constraints of robot control. To accelerate inference time and achieve a frequency of $3$ Hz, the authors employed a TokenLearner module \cite{ryoo2021tokenlearner}, which utilizes an attention mechanism to select the most relevant tokens, thereby reducing the number of tokens that the underlying control module must process to infer the action.
\input{figures/ch1/rt1_model.tex}
\input{tables/ch1/rt_1_dataset.tex}

It is worth noting the intriguing findings presented in Table \ref{table:rt1_results}. The model demonstrates robustness in performing tasks it has encountered previously and even achieves reasonable performance on tasks it has never seen before. However, a noticeable decline in performance occurs when the model faces novel backgrounds or scenarios with distracting objects, particularly when data for these situations is sparse. This trend is significant because, unlike domains such as Computer Vision and Natural Language Processing, where large-scale datasets can be easily obtained, collecting real-world robotic datasets is a labor-intensive and time-consuming process. Moreover, these datasets often have limited applicability to other research due to \textbf{disparities in action space, robot morphology, and scene representation}, as highlighted by \cite{brohan2022rt}. Therefore, the goal is to create a system capable of replicating tasks with minimal demonstrations, specifically gathered from the particular robot in use.

Furthermore, the decline in success performance when faced with distractor objects emphasizes that addressing the policy-learning problem in an end-to-end manner, which involves mapping high-dimensional and high-level inputs like images and text directly to low-dimensional, low-level outputs such as the actions to be executed, may not be the most effective approach. This is because such models might lack the necessary perceptual components that enable them to initially recognize the target object within the scene, subsequently navigate towards it, and commence the manipulation process. This process aligns with how humans approach manipulation tasks \cite{grill2003neural}.
\input{tables/ch1/rt_1_results.tex}

As we have seen up to now, all the proposed methods were tested on different robotic platforms with different scenarios and environments. As in other Computer Vision problem, there are no well known benchmark that are used by the researchers around the world. To solve this problem, authors in \cite{mees2022calvin} proposed CALVIN (Composing Actions from Language and Vision), which is an open-source simulated benchmark designed for learning long-horizon language-conditioned tasks in robotic manipulation. Specifically, CALVIN proposes a set of 34 manipulation tasks in 4 different environments (Figure \ref{fig:calvin_env}). 
\input{figures/ch1/calvin.tex}

This benchmark was used by \cite{mees2022hulc,mees2023hulc++,reuss2024multimodal}. Specifically, the work proposed in \cite{reuss2024multimodal} is actually the best performing method on the CALVIN benchmark. The authors proposed an architecture that is able to handle goals described in terms of both natural language description and goal image, moreover they used a Diffusion Transformer Model as policy (Figure \ref{fig:mdt_architecture}). To reach this goal, the authors had to solve first the problem related to how to force the Multimodal Transformer Encoder to generate the same behavior independently of the goal modality. To solve this problem authors of MDT proposed two auxiliary self-supervised loss functions:
\begin{itemize}
    \item \textit{Masked Generative Foresight} (MGF) is a reconstruction loss function designed to ensure that the MDT generates embeddings that guide robot behavior consistently across different modalities. This means that the tokens generated by the MDT can be used to construct image patches representing feature states, whether the goal is described in terms of an image or a language description. Specifically, given the latent embedding of the MDT encoder for state $s_i$ and goal $g$, MGF trains a Vision Transformer (ViT) to reconstruct a sequence of 2D image patches $(u_1, \dots, u_U) = \text{patch}(s_{i+v})$ corresponding to the future state $s_{i+v}$.
    \item \textit{Contrastive Latent Alignment} (CLA) is a contrastive loss term designed to encourage the MDT to align the embeddings generated from a goal image with those generated from a language description. For each training sample $(s_i, a_i)$ paired with a multimodal goal specification $G_{s_i,a_i} = \{o_i, l_i\}$, CLA reduces the image and language goals to vectors $z_i^o$ and $z_i^l$, respectively. The CLA loss is then computed using the InfoNCE loss, based on the cosine similarity $C(z_i^o, z_i^l)$ between the image-goal conditioned state embedding $z_i^o$ and the language-goal conditioned state embedding $z_i^l$.
\end{itemize}
\input{figures/ch1/mdt_architecture.tex}
As mentioned, this method was tested on the CALVIN benchmark, specifically in the $ABCD \rightarrow D$ test scenario, where the model was trained on the $ABCD$ environments and tested on environment $D$. The CALVIN benchmark comprises a set of 1000 rollouts, with each rollout consisting of a sequence of 5 commands. It is noteworthy that the MDT architecture successfully completed $80\%$ of the rollouts up to the fifth command. In particular, the MGF loss was observed to have a significant impact on system performance, improving the success rate for the fifth command from $69.8\%$ to $79.4\%$. This demonstrates that making the embedding informative about the system's evolution can meaningfully guide the diffusion system, which predicts actions over a certain time horizon into the future.

In the context of Language Conditioned MTIL, other important works to cite include \cite{shridhar2022cliport, shridhar2023perceiver}. Specifically, the authors in \cite{shridhar2022cliport} introduced CLIP-Port, a two-stream architecture that explicitly models the two key tasks in language-conditioned imitation learning: reasoning about \textbf{what to do} and reasoning about \textbf{how to do it}. The former task, referred to as \textit{semantic reasoning}, is derived from the text-based command and is handled using a pre-trained CLIP architecture \cite{radford2021learning}. The latter task, known as \textit{spatial reasoning}, is managed by leveraging the Transporter architecture \cite{zeng2021transporter}. The overall architecture (Figure \ref{fig:clip_port_architecture}) is an Encoder-Decoder framework that ultimately outputs an \textbf{affordance map}, which identifies the locations for executing pick or place operations (Figure \ref{fig:clip_port_affordance}).

During testing, this method was not directly compared with other Language Conditioned MTIL approaches. However, the results obtained allow for several observations. Notably, for tabletop manipulation tasks, the ability to reason using both spatial and semantic features enables a high success rate (ranging from $80\%$ to $90\%$) on tasks involving seen object attributes, such as color, even with a relatively low number of demonstrations (100). However, similar to the results reported in Table \ref{table:rt1_results}, CLIP-Port also struggles with entirely novel tasks, as evidenced by a significant drop in performance when dealing with unseen object attributes and a lower number of demonstrations.

\input{figures/ch1/clip_port.tex}

In conclusion, as discussed in this paragraph, considerable research effort has been dedicated to addressing the problem of Language Conditioned MTIL, particularly in terms of architectural designs and learning strategies. However, it remains challenging to draw definitive conclusions from these various approaches, as they are generally not evaluated on a common benchmark. Despite this issue, some trends can be observed. There is clearly room for improvement in the generalization capabilities of these methods, particularly in handling novel scenarios and tasks. Additionally, data efficiency remains a concern, as there is a noticeable decline in performance as the number of expert trajectories decreases. Another challenge is the ability to manage cluttered scenes with \textbf{relevant distractors}, objects that may have been manipulated during training but must be ignored in the target task, which can lead to occlusion and/or confusion.

\textit{Visual Conditioned} refers to methods that use visual information as a conditional signal. This information can be represented either as an image depicting the desired goal state or as a sequence of frames illustrating how the task should be performed. The idea behind this kind of methods is to build systems that are able to replicate tasks by observing the execution performed by other agents, like human can learn task by mimicking the execution of other humans.

Preliminary works, such as those by \cite{james2018task_embedded} and \cite{bhutani2022attentive_one_shot}, introduced architectures conditioned on the first and last frames of task demonstrations. Specifically, in \cite{james2018task_embedded}, the authors proposed TecNets (Task-Embedded Control Networks), an architecture illustrated in Figure \ref{fig:task_embedded}. TecNets consist of two main components:

\begin{itemize}
    \item \textit{Task-Embedding Network}: The purpose of this network is to create an embedding space where demonstrations of the same task are closely clustered, while embeddings of different tasks are kept as distant as possible. This network is trained using a contrastive loss function, defined in Formula \ref{eq:task_loss}, which ensures that the embedding of the $k^{th}$ sample of the $j^{th}$ task is similar to the "sentence" representing that task (i.e., the average of the embeddings of the $j^{th}$ task) and maximizes the distance from the sentences of other tasks.
    \begin{equation}
        \label{eq:task_loss}
        \mathcal{L}_{emb} = \sum_{\tau^{j}_{k} \in \mathcal{T}^{j}}\sum_{ \mathcal{T}^{i} \notin \mathcal{T}^{j}} \max \left[ 0, \text{margin} - s^{j}_{k}\cdot s^j + s^{j}_{k}\cdot s^i \right]
    \end{equation}
    \item \textit{Control Network}: This module implements the policy $\pi$, which takes as input the current observation $o$ and the sentence $s$ of the desired task. It then generates the corresponding action for the robot, conditioned on the task.
\end{itemize}

\input{figures/ch1/task_embedded.tex}

The results from this preliminary work demonstrated the feasibility of training a vision-conditioned system capable of solving tasks in a zero-shot manner. The system achieved a success rate of \textbf{86.31}\% on a simulated 2D reaching task and \textbf{77.25}\% on a simulated pushing task. However, it is important to note a significant limitation of these approaches: the conditioning signal often fails to adequately capture critical information about the optimal strategy for solving the task. This shortcoming arises because the conditioning signal does not fully encode the strategies needed for effective task execution.


The methodologies introduced in \cite{dasari2021transformers_one_shot} and \cite{mandi2022towards_more_generalizable_one_shot} aim to advance the concept of an ideal multi-task agent, capable of replicating new tasks based on a single demonstration, often performed by another agent (e.g., a robot or a human).

In \cite{dasari2021transformers_one_shot}, the authors focused on training a vision-based control architecture (depicted in Figure \ref{fig:tosil_architecture}) that can replicate a task demonstrated through a set of frames sampled from a video of another agent performing the task. The architecture relies on two main components:
\begin{itemize}
    \item \textit{ResNet-18} \cite{resnet}, which serves as the backbone for extracting visual embeddings from the current agent state and the demonstration frames.
    \item \textit{Transformer} \cite{vaswani2017attention}, which uses the attention mechanism to combine embeddings from the demonstration frames with the current visual observation.
\end{itemize}

This architecture was trained using multiple loss terms, specifically:
\begin{itemize}
    \item \textit{Behavioral Cloning} term $\mathcal{L}_{BC}$ (Formula \ref{eq:tosil_bc}), a supervised loss function used to learn the action to perform given the agent's observation and the demonstrated task. Specifically, the loss function is a negative log-likelihood, computed over a mixture of learned logistic distributions, where the $i^th$ component is parameterized according to the mean and variance $(\mu_{i}, \sigma_{i})$ which are estimated \\ through a MLPs.
    \begin{equation}
        \label{eq:tosil_bc}
        \mathcal{L}_{BC} = - ln(\sum_{i=0}^{k}\alpha_{i}(\phi_{t}) P(a_{t}, \mu_{i}(\phi_{t}), \sigma_{i}(\phi_{t})))
    \end{equation}
    \item \textit{Inverse Model Regularizer} term $\mathcal{L}_{inv}$ (Formula \ref{eq:tosil_inv}), a regularization term where the output is the action $a_{t}$, given two consecutive representations $\tilde{\phi}_{t}$ and $\tilde{\phi}_{t+1}$, effectively solving the inverse control problem.
    \begin{equation}
        \label{eq:tosil_inv}
        \begin{aligned}
            \mathcal{L}_{inv} &= - \ln \Bigg( \sum_{i=0}^{k} \alpha_{i}(\tilde{\phi}_{t}, \tilde{\phi}_{t+1}) \\
            &\quad \times \operatorname{logistic}(\mu_{i}(\tilde{\phi}_{t}, \tilde{\phi}_{t+1}), \sigma_{i}(\tilde{\phi}_{t}, \tilde{\phi}_{t+1})) \Bigg)
        \end{aligned}
    \end{equation}
    \item \textit{Point Prediction Auxiliary} term $\mathcal{L}_{point}$, a simple regression loss that, given the current representation $\phi_{t}$, predicts the position of the gripper in the next $t+H$ steps. This loss is used to improve the explainability of the actions performed by the robot.
\end{itemize}

The authors specifically trained and tested the architecture on the \textit{pick-place} task, which consists of a total of 16 variations (Figure \ref{fig:tosil_task}). In the proposed evaluation setting, the TOSIL architecture achieved a success rate of $\textbf{88.8}\% \pm \textbf{5.0}\%$, demonstrating the system's ability to accurately capture the requested variations represented in the demonstration frames, even when faced with differences in robot embodiment.
\input{figures/ch1/tosil.tex}

In \cite{mandi2022towards_more_generalizable_one_shot}, the authors presented MOSAIC (Multi-task One Shot imitation with self-AttentIon and Contrastive learning), which is an improvement of the work proposed in \cite{dasari2021transformers_one_shot} from both the architectural point-of-view and from the evaluation setting.
Indeed, from the architectural point of view, MOSAIC is designed to model a demonstration-conditioned policy, denoted as $\pi^{L}_{\theta}(a_{t}|s_{t},c)$, with $c$ representing the current task demonstration, such as a video depicting a robot performing the task. Specifically, the architecture is an optimization of TOSIL since, where the encoder-decoder Transformer architecture has been substituted with an encoder-only architecture leveraging the self-attention mechanism to correlate the embeddings coming from the current agent observation and the one coming from the demonstration frames (Figure \ref{fig:mosaic_architecture}). This adjustment allows to have a lower number of parameters and consequently improve the inference time. 
\input{figures/ch1/mosaic.tex}

To handle this complex scenario involving both multiple variations and multiple tasks, the authors proposed training the system using a combination of loss functions. Specifically, in addition to the $\mathcal{L}_{BC}$ loss function defined in Formula \ref{eq:tosil_bc} and the $\mathcal{L}_{inv}$ loss function defined in Formula \ref{eq:tosil_inv}, the authors introduced a contrastive loss term implemented through the InfoNCE objective (Formula \ref{eq:mosaic_contrastive}). The goal of this loss is to enable the system to create similar representations for time-adjacent frames within a given task while maximizing the distance from representations corresponding to other tasks. In this context, $q$ is the anchor embedding obtained from a randomly sampled frame in a given batch $B$, $k_{+}$ is the positive example, which is a nearby frame from a different view of the batch (i.e., the original batch with a different data augmentation applied), and $k_{i}$ represents the negative samples, obtained from any other frame in the batch.

\begin{equation}
    \label{eq:mosaic_contrastive}
    \mathcal{L}_{Rep} = \log\left(\frac{\exp(q^TWk_{+})}{\exp(q^TWk_{+})+ \sum_{i=1}^{F-1}\exp(q^TWk_{i})}\right)
\end{equation}


To test the model, a dataset containing \textbf{seven distinct tasks} and \textbf{61 variations} (Figure \ref{fig:mosaic_task}) was generated by executing hand-written policies in a simulation environment. This dataset was employed to train and compare the proposed MOSAIC architecture against other one-shot (meta-)imitation \\ learning methods, such as \cite{yu2018daml} and \cite{dasari2021transformers_one_shot}. The results, presented in Table \ref{table:mosaic}, demonstrate that MOSAIC surpasses previous methods in the Single-Task Zero-Shot Imitation Learning setting, particularly when addressing individual tasks with multiple variations and testing on previously unseen scenarios (i.e., novel object configurations).

Furthermore, when evaluated in a Multi-Task setting, where the system is trained on all tasks and subsequently tested on each task separately, MOSAIC exhibits the capability to partially replicate the demonstrated tasks. It is important to note that transitioning from a single-task to a multi-task context introduces inherent challenges. Despite being familiar with the tasks used during training, the success rate tends to decrease for almost all tasks. This phenomenon underscores the necessity for training procedures and architectures capable of generating embeddings that accurately represent both the task itself and its various sub-tasks. Such capacity is essential for reusing these embeddings when executing new instances or entirely novel tasks.

\input{tables/ch1/mosaic_res.tex}

In this context, further improvements were introduced by the authors of \cite{chang2023one,cui2023from}. Specifically, the authors in \cite{chang2023one} analyze the shortcomings of existing methods like TOSIL and MOSAIC, which often struggle due to issues such as the DAgger problem (distribution shift from offline training), last centimeter errors in fine motor control (e.g., collisions when the robot reaches the target object to pick), and misalignment with task contexts rather than the actual tasks (e.g., the system links the trajectories to the objects present in the scene rather than focusing on what is demonstrated by the expert).

To overcome these issues, the authors proposed a modular approach that separates task inference, i.e., understanding the intent of the demonstrations, from task execution, i.e., generating the actions to perform during the rollout. To achieve this, they introduced AWDA (Attributed Waypoints and Demonstration Augmentation), a modular framework for visual \\ demonstration represented in Figure \ref{fig:awda_framework}. AWDA is based on two main concepts:

\begin{itemize}
    \item \textit{Attributed Waypoints}, designed to mitigate task execution errors, primarily those related to the distributional shift problem. These are classic waypoints (i.e., sequences of 6D poses of the end-effector) augmented with attributes, such as whether there is an object in the gripper. The robot moves between these generated waypoints using hand-defined motion primitives, which helps eliminate small errors during rollout execution that could otherwise cause the system to deviate from the correct trajectory.
    
    \item \textit{Demonstration Augmentation}, introduced to decouple \\ tasks from task contexts. Specifically, two types of augmentation are introduced: \textbf{Asymmetric Demonstration Mixup}, which generates novel samples by mixing samples from two distinct trajectories according to Formula \ref{eq:awd_blanding}, where $v$ and $\tilde{v}$ are video demonstrations of two distinct tasks, and $o$ and $\tilde{o}$ are agent observations in two distinct contexts. \textbf{Additional Demonstrations via Trajectory Synthesis} involves generating free-space motions for the robot in various contexts by sampling a few points (1 to 3) uniformly at random within the agent's workspace and moving the end effector sequentially through these points using an inverse kinematics solver. Training samples are created by pairing each trajectory with itself, requiring the model to focus on the motion of the arm and ignore background elements to make correct predictions.
    
    \begin{equation}
        \label{eq:awd_blanding}
        v'_t = \alpha v_t + (1-\alpha) \tilde{v}_0 \quad o'_t = \alpha o_t + (1-\alpha) \tilde{o}_0
    \end{equation}
    
\end{itemize}
\input{figures/ch1/awd.tex}

Using this framework, the authors trained the same Transformer based architecture as in \cite{dasari2021transformers_one_shot}, achieving notable results. Specifically, on the pick-place task represented in Figure \ref{fig:tosil_task}, AWDA reached a success rate of \textbf{100}\% on two held-out variations, i.e., the system was trained on 14 variations and tested on the remaining 2. However, when tested on completely novel tasks, i.e., tasks never seen during training, the system struggled to succeed, as reported in Table \ref{table:awda_results}.
\input{tables/ch1/awda_res.tex}

In conclusion, significant research efforts have been dedicated to addressing the challenge of Visual Conditioned Multi-Task Imitation Learning (VC-MTIL). The aim is to develop systems capable of solving a given task in a zero-shot manner, starting from just a single video demonstration. Specifically, while these systems show promising results in solving new instances of seen tasks and unseen variations of known tasks, they struggle to handle a  multi-task multi-variation setting, exhibiting a consistent drop in performance. Additionally, they face difficulties in solving entirely new tasks.

Furthermore, the work discussed in this section predominantly involves experiments conducted in simulation environments, leaving unanswered whether these systems can be effectively applied in real-world settings, where demonstrations might also come in the form of human video demonstrations.

% \subsection{Object-Oriented Imitation Learning}
\label{sec:ooil}
All the methods discussed in Section \ref{sec:lfd} share a common characteristic: they are end-to-end systems that take high-level inputs, such as images, and directly generate the corresponding actions as output. While this approach can be sufficient in scenarios where the scene is simple, meaning there are no distracting objects, or if there are distracting objects, they can be easily identified by the system because they are consistently not involved in the manipulation, this end-to-end approach may struggle in more complex environments. Specifically, it can encounter difficulties when the robot workspace contains objects that are similar to each other, especially if these objects are involved in manipulation for some task variations.

Based on this consideration, this paragraph will describe methods that leverage object priors. This means that the control policy is informed not only by the embedding of the agent scene, which is obtained from a deep architecture, but also by object-level information, such as the bounding boxes of objects in the scene, obtained from an object detector.

The concept of leveraging object priors has been explored in both earlier works \cite{devin2018deep,park2021object} and more recent approaches \cite{belkhale2023plato,zhu2023viola,zhu2023learning,jiang2023vima}.

One of the preliminary works on leveraging object priors was presented by the authors of \cite{devin2018deep}. This work primarily focuses on the challenge of generalization in Learning from Demonstration (LfD) systems that use an end-to-end approach. In such systems, a task-specific model is trained to predict actions based on raw visual observations. The authors found that while it is possible to achieve good \textit{instance-level generalization}, meaning the model can solve tasks with varying initial configurations using a limited number of samples, achieving \textit{category-level generalization} is more challenging. Category-level generalization refers to the model ability to solve tasks involving different objects. To achieve this, the dataset must include a large number of trajectories involving a wide variety of objects. For instance, if the task is to pour the contents of a bottle into a cup, the dataset should contain trajectories with different types of cups and bottles. However, constructing such a dataset is time-consuming and costly. Moreover, the potential of well-known large datasets in classical computer vision tasks, such as object detection, is not fully utilized.

To address this issue, the authors proposed a paradigm shift by introducing a robotic vision framework that operates on sets of objects rather than raw pixel data. This framework leverages prior datasets to learn a generic object concept model, thereby enhancing category-level generalization without requiring an extensive and diverse dataset. The framework is illustrated in Figure \ref{fig:object_prior_framework} and is composed of several stages:
\begin{itemize}
    \item \textbf{Meta-Attention}, which is basically a Region Proposal Network (RPN) \cite{fastrcnn}, trained on the well known MSCOCO \cite{lin2014microsoft} dataset. The RPN generates objects proposals, i.e., region of the image that possibly contain an object.
    \item \textbf{Task-Specific Attention}, which aims to learn what are the object of interest with respect to the task in hand. This module is parameterized as a vector $w$ such that the attention paid to $o^i$ is proportional to $e^{w^Tf(o^i)}$.
    \item \textbf{Soft Attention}, this module gives a probabilistic meaning to the attention map obtained from the Task-Specific Attention. Specifically, a Boltzmann distribution is used to map the attention weights to a probability for each object proposal, i.e., $p\left(o^i \mid w_j\right)=\frac{e^{w_j^{\top} \frac{f\left(o^i\right)}{\left\|f\left(o^i\right)\right\|_2}}}{\sum_{i=0}^N e^{w_j^{\top} \frac{f\left(o^i\right)}{\left\|f\left(o^i\right)\right\|_2}}}$.
    \item \textbf{Movement Prediction Network}, this module predicts the next robot action, given the attended object information from the soft attention, and the robot state represented by the joint and end-effector state.
\end{itemize}
\input{figures/ch1/object_prior_framework.tex}
This preliminary work focused mainly on two tasks:
\begin{itemize}
    \item  \textit{Pouring Task}: The robot is required to pour contents from a bottle into a mug. The challenge is to locate the mug from an image without being explicitly provided its location, especially when different mugs are used during training and testing.
    \item \textit{Sweeping Task}: The robot must sweep an object (e.g., a plastic orange) into a dustpan, with both objects starting in different positions. This task requires the robot to adapt its approach based on the relative positions of the objects.
\end{itemize}
During testing, the authors focused on \textit{Category Generalization} and the ability to \textit{Ignore Distractor Objects}. For the former, the system was trained with only one type of mug and evaluated with other mugs (Figure \ref{fig:pouring_task_setting}). The results showed that the system successfully poured the contents into the correct mug, thanks to the learned task-specific attention weight that highlighted the mug features. For the latter, the authors designed a test where two mugs were present in the scene (Figure \ref{fig:task_specific_attention}). Since the model did not receive any conditioning signal indicating which mug to use, the authors fine-tuned the attention weight on trajectories where only the brown mug was used, demonstrating that this mechanism could focus on more specific features, such as the mug color.
\input{figures/ch1/object_prior_framework_task.tex}

In summary, this preliminary work demonstrated that leveraging object priors can facilitate category-level generalization by utilizing large, well-known datasets for the object-detection problem. However, the experimental setup was relatively simple, even in scenarios with distractor objects. The proposed system could handle distractor objects only after specific fine-tuning and was not able to dynamically discriminate between objects of interest and distractors based on task variations.

A recent work that follows a similar approach is proposed in \cite{zhu2023viola}. In this work, the authors introduced VIOLA (Visuomotor Imitation via Object-centric Learning) (Figure \ref{fig:viola_architecture}), an architecture inspired by the ideas presented in \cite{devin2018deep}. VIOLA uses an RPN and a ResNet18 \cite{resnet} to generate object proposals and produce a spatial feature map, respectively. It then constructs a \textit{per-step feature} vector, composed of three key elements: a \textbf{global context feature} that encodes the current task stage, an \textbf{eye-in-hand visual feature} to mitigate occlusion, and a \textbf{proprioceptive feature} that captures the robot state. These per-step features are concatenated to form a \textbf{history of observations}, which is designed to capture temporal dependencies and dynamic changes in object states. This tensor is then fed into a Transformer \cite{vaswani2017attention}, which leverages its intrinsic attention mechanism to automatically focus on the object of interest.
\input{figures/ch1/viola_architecture.tex}
This method was first evaluated in a simulation environment across three tasks, as depicted in Figure \ref{fig:viola_task}. Various testing scenarios were considered, including different object placements, the introduction of distractor objects, and changes in camera position. Generally speaking, VIOLA outperformed all baselines across these testing conditions, further demonstrating the utility of object priors in enhancing the robustness of such methods. However, similar to \cite{devin2018deep}, the testing scenarios were relatively simple, with clear distinctions between distractors and objects of interest. The distractors were objects never seen during the demonstration and were not involved in manipulation, making them relatively easy for the model to discriminate.

In the works discussed so far, the approach has primarily focused on leveraging object priors to directly predict the actions that the robot must perform. However, a different approach was proposed in \cite{belkhale2023plato}, where the authors introduced an alternative interpretation of object-centric concept. Instead of focusing on the robot perspective, they shifted the emphasis to the object perspective, proposing PLATO (Predicting Latent Affordances Through Object-Centric Play). PLATO is a learning framework that learns a \textbf{latent affordance space}, which describes how an object can be used (e.g., a block being grasped, a door knob being turned, or a drawer being opened).

The authors argue that learning these affordances (i.e., what happens to the object) rather than plans (i.e., what happens to the robot) from play leads to a simpler and more robust task representation that can operate over varying time horizons. This, in turn, results in more effective policies at test time. This paradigm shift allows the policy to reason about the environment more effectively: given access to an affordance (e.g., the door knob being turned) and a goal (e.g., an opened door), the policy can work backwards to infer the behavior needed to exploit that affordance (e.g., reaching the knob and rotating the gripper to turn it).

To reach this objective authors started from the observation that a single-object manipulation is composed of the following three phases:
\begin{enumerate}
    \item \textbf{Pre-interaction}, when the robot prepares to interact with an object (e.g., reaching for a block).
    \item \textbf{Interaction}, when the robot and the object engage in joint actions (e.g., pushing or pulling the block).
    \item \textbf{Post-interaction}, when the robot separates from the object, and the object may come to rest (e.g., the block stops moving).
\end{enumerate}
Given these three phases, the algorithm learns a \textbf{latent affordance distribution}. Specifically, the architecture comprises three learnable modules:  $E$,  $E'$, and  $\pi$.  $E$ models the posterior distribution, mapping the full interaction trajectory  $\tau^{i}$ to the corresponding latent affordance distribution, from which the affordance embedding  $z$ is sampled.  $E'$ is the prior module used during rollout. It takes the current state and the goal state as input and generates the affordance embedding  $z'$. This module is trained to match the posterior distribution modeled by  $E$.  $\pi$ represents the current policy, which generates the action  $a^{i}$ given the current state  $s^{i}$, the desired goal  $o_g$, and the latent embedding  $z$.

These three modules are trained end-to-end by minimizing the loss function in Formula \ref{eq:plato_equation}, which includes three terms. The first two terms correspond to the policy  $\pi$, ensuring it matches the ground-truth trajectories in the interaction and pre-interaction phases. The last term is the KL-divergence, used to train the posterior and prior modules  $E$ and  $E'$.

\begin{equation}
    \label{eq:plato_equation}
    \begin{aligned}
        \mathcal{L}_{PLATO} = -\log \left(\pi\left(a_{1: H}^{(i)} \mid s_{1: H}^{(i)}, o_g, z\right)\right)- \\ 
        \alpha \log \left(\pi\left(a_{1: H}^{(-)} \mid s_{1: H}^{(-)}, o_g, z\right)\right)+ \\ 
        \beta \operatorname{KL}\left(p(z) \| p\left(z^{\prime}\right)\right)
    \end{aligned}
\end{equation}
\input{figures/ch1/plato.tex}

Finally, this method was tested in a variety of scenarios, including both single-object and multiple-object manipulation with different manipulation primitives (Figure \ref{fig:plato_task}). However, in the multi-object scenarios, the system was only tested on single-object manipulation primitives.

This work is particularly noteworthy as it demonstrates that a policy can be learned by solving an inverse problem, starting from object affordances and deriving the corresponding robot trajectories. It also shows that the policy can be conditioned based on the desired goal state. However, certain aspects were not addressed in this work, such as the potential presence of distractor objects and tasks requiring the manipulation of multiple objects. Additionally, the affordances were learned in the object space (i.e., with known object poses) rather than in the high-level image space.

The works discussed in this paragraph highlight the significant research efforts aimed at modeling the manipulation problem from an object-centric perspective. These efforts either focus on the affordances of the object (i.e., the possible movements the object allows) or introduce object priors defined by regions of interest that may contain the object to be manipulated, with results demonstrated in both simulated and real-world environments. However, the methods presented so far primarily address single-task scenarios, where distractor objects can be easily identified as they remain constant across demonstrations. In contrast, this thesis proposes a solution for a more challenging scenario in which the robot operates in a multi-variation environment. This environment includes multiple similar objects, which may serve as either targets or distractors depending on the specific task variation.

\section{Problem formulation}
\label{sec:ocpl_problem}
As described in Section \ref{sec:occp_related_works}, this thesis primarily addresses the problem of Visual-Conditioned Multi-Task Imitation Learning. The goal is to train a single conditioned control function, $\pi_{\theta}(a_{t}| o^{a}_{t}, c_m)$, that can guide a robotic agent in solving both variations of a given task and entirely different tasks. Where, the input consists of a command $c_m$, represented as a video demonstration of the requested task, along with the current observation of the agent $o^{a}_{t}$.

The approach proposed in this thesis is based on the observation that solving this problem involves two key tasks:
\begin{itemize}
    \item \textit{Command analysis}: This task involves solving a cognitive problem, where the system must interpret the high-level task command, understand the task intent, identify the relevant objects, and recognize the required actions.
    \item \textit{Action generation}: This task involves solving a control problem, where the system must correlate the information from the command analysis with the agent environmental state to generate a valid action that moves the robot toward completing the requested task.
\end{itemize}

As demonstrated in the comprehensive review of related works (Section \ref{sec:occp_related_works}), the Visual-Conditioned MTIL problem is typically addressed using end-to-end architectures, which are trained with an action-centric behavioral cloning loss. While these systems are often able to control the robot and produce \textbf{reasonable trajectories} to complete tasks like pick-and-place, they may manipulate the wrong object, indicating a limitation in the cognitive ability to correctly identify the relevant object.


In this thesis, a modular approach is adopted. Specifically, Figure \ref{fig:end_to_end_vs_modular} illustrates the differences between a general end-to-end architecture and the proposed modular architecture. In the end-to-end architecture, the \textit{Backbone Module}, which can be any deep learning architecture used in state-of-the-art methods, takes both the agent observation $o^{a}_t$ and the command $c_{m_i}$ as input. It generates an embedding $z_t$ that must encapsulate all the necessary information for the \textit{Control Module} to produce a valid action. This includes details derived from both the command, such as the position of the object of interest, the task being solved, its variations, as well as the state of the agent itself.

In contrast, the modular approach utilizes two backbone modules. The first, the \textit{Command Analysis Backbone}, explicitly solves the cognitive task, producing task-relevant information ($z^{task}_t$) such as the position of the object of interest. The second, the \textit{Control Backbone Module}, generates a control embedding ($z^{control}_t$), which directly encodes information relevant to the action to be performed. In this modular approach, the \textit{Control Module} takes both the task-relevant information ($z^{task}_t$) and the control embedding ($z^{control}_t$) as inputs.

\begin{figure}[t]
    \centering
    \includegraphics[width=0.7\textwidth]{figures/images/ch3/end_to_end_vs_modular.jpg}
    \caption{(Upper) General end-to-end architecture, where the \textit{Backbone Module} takes as input both the agent observation and the command. It generates an embedding $z_{t}$ that must contain information related to both the command and the control. (Bottom) In the modular architecture, there are two backbone modules: the \textit{Command Analysis Module}, which generates the task-embedding $z^{task}_t$, and the \textit{Backbone Module}, which is trained to generate only the control-embedding $z^{control}_{t}$.}
    \label{fig:end_to_end_vs_modular}
\end{figure}


The underlying assumption of this approach is that by separating the problem into two components, cognitive and control, and designing task-specific modules trained independently of each other, the system becomes more robust. For example, the Command Analysis Module is trained specifically for the cognitive task (e.g., the Conditioned Object Detection task described in Chapter \ref{ch:cod}), while the Control Backbone is trained for the control task. This separation allows the final Control Module to be informed by optimal embeddings generated by modules trained on task-specific problems.

Section \ref{sec:cod_problem} provides an example of a cognitive problem that can be addressed by the Command Analysis Backbone. Here, the focus is on describing the problem solved in order to learn the final control policy $\pi_{\theta}$. Specifically, the goal is to learn the parameters of the policy $\pi_{\theta}$ using a supervised-learning approach.

The first step is defining the dataset. As explained in Section \ref{sec:bc}, in multi-task, multi-variation scenario, there are $n$ distinct tasks, denoted by $\mathcal{T} = \left\{T_{1}, T_{2}, \dots, T_{n}\right\}$, where each task $T_{i}$ is associated with a set of variations $\mathcal{M}_{i}$. For each task, a specific dataset $\mathcal{D}_{i} = ((c_{m}, \tau_{m}), m \in \mathcal{M}_{i})$ is constructed, containing pairs of demonstrator videos $c_{m}$ and corresponding target trajectories $\tau_{m}$ for each variation. The demonstrator video consists solely of visual observations, represented as $c_{m} = \left\{o^{d}_{1}, o^{d}_{2}, \dots, o^{d}_{T'}\right\}$, while the target trajectory includes both observations and associated actions: $\tau_{m} = \left\{(o^{a}_{1},a_{1}), \dots, (o^{a}_{T}, a_{T})\right\}$.

Building upon the complete dataset $\mathcal{D} = \left\{\mathcal{D}_{1}, \dots, \mathcal{D}_{n}\right\}$, the optimal parameters $\theta^{*}$ are obtained by solving the minimization problem described in Formula \ref{eq:minimization_prob}, where the loss function $\mathcal{L}$ is minimized.
\begin{equation}
    \label{eq:minimization_prob}
    \theta^{*} = \underset{\theta}{\text{arg} \ \min} \ \mathcal{L}(\pi_{\theta}, \mathcal{D})
\end{equation}

Section \ref{sec:ocpl_architecture} will present the specific instance of the proposed architecture, along with the loss function and modules used. The results of the experiments will be discussed in Section \ref{sec:ocpl_experimental}.

\section{Proposed Architecture}
\label{sec:ocpl_architecture}
This section provides the effective implementation of the general modular architecture described in Section \ref{sec:cod_problem}. Since the Command Analysis task being solved is Conditioned Object Detection (Chapter \ref{ch:cod}), the focus here is on how the COD module is effectively integrated into the control framework.

The COD module is integrated into two different architectures, which vary in the number of control modules they use. Section~\ref{sec:ocpl_architecture_scm} outlines an architecture that employs a single control module to predict actions for the entire trajectory. In contrast, Section~\ref{sec:ocpl_architecture_dcm} describes an architecture that splits the control module into two distinct parts: one for computing actions during the reaching phase, and another for the final phase, where the specific primitive depends on the task.

\subsection{Single control module}
\label{sec:ocpl_architecture_scm}
The architecture (Figure \ref{fig:single_control_module}) composed of a single control module is essentially an instance of the general modular architecture depicted in Figure \ref{fig:end_to_end_vs_modular}. Specifically, the Command Analysis Module is replaced by the CTOD module (Section \ref{sec:cod_tod}), which takes as input the current agent observation $o^a_t$ and the command $c_m$, producing a task embedding represented by the \textit{target-object bounding box}, i.e., $z^{task}_t = bb^{target}_t$.

The Backbone Module, responsible for generating the control embedding $z^{control}_{t}$, is replaced by the same backbone used in MOSAIC. This backbone is a combination of a Convolutional Network, which encodes the agent observation $o^a_t$ and the demonstration frames $c_m$, and a Self-Attention mechanism to create correlated agent and command embeddings (Section \ref{sec:bc}). 

Finally, the Control Module follows the same implementation as in MOSAIC \cite{mandi2022towards_more_generalizable_one_shot}. In this case, the actions generated by the control module are sampled from a multivariate logistic distribution (Equation \ref{equation:logistic_distribution}), where the distribution parameters $\mu_{i}$ and $\sigma_{i}$ are estimated by MLPs implementing the Control Module.

\begin{equation}
    \label{equation:logistic_distribution}
    a_{t} \sim \sum_{i=1}^{m} \alpha(z_t) \, logistic(\mu_{i}(z_t), \sigma_{i}(z_t))
\end{equation}

Here, the embedding $z_t$ is a concatenation of $z^{control}_{t}$, generated by the MOSAIC Backbone, and $bb^{target}_t$, produced by the CTOD Module.

\begin{figure}[t]
    \centering
    \includegraphics[width=0.9\textwidth]{figures/images/ch3/single_control_module.jpg}
    \caption{Proposed Single-Control Module Architecture. In contrast to the general architecture described in Figure \ref{fig:end_to_end_vs_modular}, the Command Analysis module is replaced by the CTOD module, which generates the bounding box related to the target object. The chosen backbone is the MOSAIC architecture \cite{mandi2022towards_more_generalizable_one_shot}. The control module is now informed by both low-level positional information ($bb^{target}_{t}$) and a control-oriented embedding ($z^{control}_{t}$), enabling it to make more informed decisions.}
    \label{fig:single_control_module}
\end{figure}


\subsection{Double control modules}
\label{sec:ocpl_architecture_dcm}
Regarding the architecture consisting of multiple control modules, the discussion begins with the key observation that the tasks can be broadly divided into two phases. The first phase is typically a \textit{reaching phase}, where the robot must reach a target location. The second phase varies depending on the task: placing for Pick-Place, assembly for Nut-Assembly, stacking for Stack-Block, and pushing for Press-Button. Based on this, a dual control module architecture is proposed, where each control module is trained to learn the primitive associated with each phase. The rationale behind this approach is that training separate modules for simpler atomic primitives can lead to a more robust and reliable control system.

Figure \ref{fig:double_control_module} illustrates the overall architecture. The main difference can be observed in the bounding boxes received by the control modules. Specifically, the MOSAIC Backbone generates the control embedding $z^{control}_t$, as before. However, the Command Analysis Module, now referred to as the COD Module, generates both the bounding box for the target object, $bb^{target}_t$, and the bounding box for the final placing position, $bb^{place}_t$. 

The $bb^{target}_t$ is provided as input to the \textit{Reaching Control} module, while the $bb^{place}_t$ is supplied to the \textit{Placing Control} module. 

Additionally, each control module operates based on an enabling signal $s^{en}$, which is set to 1 at the beginning of the rollout and remains active until the Reaching Control module generates its first prediction for the closing command. After this point, $s^{en}$ is set to 0, and control is transferred to the Placing Control module.    
\begin{figure}[t]
    \centering
    \includegraphics[width=0.9\textwidth]{figures/images/ch3/double_control_module.jpg}
    \caption{Proposed Double-Control Module Architecture. In this architecture, the Control Module is split into two distinct modules, each responsible for learning a specific primitive: the \textit{reaching} primitive and the \textit{placing} primitive. The first module takes as input the bounding box corresponding to the target object ($bb^{target}_{t}$), while the second module receives the bounding box related to the final placing location ($bb^{placing}_{t}$). This separation allows for specialized control during both the reaching and placing phases.}
    \label{fig:double_control_module}
\end{figure}

% \smalltodo{add figure}
\section{Experimental results}
\label{sec:ocpl_experimental}

In this section, the performed experiments are going to be described. Specifically, in  
Section~\ref{sec:ocpl_dataset} the dataset used for training procedure will be described. Section~\ref{sec:ocpl_results} will report the obtained results.
\subsection{Dataset}
\label{sec:ocpl_dataset}

\subsection{Results}
\label{sec:ocpl_results}
This section presents the obtained results, divided into two main blocks. The first block (Section~\ref{sec:ocpl_results_scm}) discusses the results of the method described in Section\ref{sec:ocpl_architecture_scm}. The second block (Section~\ref{sec:ocpl_results_dcm}) covers the results of the method described in \ref{sec:ocpl_architecture_dcm}. For each method, results are reported for two different scenarios: first, where the method is trained in a single-task multi-variation scenario; and second, where the method is trained in a multi-task multi-variation scenario.

\subsubsection{Single control module}
\label{sec:ocpl_results_scm}
\paragraph*{Single-task multi-variation scenario}\mbox{}\\
\paragraph*{Multi-task multi-variation scenario}\mbox{}\\

\subsubsection{Double control modules}
\label{sec:ocpl_results_dcm}
\paragraph*{Single-task multi-variation scenario}\mbox{}\\
\paragraph*{Multi-task multi-variation scenario}\mbox{}\\
% \chapter{Experimental validation in real-world scenario}
\label{ch:real_world_application}
This chapter details the validation of the proposed methods in a real world scenario. Specifically, Section~\ref{sec:real_world_exp_setting} describes the experimental setup. Section~\ref{sec:real_world_dataset} discusses the dataset used to train the system. Finally, Section~\ref{sec:real_results} presents the results obtained.

\section{Experimental Setting}
\label{sec:real_world_exp_setting}
In this section, the experimental setup is explained and defined. Figure \ref{fig:workspace} illustrates the workspace where the robot operates, and compares it to the corresponding simulation environment. As can be observed, the simulation environment closely resembles the real-world counterpart. Both environments consist of the same robot agent, with identical camera configurations and workspace setup.

\begin{figure}[t]
    \centering
    \includegraphics[width=1.0\textwidth]{figures/images/ch5/workspace.jpg}
    \caption{Workspace comparison between real-world (left) and simulated (right) scenarios. Images are taken from the frontal camera (Top-Left), the gripper camera (Top-Right), lateral-left camera (Bottom-Left) and lateral-right (Bottom-Right).}
    \label{fig:workspace}
\end{figure}

Specifically, the experimental setup includes:
\begin{itemize}
    \item The Universal Robots UR5e robot \cite{ur5e}, equipped with the Robotiq 2F-85 gripper \cite{robotiq}, which acts as the agent.
    \item Four Zed-Mini stereo cameras \cite{zed}, one camera is mounted on the gripper, while the remaining three are positioned around the robot to ensure complete coverage of the workspace.
    \item A 100$\times$100 cm working table.
\end{itemize}

The reason for maintaining a high similarity between the real-world and simulation environments is to evaluate the potential for pre-training the model on a large and comprehensive simulation dataset, followed by fine-tuning on a smaller and incomplete real-world dataset. This approach allows leveraging the advantages of simulation for initial training while adapting the model to real-world conditions with minimal additional data.

\section{Dataset}
\label{sec:real_world_dataset}
For the real-world validation, the focus will be on testing the model on the \textbf{multi-variation} \textit{Pick-Place} tasks, focusing on a smaller set of variations compared to the complete 16 variations described in Section \ref{sec:cod_dataset}.

\begin{figure}[t]
    \centering
    \includegraphics[width=1.0\textwidth]{figures/images/ch5/real_world_dataset.jpg}
    \caption{Set of variations used in the real-world robot evaluation. For each variation, the first and last frames are provided}
    \label{fig:real_world_dataset}
\end{figure}


Figure \ref{fig:real_world_dataset} represents the 6 variations used in the real-world validation. As can be noted, these are essentially the first six variations of the Pick-Place Task used in the simulation environment.

A preliminary dataset composed of \textbf{40 trajectories} for each variation has been collected by teleoperating the robot with a console controller. Regarding the object placement, the same algorithm described in Section \ref{sec:ocpl_dataset} was applied. This means that the set of 4 bins ($15 \times 15 \times 7$ cm) was fixed in position, while the 4 boxes ($4 \times 4 \times 6$ cm) could vary in their position within a region of $60$ cm in length and $15$ cm in height. 

In this extensive placement area, a specific protocol was implemented for collecting the trajectories. The protocol dictates how the objects are positioned within the picking region. The various placements of the target object are illustrated in Figure \ref{fig:box_placement}. Although the workspace may appear somewhat limited, this dataset is the first real-world dataset in the context of Visual-Conditioned Imitation Learning to encompass such a large operational space. In contrast, other works are either restricted to simulation environments \cite{dasari2021transformers_one_shot,mandi2022towards_more_generalizable_one_shot} or present very simple scenarios where objects are placed in a much smaller area \cite{mandlekar2022matters}.
\begin{figure}[t]
    \centering
    \includegraphics[width=0.8\textwidth]{figures/images/ch5/box_placement.jpg}
    \caption{Placement configuration used for the trajectory collection. For each placement configuration, 4 trajectories were collected. The target object initially starts at the rightmost position, and in each subsequent trajectory, the box is moved to the adjacent position. This process is then repeated twice, each time with a different object orientation, resulting in a total of 40 trajectories for a given configuration.}
    \label{fig:box_placement}
\end{figure}

% \smalltodo{add figure}
% \smalltodo{add figure}
Trajectories are collected through teleoperation the robot is controlled in its operational space, with the controller that sends continous velocity commands on a specific axis, this velocity command is defined by the value read by the controller's potentiometers.
During teleoperation different informatin are recorded:
\begin{itemize}
    \item \textit{Images} from the front, laterals and gripper cameras, both RGB and Depth images are recorded.
    \item \textit{Proprioceptive information} like joints positions and velocities.
    \item \textit{Trajectory state}, which is manually changed by the human operators based on the task state. The phases are:
        \begin{enumerate}
            \item \textit{Start}, this phase starts at the beginning of the trajectory till the robot gripper is perpendicular to the target object.
            \item \textit{Approaching}, this phase starts when the robot approaches the target object with its discending movement.
            \item \textit{Picking}, this phase is characterized by the gripper that is ready to be closed to pick the object, and contains the closing command.
            \item \textit{Moving}, this phase starts after the robot picks the object and lift it from the table, this phase ends when the gripper is perpendicular to the target bin.
            \item \textit{Placing}, this phase starts when the robot is ready to place the object, starting the discending phase towards the target bin, this phase also contains the opening command.
        \end{enumerate}
    \item \textit{Objects bounding boxes}, which are are generated automatically, requiring minimal input from human operators. At the start of the trajectory, the operator needs to specify the position of each object in the scene, including both boxes and bins. This is done by displaying the first frontal frame of the trajectory and clicking on the objects with the cursor. The positions, initially defined in discrete pixel-space, are then converted into continuous world-space through a series of transformations, using the camera's intrinsic and extrinsic parameters. The extrinsic parameters are obtained via a calibration procedure that uses ARUCO markers.
\end{itemize}
The overall space coverage of the real-world dataset is shown in Figure \ref{fig:real_world_coverage}. As can be observed, the coverage is more limited and considerably noisier compared to the simulated dataset, even when restricted to the same variations. This is due to the collection of fewer trajectories and the use of teleoperation without any hand-written control rules, which typically generate smoother and more deterministic robot behaviors. These limitations introduce additional challenges in learning a robust control policy, especially for generalizing to different placements of the target object. This issue will be addressed in this thesis by initially training the policy in the simulation environment, where a complete dataset is available, and then fine-tuning it on the noisy real-world dataset.
\begin{figure}[t]
    \centering
    \includegraphics[width=1.0\textwidth]{figures/images/ch5/real_world_dataset_coverage.jpg}
    \caption{(Left) Trajectory distribution along the x-y axis of the real-world dataset. (Right) Trajectory distribution along the x-y axis of the simulated dataset, constrained to the same variations and number of trajectories as the real-world counterpart. It can be observed that the real-world dataset exhibits a much sparser and noisier distribution, due to the fact that the trajectories are collected via teleoperation.}
    \label{fig:real_world_coverage}
\end{figure}

% \smalltodo{add figure}
\section{Results}
\label{sec:real_results}


\section{Conclusion}
In conclusion, this chapter presented the experimental validation of the proposed methods in a real-world scenario. Both the proposed detectors and control policies were tested in a single-task multi-variation scenario.

Regarding the conditioned object detectors, it was observed that the system successfully identified the target object even when the video demonstration was provided by a simulated robot, which had a different visual appearance compared to the real robot. This highlights the strong potential of the proposed method to generalize across different demonstrators and environments, such as human demonstrators. Generally, the patterns observed in the simulated environment were replicated in the real-world tests, where the system consistently identified the target object during the reaching phase but generated false positives during the manipulation and placing phases. However, the drop in precision was more pronounced in the real-world setting due to occlusions and inaccuracies in the automated bounding box generation process.

As for the control policies, the results demonstrated that the proposed architectures were able to successfully complete tasks in real-world scenarios. The object conditioned control policies consistently reached the target object, leading to successful task completion. Across all variations of the proposed method, an average reaching rate of $85.40\%$ was achieved, compared to $14.12\%$ for the MOSAIC baseline. This indicates that object priors can be effectively leveraged to simplify the control problem, resulting in a policy that consistently reaches the target object, even when using a dataset with low-quality demonstrations. In terms of overall success rate, the MOSAIC-COD module achieved the highest rate of $55.00\%$. While this may not seem particularly high, it is a promising result given the challenging nature of the dataset, which contains fewer and noisier trajectories compared to the simulated environment. This result becomes even more significant when compared to the MOSAIC baseline's success rate of $0.00\%$.

Generally speaking, the most critical error observed involves collisions with the target object. This type of error could be significantly reduced by integrating additional exteroceptive modalities, such as depth images from the gripper camera. Incorporating this data would enable the system to better detect the target object and avoid collisions during the picking phase, which is expected to result in a substantial improvement in the overall success rate.
% \chapter{Graph Neural Network for heuristic estimation}
The goal of this chapter is to evaluate and review the possibility of leveraging Graph Neural Network (GNN) in the context of robotics learning. Specifically, this chapter will devided into different sections: Section \ref{sec:gnn_related_works} will discuss the related works and in general the application of GNN in the context of Learning from Demonstration.
\smalltodo{to continue}
\section{Related Works}
\label{sec:gnn_related_works}
This section reviews prior research on the application of Graph Neural Networks (GNNs) to robot learning tasks. As illustrated in Figure \ref{fig:gnn_taxonomy}, GNNs can be utilized in various ways; however, all applications share a fundamental concept: leveraging GNNs' ability to explicitly represent dynamic relationships between objects involved in robotic manipulation, which is crucial information for robotics and task execution.

\begin{figure}[t]
    \centering
    \includegraphics[width=0.8\textwidth]{figures/images/ch4/gnn_taxonomy.jpg}
    \caption{Proposed taxonomy for GNNs methods used in the context of robotic learning.}
    \label{fig:gnn_taxonomy}
\end{figure}


In general, most research has focused on solving \textit{task planning} problems. According to classical literature \cite{geffner2013concise}, a task-planning problem can be defined as a state-transition model $\Pi = \left( S, A, s_{0}, G \right)$, where $S$ is the set of states, $A$ is the set of actions, $s_{0}$ is the initial state, and $G$ is the goal state. Here, an action $a \in A$ is a function that maps a state $s \in S$ to a new state (successor) $a(s)$, i.e., $a: S \rightarrow S$. Essentially, a plan is a sequence of actions $\tau = a_{1}, a_{2}, \dots, a_{n}$ that enables the system to transition from the initial state $s_{0}$ to the goal state $s_{n} = a(s_{n-1}) = G$. Various approaches have been proposed to solve the problem of finding a plan given a specific initial and goal state. These approaches either adhere to the classical PDDL formalism \cite{aeronautiques1998pddl}, and thus fit within the traditional planning framework, or introduce novel representations that do not rely on PDDL to address the planning problem.

This review begins by examining methods that follows the \textit{Classic Planning Formalism}  (Section \ref{sec:pddl_formalism}), followed by a brief discussion of novel approaches that do not use this kind of formalism (Section \ref{}).

\subsection{Classic Planning Formalism}
\label{sec:pddl_formalism}
In this section, we will describe the methods that follow the classic task-planning formalism. Before introducing the methods, it is crucial to define the formalism with a set of foundational definitions.

Task-planning is a well-known problem that has been studied for many years. One of the most influent formalizations, still widely used today, is STRIPS (Stanford Research Institute Problem Solver) \cite{fikes1971strips}, which was developed in the 70s. STRIPS was initially an automated planning solver, but its primary contribution lies in its definition of a planning problem, which has served as the foundation for much of the research in subsequent decades.

According to the STRIPS formalism, a planning task is defined as a tuple $\Pi = (P, A, s_0, G, c)$, where:
\begin{itemize}
    \item $P$ is a set of propositions (also called facts).
    \item $A$ is a set of actions.
    \item A state $s$ is a subset of predicates, $s \subseteq P$, where $s_0$ is the initial state,
    \item $G \subseteq s$ represents the goal conditions.
    \item $c: A \leftarrow R$ is the cost function, which assign to an anction $a \in A$ a real-value, $c(a)$ representing the cost of performing a given action.
\end{itemize}

An action $a \in A$ is defined as a tuple $a = \left(pre(a), add(a), del(a)\right)$, where $pre(a), add(a), del(a) \subseteq P$. These represent:
\begin{itemize}
    \item $pre(a)$: the preconditions (i.e., predicates that must be true to perform the action).
    \item $add(a)$: the added conditions (i.e., predicates that become true after the action is executed).
    \item $del(a)$: the deleted conditions (i.e., predicates that become false after the action is executed).
\end{itemize}
with the constraint that $add(a) \cap del(a) = \emptyset$. According to this formalism, an action is applicable in a state $s$ if $pre(a) \subseteq s$, and it results in the successor state $s' = (s \setminus del(a)) \cup add(a)$.

Building on these foundational concepts, the \textbf{Planning Domain Definition Language} (PDDL) was introduced in the 90s \cite{aeronautiques1998pddl}. PDDL is a human-readable format for describing automated planning problems. It provides a way to describe, the possible states of the world, the set of available actions, where each action description includes the prerequisites and the effects of the action, a specific initial state of the world and a specific set of desired goals. 

PDDL separates the model of a planning problem into two main components:
\begin{enumerate}
    \item The domain description, which defines the elements that are common across all problems within a given domain.
    \item The problem description, which specifies the specific planning problem, including the initial state and the goals to be achieved.
\end{enumerate}

\input{figures/ch4/example_domain_problem.tex}

Figure \ref{fig:domain_problem} provides examples of domain and problem definitions for the Blocks-World environment. As shown, when using the PDDL formalism, it is essential to fully define the elements of the world (e.g., blocks), the available actions (e.g., pick-up), along with their corresponding preconditions (i.e., predicates that must be true for the action to be executed) and postconditions (i.e., the effects of the actions). In the problem definition, specific instances of objects in the environment are specified, along with the initial state and goal state, both of which are defined based on the predicates available in the domain definition.

Once the problem is defined using the PDDL formalism, a complete parameterized instance must be specified to obtain a solution to the planning problem. This involves solving what is known as the \textbf{grounding problem}. During this process, all predicate and action variables must be instantiated with the objects defined in the problem. For instance, in the example shown in Figure \ref{fig:domain_problem}, the variable \( x \) is replaced with objects \( b1 \), \( b2 \), and \( b3 \) from the problem definition, leading to a complete enumeration of all predicates and actions. The result of this process is known as the \textit{Grounded Planning Problem}, which follows the same formalism as the STRIPS representation described earlier.

The grounded problem then serves as the input for planning algorithms, such as the well-known Fast Downward \cite{helmert2006fast} or LAMA \cite{richter2010lama}. While delving into the specifics of these algorithms is beyond the scope of this section, it is crucial to note that they are essentially \textit{search algorithms}. Given a state-space represented as a graph, which is derived from the Grounded STRIPS representation, these algorithms compute the (estimated) the shortest path in the state-space, connecting the source node (representing the initial state) to the target node (representing the goal state).

The limitations of these approaches have been well-known for a long time. Specifically, these algorithms do not scale well as the size of the problem (in terms of actions, predicates, and objects) increases. There are two main areas where higher problem dimensionality can cause performance issues:

\begin{enumerate}
    \item \textit{Search Phase}, as the dimensionality of the problem increases, the state-space expands correspondingly. This increases the time required to explore the state-space and find a solution path.
    \item \textit{Grounding Problem}, as the problem dimensionality grows, the time needed to generate a complete enumeration of all object combinations grows exponentially.
\end{enumerate}

To address the first issue, research has focused on developing methods to accelerate the search phase by introducing \textit{heuristics} that guide the search more efficiently towards the goal state. For the second issue, the literature has proposed methods that perform \textit{partial} grounding, concentrating on the action instances that are most relevant to the specific problem instance. Examples of these methods will be discussed in the following two paragraphs.

Here, a heuristic is a function $h: S \rightarrow \mathbb{R}$, which maps a state $s \in S$ to a real value $h(s)$, representing an estimate of the cost to reach the goal state from the state $s$. The optimal heuristic, denoted $h^{*}(s)$, is the heuristic that gives the exact cost of the optimal plan to reach a goal state from $s$. A heuristic is considered \textit{admissible} if it never overestimates the actual cost, i.e., $h(s) \leq h^{*}(s), \forall s \in S$. By utilizing the heuristic cost, a search algorithm can focus on exploring the most promising states instead of performing an exhaustive search of the entire state space.

In the context of task planning, a common heuristic is computed by solving the \textbf{Relaxed-STRIPS} problem. This is a simplified version of the STRIPS problem, where the delete conditions ($del$) are ignored, i.e., an action $a$ is represented as $a = \left(pre(a), add(a), \emptyset \right)$. The Relaxed-STRIPS problem is easier to solve because removing the $del$ conditions relaxes some of the constraints, allowing the state to include predicates that logically cannot be true simultaneously. By solving this relaxed problem, an estimated cost for reaching the goal from a given state can be computed. This approach has been used in well-known algorithms such as Fast-Downward \cite{helmert2006fast} and combined with novel heuristics in LAMA \cite{richter2010lama}.

\paragraph*{Heuristic Estimation}\mbox{}\\
In this paragraph the methods that methods that propose GNN for computing heuristics estimation through the usage of GNNs are discussed. Specifically, the review will focus on two recent works \cite{shen2020learning,chen2024learning}.

The first remarkable work towards the use of GNN for heuristic estimation was proposed by authors of \cite{shen2020learning}. Here, authors started from the idea to leverage data-driven approaches to learn an heuristic function, and explore the possibility of such methods to generalize to different cardinalities and problems, and comparing the performance of such methods with classic heuristic functions.

The first step to solve this problem is related to the definition of the input. Specifically, the authors started by the Relaxed-STRIPS formulation. The STRIPS problem can be easly formulated as a graph if the following observations are done:
\begin{itemize}
    \item A node, can be described a predicate which can be either a pre-condition or an added condition.
    \item The edge is represented by the action that connects the pre-conditions to the added-conditions. 
\end{itemize}
\smalltodo{add figure}
Figure \ref{} illustrates the mapping of predicates and actions to graph nodes and edges. As can be observed, an action connects multiple nodes. To model this, the authors use a generalized graph structure known as a \textit{hypergraph}. Formally, a hypergraph is defined as a triple $G = (u, V, E)$, where $V = \left\{ \textbf{v}_{i}: i \in \left\{1, 2, \dots, N^{v} \right\} \right\}$ is the set of $N^{v}$ vertices, and $E = \left\{ (e_{k}, R_{k}, S_{k}): k \in \left\{1, 2, \dots, N^{e} \right\} \right\}$ is the set of $N^{e}$ hyper-edges. In this structure, $R_{k}$ is the receiver set, containing the indices of the nodes for which the $k$-th edge acts as an incoming edge, and $S_{k}$ is the sender set, containing the indices of the nodes for which the $k$-th edge acts as an outgoing edge. Then, $\textbf{v}_{i}$ is the node embedding, which has been implemented as a binary-vector $v_i = [x_s, x_g]$, where $x_s, x_g \in \left\{0,1\right\}$. $x_s=1$ iff the i-th predicate is true in the state s and $x_g=1$ iff the predicate is true in the goal-state. While, $e_{k}$ is the edge embedding, $e_k = [c(a_k), |Pre(a_k)|, |Add(a_k)|]$, where $c(a_k)$ is the cost of the operator, $|Pre(a_k)|$ is the number of pre-condition and $|Add(a_k)|$ is the number of added condition.

Once the graph structure has been defined, the module that processes this input can be described. Specifically, the authors propose leveraging the concept behind the \textit{Interaction Network} \cite{battaglia2016interaction}. The architecture, depicted in Figure \ref{}, implements the computational flow shown in Figure \ref{}, and can be divided into the following three steps.

\begin{enumerate}
    \item The \textit{Encoding Block} consists of two functions, $\phi^{e}$ and $\phi^{v}$, both implemented as Multi-Layer Perceptrons (MLPs). These functions transform the initial binary inputs into higher-dimensional representations, specifically, $\phi^{e}(e_j)$ produces $e_{hid}^{0}$ and $\phi^{v}(v_i)$ produces $v_{hid}^{0}$, where both $e_{hid}^{0}$ and $v_{hid}^{0} \in \mathbb{R}^{32}$. This step expands the original binary vectors (of size 2 or 3) into 32-dimensional vectors.

    \item The \textit{Core Block} implements the Message-Passing Iterative Procedure and is divided into the following steps:

        \begin{enumerate}
            \item \textit{Edge Block}: For each edge (representing an action), $e_{hid,j}^{t-1}$ is generated by concatenating the embeddings of the receiver and sender nodes associated with the edge. An MLP encoder is then applied to this concatenated information to create a new edge representation $e_{hid,j}^{t}$.

            \item \textit{Node Block}: Similar to the Interaction Network, only the receiver nodes' states are updated. The edge information influencing each receiver is aggregated by summing the embeddings of the incoming edges using the function $\rho^{e \rightarrow v}$. The node input is constructed by concatenating the incoming edge embeddings $e_{hid,j}^{t}$, the initial node embedding $v_{hid,i}^{0}$, and the previous node state $v_{hid,i}^{t-1}$. This concatenated input is then passed through an MLP to generate the updated node representation $v_{hid,i}^{t}$.

            \item \textit{Global Block}: This block updates the global graph representation by aggregating both node and edge embeddings using the functions $\rho^{e \rightarrow u}$ and $\rho^{v \rightarrow u}$, respectively. The aggregated information is passed through an MLP to produce the new global graph representation.
        \end{enumerate}

    \item The \textit{Decoding Block} acts as a decoder, implemented as an MLP. It takes the updated global graph representation as input and generates a heuristic value for a given state, aiding in the decision-making process.
\end{enumerate}

The operations in the core block are repeated for $M$ iterations, where $M = 10$ in this work.
The system is trained by minimizing the loss function defined in Equation \ref{equation:loss_func}. Here, $h^{*}(s)$ represents the ground truth heuristic value for state $s$, obtained by solving the training planning problem with a classical planning algorithm.
\begin{equation}
    \mathcal{L}_\theta(\mathcal{B})=\frac{1}{|\mathcal{B}|} \sum_{\left(G, h^*(s)\right) \in \mathcal{B}} \frac{1}{M} \sum_{t \in \{1, \ldots, M\}}\left(h_t^\theta(G)-h^*(s)\right)^2
    \label{equation:loss_func}
\end{equation}

Interesting results were obtained during the experiments. Notably, the system demonstrated the ability to learn both domain-dependent heuristics (when tested on the same domain it was trained on) and domain-independent heuristics (when tested on a different domain). This shows that the trained GNN was capable of generating meaningful heuristic values that could be effectively used by search algorithms like A*. However, a general drawback of this approach is the time required to compute the heuristic, particularly when the model is run on a CPU instead of a GPU, which significantly impacts performance.

Development of such work was proposed by \cite{chen2024learning},
\smalltodo{ToContinue}

\paragraph*{Grounding Problem}\mbox{}\\
As stated before, one of the preliminary problem to solve, before running the search algorithm, is the Grounding Problem. In this problem 
\smalltodo{ToContinue}
\subsection{Problem formulation}
\label{sec:problem_formulation}
In this section, the problem of Learning from Demonstration, also known as Imitation Learning (IL), will be formalized.

The core idea behind IL is to program a robot by allowing a human expert to demonstrate how to solve a specific task. This approach avoids methods that demand intricate, handcrafted rules dictating the actions of machines and the dynamics of their operating environments, which require significant time and coding expertise. To achieve this goal, a way to transfer knowledge from the expert to the robot is necessary. One possibility is to leverage \textit{data-driven} methods that can extrapolate control rules from demonstrated trajectories~\cite{osa2018algorithmic}.

In this context, we need to define two general actors: the \textbf{expert}, who demonstrates the desired behaviors, and the \textbf{learner}, whose goal is to learn and replicate the behaviors demonstrated by the expert~\cite{osa2018algorithmic,zare2024survey}. Both the expert behaviors and the learner behaviors are described in terms of \textbf{policy}, generally denoted as $\pi^{E}$ and $\pi^{L}$ respectively.

Since IL relies on expert demonstrations, these are collected in a dataset $\mathcal{D}^{E}=\left\{\left(\boldsymbol{\tau}^{E}_{i}, c_{i}\right)\right\}_{i=1}^{N}$, where:
\begin{itemize}[noitemsep]
    \item $\boldsymbol{\tau}^{E}_{i}$ is the $i^{th}$ demonstrated trajectory, generated by the expert policy $\pi^{E}\sim\boldsymbol{\tau}^{E}_{i}$. It can be described as:
        \begin{itemize}[noitemsep]
            \item A \textit{state-action sequence}, i.e., $\boldsymbol{\tau}_{i} = [s_{0}, a_{0}, \dots, s_{T}, a_{T}]$, when the ground truth action performed by the expert is available.
            \item A \textit{state-only sequence}, i.e., $\boldsymbol{\tau}_{i} = [s_{0}, \dots, s_{T}]$, when the ground truth action is not available.
        \end{itemize}
    \item $c_{i}$ is the \textit{context-vector}, containing task-related information such as the initial state of the system $s_{0}$, the position of the target object, or a representation of the task to be executed (e.g., a natural language description of the task or video demonstrations).
\end{itemize}

The state $s_{t}$ can be defined in various ways. RGB images have been used in different works, either from a third-person point of view \cite{james2018task_embedded}, representing the robot and its workspace, or from a first-person point of view with the camera mounted on the robot \cite{jang2022bc_z} or the gripper \cite{mees2022calvin}. Additionally, 3D point-clouds generated by RGB-D images can also be used \cite{shridhar2023perceiver}. This high-level state representation can be enriched with proprioceptive signals such as joint positions, velocities, and torques \cite{zhang2018deep_vr_teleoperation}.

Regarding the policy, the predicted action, $\hat{a}_{t}$, can be generated into two distinct ways. In one case, it is described as a \textit{deterministic function}, meaning that it directly determines the action as follows: $\hat{a}_{t} = \pi^{L}(s_{t}, c_{i})$\cite{jang2022bc_z}. In the other case, it takes on a probabilistic definition, where it represents a probability distribution from which to sample the desired action: $\hat{a}_{t} \sim \pi^{L}(s_{t}, c_{i})$~\cite{mandi2022towards_more_generalizable_one_shot}.

According to the definitions given in~\cite{osa2018algorithmic,fang2019survey}, the learner policy $\pi^{L}$ can be defined with respect to different abstraction levels:
\begin{itemize}
    \item \textit{Symbolic Characterization}, the policy maps states, and context to a sequence of options, i.e., $\pi: s_{t}, c \rightarrow [o_1, \dots, o_T]$, where each option is a sequence of actions. With this representation, complex tasks can be decomposed into a sequence of simple movements. However, it is hard to achieve an accurate task segmentation and motion ordering;
    \item \textit{Trajectory Characterization}, the policy maps context to trajectory, i.e., $\pi: c \rightarrow \boldsymbol{\tau}$. Because it allows the initial state to be mapped to a complete sequence of actions, this representation can be used to obtain the options in the Symbolic Representation. However, they need as many dynamic features as possible, that can be difficult to obtain;
    \item \textit{State-Action Characterization}, the policy maps states(-context) to actions, i.e., $\pi: s_{t}, c \rightarrow a_{t}$. This representation makes it possible to map the current state directly to the corresponding action. However, it is easy for errors to accumulate in long-term processes. The action $a_{t}$ can also be defined in various ways, depending on the type of control implemented by the method. Some methods operate in the joint-space, predicting motor control signals such as positions, velocities, and torques~\cite{zhang2018deep_vr_teleoperation}. Other methods work in the operational space, where actions are assigned either an absolute pose with respect to a world frame $\mathcal{W}$, i.e., $a_{i} = \left[p_x^{\mathcal{W}}, p_y^{\mathcal{W}}, p_z^{\mathcal{W}}, r_x^{\mathcal{W}}, r_y^{\mathcal{W}}, r_z^{\mathcal{W}}\right]$~\cite{mandi2022towards_more_generalizable_one_shot}, or a displacement relative to the current gripper position, i.e., $a_{i} = \left[\Delta p_x^{\mathcal{W}}, \Delta p_y^{\mathcal{W}}, \Delta p_z^{\mathcal{W}}, \Delta r_x^{\mathcal{W}}, \Delta r_y^{\mathcal{W}}, \Delta r_z^{\mathcal{W}}\right]$~\cite{jang2022bc_z}.

\end{itemize}

The expert and the learner, through their policies $\pi^{E}$ and $\pi^{L}$, act on an environment modeled as a \textit{Markov Decision Process} (MDP)~\cite{kroemer2021review_robot_learning}. An MDP is defined as a tuple $(S, A, R, T, \gamma)$, where:
\begin{itemize}
    \item $S \subseteq \mathbb{R}^{n}$ is the set of states (e.g., joint positions and/or images).
    \item $A \subseteq \mathbb{R}^{n}$ is the set of actions (e.g., desired end-effector pose, desired joint torques).
    \item $R(s, a, s')$ is the \textit{reward function}, which expresses the immediate reward for executing action $a$ in state $s$ and transitioning to state $s'$.
    \item $T(s' | s, a)$ is the \textit{transition function}, which defines the probability of reaching state $s'$ after executing action $a$ in state $s$. This distribution, which describes the \textit{system dynamics}, can be given a priori or learned (Model-Based methods), or it may not be considered at all (Model-Free methods).
    \item $\gamma \in [0,1]$ is the discount factor, expressing the agent's preference for immediate rewards over future rewards.
\end{itemize}

With respect to the given MDP definition, the reward function plays different roles depending on the approach used:
\begin{itemize}
    \item In \textit{Behavioral Cloning} (BC) methods, the reward function is not explicitly used. Instead, a surrogate loss function is employed.
    \item In \textit{Inverse Reinforcement Learning} (IRL), the reward function is learned, under the assumption that the expert acts (near-)optimally with respect to some unknown reward function.
    \item In \textit{Generative Adversarial Imitation Learning} (GAIL) and \textit{Learning from Observations} (LfO), the role of the reward function varies based on the specific method, as will be explained in Section \ref{sec:gail} and Section \ref{sec:lfo}.
\end{itemize}



\subsection{Source of demonstration}
\label{sec:sod}
As stated in Section \ref{sec:problem_formulation}, IL methods rely on a dataset $\mathcal{D}^{E}$ of expert demonstrations. In this section, we will review the different ways to obtain these demonstrations, following the taxonomy proposed in \cite{fang2019survey}.
\begin{figure}[tb]
     \centering
     \begin{subfigure}[b]{0.45\textwidth}
         \includegraphics[width=\textwidth]{figures/images/direct_demonstration/kinesthetic.jpg}
         \caption{Example of kinesthetic teaching~\cite{johns2021coarse_to_fine}.}
         \label{fig:kinesthetic}
     \end{subfigure}
     \hfill
     \begin{subfigure}[b]{0.5\textwidth}
         \includegraphics[width=\textwidth]{figures/images/direct_demonstration/teleoperation.jpg}
         \caption{Example of teleoperation~\cite{zhang2018deep_vr_teleoperation}.}
         \vspace{0.46cm}
         \label{fig:teleoperation}
     \end{subfigure}

    \caption{Examples of direct demonstration}
    \label{fig:direct_demonstrations}
\end{figure}


\paragraph*{Direct Demonstration}  \mbox{} \\
In the case of \textit{Direct Demonstration}, the expert trajectory is a state-action sequence, where the action is obtained directly from the robot. Specifically, the robot can be guided in task execution through \textit{kinesthetic teaching} \cite{caccavale2019kinesthetic,johns2021coarse_to_fine}, \textit{teleoperation} \cite{zhang2018deep_vr_teleoperation,mandlekar2018roboturk,jang2022bc_z,brohan2022rt,ebert22Bridge,mandlekar2023mimicgen}, or a \textit{(hand-)written policy} \cite{dasari2020robonet,dasari2021transformers_one_shot,mandi2022towards_more_generalizable_one_shot,chang2023one}.

In kinesthetic teaching (Figure \ref{fig:kinesthetic}), the human operator contacts and guides the robot, recording parameters such as the gripper pose, joint positions, and velocities. This has been one of the first approaches for the LfD problem \cite{lee2011incremental,saveriano2015incremental} because there is no need to consider differences in kinematics between human and robot. As a result, the data has less noise, and there is no need for expensive external tools for teleoperation. However, the robot must be passively controllable and require direct contact, which introduces safety problems and can be unintuitive for robots with multiple degrees of freedom.

In teleoperation (Figure \ref{fig:teleoperation}), the human operator remotely guides the robot with a joystick, control panel, or wearable device. These tools allow for higher safety since there is no direct contact between the robot and the human expert. Teleoperation systems have been used in various works. For example, the authors in \cite{mandlekar2018roboturk,mandlekar2019scaling} proposed a teleoperation framework named Roboturk, which enables the collection of large-scale demonstration datasets \cite{mandlekar2019scaling,mandlekar2022matters} for both simulated and real-world robots using a mobile phone as the controller. The authors in \cite{zhang2018deep_vr_teleoperation,jang2022bc_z,brohan2022rt} used virtual reality controllers, allowing the human operator to intuitively move in the 3D environment, mapping the pose of the controllers to the corresponding gripper pose. This technology is of interest because it provides a safe and intuitive way to teleoperate a robot. However, the main drawback is the lack of haptic feedback, which can be mitigated by using haptic interfaces such as \cite{cyberglove,touch}.

Demonstrations collected with hand-written policies rely on the fact that the expert has access to ground truth information, such as the position of the object of interest. This assumption can be valid when a simulated environment is used to train, test, and validate the proposed methods, as seen in \cite{dasari2021transformers_one_shot,mandi2022towards_more_generalizable_one_shot,chang2023one}. In these cases, the authors used a well-known simulation environment in the robotic learning community named Robosuite \cite{zhu2020robosuite}. Here, demonstrations were collected to train methods using hand-written policies that have access to ground truth positional information about the object of interest, solving tasks such as Pick-Place, Nut-Assembly, Stack-Block, and more. Simulation environments facilitate the evaluation of the proposed methods and ensure reproducibility and consistent testing. The idea of using hand-engineered policies is not limited to simulation environments. Indeed, the authors in \cite{dasari2020robonet} used automatic grasping primitives combined with diagonal Gaussian distributions to collect demonstrations on a real-world robot. This approach aims to collect as many trajectories as possible with minimal human effort.

\begin{figure}[tb]
     \centering
     \begin{subfigure}[b]{0.63\textwidth}
         \includegraphics[width=\textwidth]{figures/images/wearable_indirect_teaching.jpg}
         \caption{Example of indirect demonstration based on wearable device~\cite{liu2019_mirroring_without_overimitation}.}
         \label{fig:wearable_indirect}
     \end{subfigure}
     \vfill
     \vfill
     \begin{subfigure}[b]{0.63\textwidth}
         \includegraphics[width=\textwidth]{figures/images/visual_indirect_teaching.jpg}
         \caption{Example of direct demonstration based on human video demonstration~\cite{smith2019avid}.}
         \label{fig:visual_indirect}
     \end{subfigure}

    \caption{Examples of indirect demonstration.}
    \label{fig:indirect_demonstrations}
\end{figure}


\paragraph*{Indirect Demonstration}   \mbox{} \\ 
As discussed previously, in direct demonstration, an expert controls the learner agent and records the actions performed. The concept behind Indirect Demonstration (Figure \ref{fig:indirect_demonstrations}) is to collect demonstrations that are completely disconnected from the target robotic platform. In the most promising scenario, a human demonstrator performs the desired tasks and records their operations. The learner, starting from this set of recordings, must be able to extrapolate the knowledge needed to replicate the observed tasks.

In this case, the expert demonstrations are state-only trajectories. Since the action space between the human demonstrator and the robot is different (consider just the different embodiment), it is not possible to directly use the human joint trajectories to minimize a supervised loss where the predicted value is related to the robot action space.

Initially, methods that follow this approach used wearable devices to capture human movement and record it \cite{nakaoka2007learning,liu2019_mirroring_without_overimitation}. For example, in \cite{nakaoka2007learning}, the authors used a motion capture system to record the movement of a dancer and then transfer these trajectories to a humanoid robot. Similarly, in \cite{liu2019_mirroring_without_overimitation}, the authors used a tactile glove \cite{liu2017glove_force} to record the movement performed by a human hand in the operation of opening bottles (Figure \ref{fig:wearable_indirect}).

In this line of research related to indirect demonstrations, there are also novel methods \cite{smith2019avid,torabi2019recent_advances_lfo,xiong2021learning_by_watching,wang2023mimicplay,qian2024contrast} that, inspired by the way humans learn by watching task execution, remove the assumption of having access to recorded human joint trajectories. In this case, the demonstrations are just videos of the human demonstrator (Figure \ref{fig:visual_indirect}). Here, the system must infer from the video not only the intent of the task but also how this task can be solved and transform this information into its own action space.

Generally, methods based on wearable devices allow for very intuitive demonstrations, including critical information for manipulation tasks such as force and tactile information \cite{liu2019_mirroring_without_overimitation}. While methods based on just video demonstrations are very promising because they allow for the collection of demonstrations in the most intuitive and scalable way possible (potentially any video of a performed task can be used). However, both these approaches have to solve the \textit{correspondence problem}, i.e., the system must be able to map motion captured in human space into the corresponding motion of the robot. In Section \textcolor{red}{TODO}, the different ways this problem has been solved in the context of visual demonstration will be explained in detail.


% \chapter{Conclusions}
\label{sec:conclusion}
In conclusion, this thesis is framed in the context of \textit{Learning from Demonstration}, a supervised learning problem that leverages data-driven methods to learn control functions, enabling robots to perform tasks. Section \ref{sec:sota} explores the formulation of the learning problem, addressing all its aspects, from the mathematical foundations to the techniques used for data collection (Section \ref{sec:problem_formulation}), and concludes with a comprehensive taxonomy of how this problem has been addressed in the literature (Section \ref{sec:lfd}). From this analysis, it is clear that the current methodologies do not lead to a ``general-purpose" robotic platform. Instead, the learned policies are typically limited to controlling the robot for the specific task on which they were trained, with limited generalization capabilities to different initial configurations of known objects. This limitation prevents a human operator from commanding the robot to perform arbitrary tasks, a highly desirable feature in agile and user-friendly robotic platforms, especially in modern industrial collaborative environments.

For this reason, this thesis addresses the problem of \textit{Multi-Task Imitation Learning} (Section \ref{sec:occp_mtil}), where a policy is learned to perform multiple variations of a given task, as well as multiple distinct tasks. The policy also incorporates the capability to select, at runtime, the desired task to execute, either through commands based on textual prompts or by using videos of other agents performing the desired task in different configurations. Among these two modalities, the thesis focuses on \textit{Video-Conditioned Multi-Task Imitation Learning}, which aims to train control policies where the desired task is specified via video demonstrations of other agents completing the task.

Specifically, in this thesis a different approach to the problem is leveraged. Indeed, started from the consideration that in the Multi-Task scenario there are two main problems to be solved, one related to the \textbf{command analysis and understanding}, which can be seen as a \textit{cognitive-problem}, where starting from the current observation of the scene and the command, the system must be able to understand the task intent by localizing for example the objects of interest, and the second related to the \textbf{action generation}, which can be seen as a \textit{control-problem}, where starting from the current agent state and the information coming from the command analysis, must generate the correct action towards the task completion. Based on these considerations, an intuitive way to model the problem is through a modular system, however all the state-of-the-art methods proposed in literature for the problem in hand are based on end-to-end architectures (Section \ref{sec:occp_related_works}), which means that models starting from high-dimentional inputs (images), are converted into low-level actions, generally the pose of the robot's end-effector with respect to some fixed reference frame. 

In the state-of-the-art methods validation (Section \ref{sec:ocpl_results}) it was observed how the SoTA end-to-end architectures are able to generate valid and reasonable trajectories, however they complete the task by manipulating the wrong object, this means that the Backbone Module is not able to inform the Control Module with the position of the target object. Indeed, it was observed that by performing just 2 ground truth actions (e.g., actions generated through an hand-written control with access to ground state information), the success-rate improved in a considerable way.

For this reason the problem of \textit{Conditioned Object Detection} has been formulated and proposed in Chapter \ref{ch:cod}, where a conditioned convolutional neural network has been proposed with the ability to generate category agnostic bounding-boxes that given in input the current agent observation and the command is able to localize the objects of interest (e.g., the box to manipulate as well as the bin where place the object). This module was experimental tested in a simulated environment both in a Single-Task Multi-Variation and in a Multi-Task Multi-Variation, and it was observed how the proposed module is actually able to identify the locations of the objects of interest with a high precision, considering that the lowest average IoU is equal to \textbf{0.563}. 

Once the COD module was validated and tested, it was integrated into the SoTA framework to validate the hypothesis that solving the localization problem with a dedicated module can reduce the \textbf{object misidentification problem}, thereby improving both the overall success rate and the system's interpretability. The generated bounding box provides information about where the robot is going to move. Specifically, two modules were proposed and tested: MOSAIC-CTOD and MOSAIC-COD. The first module integrates the CTOD module (Section \ref{sec:cod_tod}), which detects only the target object, while the second module integrates the COD module (Section \ref{sec:cod_tofpd}), which detects both the target object and the final location of interest.

These modules were tested in both single-task multi-variation scenarios and multi-task multi-variation scenarios. The results, described in Section \ref{sec:ocpl_results_scm} and Section \ref{sec:ocpl_results_dcm}, show that in the single-task setting, the target object prior effectively facilitates consistent and robust target object reaching and picking, with a picking rate always greater than \textbf{80\%}. 

Regarding the success rate, the highest average value was achieved with the final MOSAIC-COD module, which across four tested tasks showed an average success rate of \textbf{90.13\%}. This highlights how including information about the object's location improves system robustness. 

In the multi-task scenario, it was observed that the use of object priors also resulted in a more successful and robust system. The MOSAIC-COD system achieved an average success rate of \textbf{79.24\%}, a significant improvement compared to the baseline (46.01\%). However, there is still room for improvement, particularly in balancing performance across different tasks.

In conclusion, the proposed system was also validated in a real-world scenario (Chapter \ref{ch:real_world_application}). Specifically, MOSAIC-CTOD and MOSAIC-COD were tested in a restricted single-task, multi-variation scenario with two different configurations. The first configuration involved training the system from scratch, while the second used a system finetuned from a model pre-trained in simulation. Results show that the finetuned systems outperformed the systems trained from scratch, demonstrating the potential of leveraging different data domains to initialize the system, which can then be finetuned for the target context.

It was generally observed that the C(T)OD module could predict the location of target objects based on demonstrations from a simulated agent, indicating that the system can interpret task execution from demonstrations by different agents and in different domains. This opens interesting possibilities for future development, including the use of human-based demonstrations.

Regarding the final control policy, the highest success rate was achieved with the MOSAIC-COD module, reaching \textbf{55.00\%}. While this success rate is lower compared to the corresponding results in simulation, several factors need to be considered, including the characteristics of the dataset and the real-world challenges. Notably, the system consistently reached the target object with a reaching rate of \textbf{86.67\%}, indicating that the COD can reliably inform the control module about the target object's position. The primary source of errors involved collisions with the target object, mostly due to noisy and low-quality data. These issues could be mitigated by enhancing the robot's perceptual capabilities, such as using depth information or a camera mounted on the robot.

% Implications of the findings:
% 1. Sistema che è in grado di replicare un task eseguito da un altro agente in un dominio diverso, aprendo quindi la possibilità di avere sistemi robotici in grado di imparare l'esecuzioni di nuovi task a partire da dimostrazioni video di altri agenti, anche in domini diversi.
% 2. Questo attraverso un approccio modulare, dove 
\smalltodo{Commento  Benoit: the conclusions could include more detailed perspectives on future work and broader implications of the findings.}
\smalltodo{Commento  George: the discussion could benefit from a deeper exploration of potential limitations and future research directions, such as scalability to more complex environments or broader task sets.}

\updated{
In conclusion, the methods discussed in this thesis represent a significant advancement toward developing a system capable of replicating tasks performed by another agent, even under varying environmental conditions (e.g., differences in object positions between the demonstrator and the agent) and in challenging scenarios involving multiple similar objects, where the semantic role of each object (e.g., target or distractor) is dynamically defined at runtime by the demonstrator. The proposed approach, based on a modular architectures, has demonstrated its effectiveness. Specifically, by first localizing the target objects, not only is the control problem simplified, since the network learns a simpler mapping from pixel coordinates to spatial coordinates, but also the interpretability of the system is enhanced. The predicted bounding box serves as \textit{``human-readable"} information, which can be used to estimate where the robot is going to move.

Generally speaking, such capabilities are highly desirable when the goal is to develop general-purpose robotic platforms that are capable not only of performing multiple tasks but also of learning to execute novel tasks through an \textit{intuitive programming paradigm} based on \textit{video demonstrations}. For instance, in a highly adaptable manufacturing scenario, a human operator could demonstrate a new assembly procedure by recording just a few executions. The robot could then be tasked with learning the procedure and replicating it across various configurations.

Building on the proposed modular approach, future works could focus on the following aspects:
\begin{itemize}
    \item \textbf{Scalability to more complex objects}. Once the modular approach has demonstrated its effectiveness with simple-shaped objects (e.g., boxes and bins), the system can be enhanced to handle more complex objects, such as those with irregular shapes or multiple parts that need to be assembled. To achieve this, more advanced \textit{conditioned object detection} techniques can be employed, drawing inspiration from novel Visual-Question Answering models \cite{wen2024object}, where Visual-Language Architectures are used to selectively localize the objects of interest based on a given query. 

    \item \textbf{Generalization to multi-step tasks}. In industrial settings, tasks are often composed of multiple steps, requiring the agent to perform a sequence of actions to complete it. To address this challenge, a novel methodological approach is needed to decompose the task into sub-tasks. One possible approach is to use a Graph-Based scene representation, where each node corresponds to an object and the edges represent the relationships between them. This representation can then be used to model the current task state and facilitate learning a policy that generates the actions needed to transition from one state to another, i.e., determining how the current graph must be modified to achieve the desired goal of the $i$-th sub-task. 
\end{itemize}
}


% This kind of ability is achieved by leveraging a \textit{\textbf{modular approach}}, where first the target objects are localized within the agent space, and then the control policy generates the actions needed to manipulate the object and solve the demonstrated tasks. Improving not only the success-rate, as extensively demonstrated in the results, but also the interpretability of the system, since the predicted bounding-box can be seen as a \textit{``human-readable"} information, which can be used to estimate where the robot is going to move.  

% Overall, 
%\addcontentsline{toc}{chapter}{Conclusions}
%\include{Conclusions}
%\thispagestyle{empty}


% ---------- Appendix -------------- 
% Togliere il commento alle prossime righe qualora voglia inserire l'Appendice
%
%\renewcommand{\chaptermark}[1]{\markboth{Appendix \thechapter.\ #1}{}}
%\appendix
%\include{annex/Annex1}
%\thispagestyle{empty}
%\include{app2}
%\thispagestyle{empty}
%\include{app3}
%\thispagestyle{empty}

\small
% ---------- Bibiography --------------
\addcontentsline{toc}{chapter}{Bibliography}
%\bibliographystyle{mystyle}  % Tipo  - da B. D'Auria (da usare con ``natbib'')
\bibliographystyle{IEEEtran}
%\bibliographystyle{amsalpha} % Stile bibliografia [iniziali autori + 2 cifre per l'anno] (da usare senza ``natbib'')
%\bibliographystyle{plainnat} % Stile della bibliografia [autore (et al.), anno] - usare \citep{}.
\bibliography{thesis}     % Comando per inserire il file della bibliografia (ex. PhDThesis.bib)

% \listoffigures
% \listoftables

\end{document}
